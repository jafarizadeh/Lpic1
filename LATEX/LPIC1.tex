\documentclass[12pt,a4paper]{report}
\usepackage[margin=1in]{geometry}
\usepackage[T1]{fontenc}
\usepackage[utf8]{inputenc}
\usepackage[hidelinks]{hyperref}
\usepackage{amsmath}
\usepackage{amssymb}
\usepackage{enumerate}
\usepackage{setspace}
\usepackage{listings}
\usepackage{xcolor}

% Configuration pour les listings de code
\definecolor{codebg}{RGB}{245,245,245}
\lstset{
    language=bash,
    backgroundcolor=\color{codebg},
    basicstyle=\ttfamily\small,
    frame=single,
    breaklines=true,
    columns=fullflexible,
    keywordstyle=\color{blue},
    commentstyle=\color{gray},
    stringstyle=\color{orange},
    showstringspaces=false
}
\setstretch{1.15}

\begin{document}

%-------------------------------------------------------
% Title Page
%-------------------------------------------------------
\begin{titlepage}
    \centering
    \vspace*{3cm}
    {\Huge \textbf{LPIC-1 Exam Workbook}}\\
    \vspace{1cm}
    {\large \textit{A Chapter-by-Chapter Syllabus with Practice Questions}}\\
    \vfill
    {\large \textbf{Version 1.0}}\\
    \vspace{2cm}
    \vfill
    \textbf{Author:} Mehdi JAFARIZADEH \\
    \textbf{Date:} January 1, 2025
    \vspace{2cm}
\end{titlepage}

\tableofcontents
\newpage

%=======================================================
% TOPIC 101: SYSTEM ARCHITECTURE
%=======================================================
\chapter{Topic 101: System Architecture}

\section*{101.1 Determine and Configure Hardware Settings}
\addcontentsline{toc}{section}{101.1 Determine and Configure Hardware Settings}

\textbf{Reference to LPI Objectives:}  
\begin{itemize}
    \item \textbf{LPIC-1 v5, Exam 101, Objective 101.1}
    \item \textbf{Weight:} 2
\end{itemize}

\subsection*{Key Knowledge Areas}
\begin{itemize}
    \item Enabling/disabling integrated peripherals (BIOS/UEFI).
    \item Identifying different types of mass storage devices.
    \item Determining hardware resources for devices (IRQ, DMA, etc.).
    \item Using tools (\texttt{lsusb}, \texttt{lspci}, \texttt{lsmod}) for hardware inspection.
    \item Manipulating USB devices.
    \item Understanding \texttt{sysfs}, \texttt{udev}, and \texttt{dbus} concepts.
\end{itemize}

\subsection*{Important Files, Terms, and Utilities}
\begin{itemize}
    \item \textbf{/sys/}
    \item \textbf{/proc/}
    \item \textbf{/dev/}
    \item \texttt{modprobe}
    \item \texttt{lsmod}
    \item \texttt{lspci}
    \item \texttt{lsusb}
\end{itemize}

\section*{Lesson Overview}

Modern computers rely on standards for firmware and hardware interaction. On x86 platforms, the firmware could be traditional \textbf{BIOS} or newer \textbf{UEFI}. Both allow for configuring hardware resources (e.g., integrated peripherals, IRQs, DMA settings) even before the operating system loads.

Once Linux is running, device detection and configuration rely on the kernel and support from user-space utilities such as \texttt{lspci}, \texttt{lsusb}, \texttt{lsmod}, and various pseudo-filesystems in \textbf{/proc} and \textbf{/sys}.

\subsection*{1. BIOS and UEFI Configuration}
\begin{itemize}
    \item \textbf{Accessing Firmware:} Typically press \texttt{Del}, \texttt{F2}, or \texttt{F12} at startup.
    \item \textbf{Common Configurations:}
    \begin{itemize}
        \item Enable/disable integrated peripherals (USB ports, onboard audio, etc.).
        \item Set boot order and define the primary device for the bootloader.
        \item Adjust CPU features or RAM parameters if needed.
    \end{itemize}
    \item \textbf{Impact:} Misconfiguration (e.g., wrong boot device) can prevent the OS from loading.
\end{itemize}

\subsection*{2. Device Detection in Linux}
\begin{itemize}
    \item \textbf{Goal:} Match hardware parts to the correct driver (\textbf{kernel module}).
    \item \textbf{Basic Workflow:}
    \begin{enumerate}
        \item \textbf{Check if hardware is detected} (e.g., \texttt{lspci}, \texttt{lsusb}).
        \item \textbf{Verify if a driver is loaded} (e.g., \texttt{lsmod}, \texttt{lspci -k}).
        \item \textbf{Confirm functionality} via logs, testing, or additional tools.
    \end{enumerate}
\end{itemize}

\subsection*{3. Commands for Hardware Inspection}

\begin{enumerate}
    \item \textbf{\texttt{lspci}}
    \begin{itemize}
        \item Lists PCI devices (graphics cards, network interfaces, etc.).
        \item Use \texttt{-v} for more detail and \texttt{-k} to see which kernel modules are in use.
        \item Example:
        \begin{lstlisting}[language=bash]
lspci -s 04:02.0 -v
lspci -s 01:00.0 -k
        \end{lstlisting}
    \end{itemize}

    \item \textbf{\texttt{lsusb}}
    \begin{itemize}
        \item Lists USB devices (keyboards, mice, USB hubs, etc.).
        \item Use \texttt{-v} for verbose output and \texttt{-d <vendor:product>} to focus on a specific device.
        \item Example:
        \begin{lstlisting}[language=bash]
lsusb -v -d 1781:0c9f
lsusb -t  # Show devices in a tree structure
        \end{lstlisting}
    \end{itemize}

    \item \textbf{\texttt{lsmod}}
    \begin{itemize}
        \item Shows loaded kernel modules.
        \item Columns: \textbf{Module}, \textbf{Size}, \textbf{Used by} (dependency information).
        \item Example:
        \begin{lstlisting}[language=bash]
lsmod | grep snd_hda_intel
        \end{lstlisting}
    \end{itemize}

    \item \textbf{\texttt{modprobe}}
    \begin{itemize}
        \item Loads or unloads modules (with dependencies).
        \item \texttt{modprobe -r <module>} removes a module if not in use.
        \item \texttt{modinfo <module>} shows module details (author, license, parameters, etc.).
        \item Configuration files in \texttt{/etc/modprobe.d/} can blacklist or set module parameters.
    \end{itemize}
\end{enumerate}

\subsection*{4. Hardware Information Files}
\begin{itemize}
    \item \textbf{/proc} (pseudo-filesystem for processes and hardware info)
    \begin{itemize}
        \item \texttt{/proc/cpuinfo}, \texttt{/proc/interrupts}, \texttt{/proc/ioports}, \texttt{/proc/dma}
    \end{itemize}
    \item \textbf{/sys} (\texttt{sysfs} for device and kernel data)
    \item \textbf{/dev} (device files)
    \begin{itemize}
        \item Each entry represents a device (e.g., \texttt{/dev/sda1}, \texttt{/dev/fd0}).
        \item \textbf{\texttt{udev}} dynamically creates/removes these files as devices connect or disconnect.
    \end{itemize}
\end{itemize}

\subsection*{5. Storage Devices}
\begin{itemize}
    \item \textbf{Block Devices:} Accessed in fixed-size blocks (hard disks, SSDs, etc.).
    \item \textbf{Naming Conventions:}
    \begin{itemize}
        \item Newer kernels use \texttt{sd} prefix for most disks; partitions are numbered (\texttt{/dev/sda1}).
        \item \textbf{IDE} devices also appear as \texttt{sd} on modern kernels
        \item \textbf{NVMe} devices get names like \texttt{/dev/nvme0n1p1}.
        \item \textbf{SD Cards} often appear as \texttt{/dev/mmcblk0p1}.
    \end{itemize}

    \item \textbf{Hotplug and Coldplug:}
    \begin{itemize}
        \item \textbf{Hotplug:} device recognized after boot (e.g., USB).
        \item \textbf{Coldplug:} device recognized during boot (built-in or already connected).
    \end{itemize}
\end{itemize}

\section*{Workbook Exercises}

\begin{enumerate}
    \item \textbf{Accessing BIOS/UEFI}
    \begin{itemize}
        \item Reboot a test machine and enter BIOS/UEFI.
        \item Locate the sections that let you enable/disable integrated peripherals.
        \item Identify the menu where boot order is set.
    \end{itemize}

    \item \textbf{Listing Hardware}
    \begin{itemize}
        \item On a Linux system, run \texttt{lspci -k}.
        \begin{itemize}
            \item Identify which driver is used by the video card.
        \end{itemize}
        \item Run \texttt{lsusb -t}.
        \begin{itemize}
            \item Check which USB driver modules are in use (e.g., \texttt{btusb, usbhid}).
        \end{itemize}
    \end{itemize}

    \item \textbf{Exploring /proc and /sys}
    \begin{itemize}
        \item View CPU details with \texttt{cat /proc/cpuinfo}.
        \item Inspect interrupts with \texttt{cat /proc/interrupts}.
        \item Explore \texttt{/sys/class} and \texttt{/sys/block} to see how devices are represented.
    \end{itemize}

    \item \textbf{Managing Kernel Modules}
    \begin{itemize}
        \item Use  \texttt{lsmod} to list all loaded modules.
        \item Pick a module (e.g., a sound driver) and unload it with \texttt{sudo modprobe -r <module>}.
        \begin{itemize}
            \item Check if removal is allowed (the module should not be in use).
        \end{itemize}
        \item Use \texttt{modinfo -p <module>} to see possible parameters, and note how you might apply them in \texttt{/etc/modprobe.d/}.
        
    \end{itemize}

    \item \textbf{Blacklisting a Module}
    \begin{itemize}
        \item Create a test file in \texttt{/etc/modprobe.d/} to blacklist an unwanted module (e.g., \texttt{nouveau}).
        \item Reboot and confirm it is not loaded by checking \texttt{lsmod}.
    \end{itemize}
\end{enumerate}

\section*{Summary}
\begin{itemize}
    \item Modern systems rely on firmware (BIOS/UEFI) for initial hardware configuration.
    \item Linux identifies devices via kernel modules; tools like \texttt{lspci}, \texttt{lsusb}, \texttt{lsmod}, and \texttt{modprobe} allow you to inspect and manage hardware.
    \item \texttt{/proc} and \texttt{/sys} provide detailed, real-time system information, while \texttt{udev} dynamically manages device nodes in \texttt{/dev}.
    \item Storage device naming conventions follow standard patterns such as \texttt{sd}, \texttt{nvme}, \texttt{mmcblk}, and partition numbers like \texttt{/dev/sda1}.
    \item Understanding how to enable/disable devices, load/unload modules, and explore hardware information files is crucial for effective system administration and LPIC-1 success.
\end{itemize}




%-------------------------------------------------------
% Multiple-Choice Questions (101.1)
%-------------------------------------------------------
\section*{Multiple-Choice Questions for 101.1}

\begin{enumerate}[1.]
\item When trying to enable or disable motherboard-integrated peripherals, which component of the system is typically used?
  \begin{enumerate}[A)]
    \item The BIOS or UEFI configuration utility
    \item The Linux kernel’s initrd
    \item The \texttt{/boot} partition
    \item The \texttt{lsusb} command

  \end{enumerate}

\item Which command lists devices currently connected to the PCI bus?
  \begin{enumerate}[A)]
    \item \texttt{modprobe}
    \item \texttt{lsmod}
    \item \texttt{lspci}
    \item \texttt{lshw}
  \end{enumerate}

\item Which of the following commands helps you list USB devices in a tree-like hierarchy?
  \begin{enumerate}[A)]
    \item lsusb -a
    \item lsusb -s
    \item lsusb -f
    \item lsusb -t
  \end{enumerate}

\item To remove a kernel module (along with its dependencies) while the system is running, which command should be used?
  \begin{enumerate}[A)]
    \item modinfo -r
    \item modprobe -r
    \item rmmod --all
    \item lsmod -r
  \end{enumerate}

\item On modern Linux systems, SATA disks are generally identified as which kind of device name?
  \begin{enumerate}[A)]
    \item /dev/sdX
    \item /dev/hdX
    \item /dev/nvmeXnY
    \item /dev/fdX
  \end{enumerate}

\item Which file below would you edit to permanently blacklist a problematic kernel module such that it doesn’t load automatically?
  \begin{enumerate}[A)]
    \item /etc/rc.local
    \item /etc/modprobe.d/blacklist.conf
    \item /boot/grub/grub.cfg
    \item /proc/blacklist/modules
  \end{enumerate}

\item Which pseudo-filesystem is most specifically devoted to storing device and kernel data related to hardware?
  \begin{enumerate}[A)]
    \item /dev
    \item /proc
    \item /sys
    \item /home
  \end{enumerate}

\item Which command line will show a specific USB device’s verbose information using its vendor:product ID (e.g., 1781:0c9f)?
  \begin{enumerate}[A)]
    \item lsusb -d 1781:0c9f -v
    \item lsusb -p 1781:0c9f -v
    \item lsusb -i 1781:0c9f
    \item lsusb -v -s 01:02
  \end{enumerate}

\item In the output of lsmod, the “Used by” column indicates:
  \begin{enumerate}[A)]
    \item the file size of the module on disk
    \item the user-level applications that installed the module
    \item the modules or processes depending on that module
    \item kernel version compatibility for that module
  \end{enumerate}

\item If you need to confirm which kernel driver is in use by a particular PCI device, which \texttt{lspci} option combination is most helpful on recent distributions?
  \begin{enumerate}[A)]
    \item lspci -m
    \item lspci -k
    \item lspci -D
    \item lspci --driver
  \end{enumerate}

\item What does the output of \texttt{lsusb -t} specifically highlight that differs from plain \texttt{lsusb}?
  \begin{enumerate}[A)]
    \item The exact partition layout of attached USB drives
    \item A hierarchical (tree-like) representation of USB devices and drivers
    \item The SCSI ID mappings of USB-attached devices
    \item A summary of device’s kernel modules only
  \end{enumerate}

\item Which best describes the function of the \texttt{modinfo} command?
  \begin{enumerate}[A)]
    \item It removes the specified module from the kernel
    \item It displays all processes currently using a kernel module
    \item It lists detailed information about a specified module, including parameters
    \item It inserts the specified module and resolves dependencies
  \end{enumerate}

\item What is the role of \texttt{udev} on a modern Linux system?
  \begin{enumerate}[A)]
    \item It is a pseudo-filesystem used to track hardware devices in \texttt{/sys}
    \item It permanently stores device drivers in \texttt{/boot}
    \item It manages device nodes in \texttt{/dev}, handling hotplug/coldplug events
    \item It only configures CPU frequency scaling
  \end{enumerate}

\item Which file inside \texttt{/proc} would you inspect to see how many interrupts have occurred for each device?
  \begin{enumerate}[A)]
    \item \texttt{/proc/ioports}
    \item \texttt{/proc/dma}
    \item \texttt{/proc/cpuinfo}
    \item \texttt{/proc/interrupts}
  \end{enumerate}

\item If a device is recognized by the kernel but not functioning correctly, which of the following is the most likely underlying cause?
  \begin{enumerate}[A)]
    \item The BIOS is not set to read the device’s firmware
    \item The associated kernel module (driver) is not loaded or is misconfigured
    \item The CPU lacks the required SSE instruction set
    \item The device was not assigned a correct IRQ in the \texttt{/etc/fstab}
  \end{enumerate}

\item Which file is typically used to pass persistent module load options like \texttt{options nouveau modeset=0}?
  \begin{enumerate}[A)]
    \item \texttt{/etc/udev/rules.d/99-custom.rules}
    \item \texttt{/proc/meminfo}
    \item \texttt{/etc/modprobe.d/<module>.conf}
    \item \texttt{/etc/modules-load.d/module.options}
  \end{enumerate}

\item What is the main purpose of SysFS (\texttt{/sys}) in a Linux system?
  \begin{enumerate}[A)]
    \item Stores process information like CPU usage
    \item Holds user configuration data for \texttt{/home}
    \item Exports device and driver information from the kernel to user space
    \item Contains scripts to mount all system filesystems
  \end{enumerate}

\item Which command is most appropriate for listing all currently loaded kernel modules?
  \begin{enumerate}[A)]
    \item \texttt{ls -la /lib/modules/\$(uname -r)}
    \item \texttt{depmod -a}
    \item \texttt{lsmod}
    \item \texttt{insmod}
  \end{enumerate}

\item To selectively unload the \texttt{snd-hda-intel} module along with related dependent modules, which command would you use?
  \begin{enumerate}[A)]
    \item \texttt{modinfo snd-hda-intel --remove}
    \item \texttt{lsmod --unload snd-hda-intel}
    \item \texttt{depmod -r snd-hda-intel}
    \item \texttt{modprobe -r snd-hda-intel}
  \end{enumerate}

\item If you see a disk labeled as \texttt{/dev/mmcblk0p1}, which type of physical device is this likely referring to?
  \begin{enumerate}[A)]
    \item A SATA SSD
    \item An older IDE HDD
    \item An SD card or MMC device
    \item A USB DVD drive
  \end{enumerate}
\end{enumerate}

%-------------------------------------------------------
% Fill-in-the-Blank Questions (101.1)
%-------------------------------------------------------
\section*{Fill-in-the-Blank Questions for 101.1}

\begin{enumerate}[1.]

\item The older firmware commonly used before the UEFI standard is called \underline{\hspace{2cm}}.

\item The \underline{\hspace{2cm}} command lists all kernel modules currently loaded into the system.

\item A kernel module responsible for controlling hardware in Linux is often referred to as a \underline{\hspace{2cm}}.

\item The Linux subsystem that manages device node creation in \texttt{/dev} and handles hotplug/coldplug events is called \underline{\hspace{2cm}}.

\item The special, memory-based filesystem used for storing process and hardware information is the \underline{\hspace{2cm}} directory.

\item To configure boot device priority and enable or disable onboard peripherals, a user must typically access the \underline{\hspace{2cm}} or UEFI setup utility.

\item In Linux, disks commonly appear under \texttt{/dev} as \underline{\hspace{2cm}} devices (e.g., \texttt{/dev/sda}, \texttt{/dev/sdb}) on modern systems.

\item The \underline{\hspace{2cm}} command is used to insert or remove kernel modules and their dependencies.

\item When blacklisting a kernel module to prevent it from loading automatically, the configuration file is often placed in \underline{\hspace{2cm}}.

\item To see a hierarchical (tree-like) view of USB devices and the drivers handling them, you can run \underline{\hspace{2cm}} with the \texttt{-t} option.
 
\end{enumerate}

%-------------------------------------------------------
% (Placeholders for Other Sections)
%-------------------------------------------------------
\newpage

\section*{101.2 Boot the System}
\addcontentsline{toc}{section}{101.2 Boot the System}

\textbf{Reference to LPI Objectives:}  
\begin{itemize}
    \item \textbf{LPIC-1 v5, Exam 101, Objective 101.2}
    \item \textbf{Weight:} 3
\end{itemize}

\subsection*{Key Knowledge Areas}
\begin{itemize}
    \item Providing common bootloader commands and kernel options at boot.
    \item Understanding the boot sequence (BIOS/UEFI through OS startup).
    \item Familiarity with SysVinit, systemd, and Upstart.
    \item Checking boot events and logs (\texttt{dmesg}, \texttt{journalctl}).
\end{itemize}

\subsection*{Important Files, Terms, and Utilities}
\begin{itemize}
    \item \textbf{dmesg}
    \item \textbf{journalctl}
    \item \textbf{BIOS} / \textbf{UEFI}
    \item \textbf{bootloader} (GRUB)
    \item \textbf{kernel}
    \item \textbf{initramfs}
    \item \textbf{init} (SysVinit, systemd, Upstart)
    \item \textbf{/proc/cmdline}
    \item \textbf{/var/log/}
\end{itemize}

\section*{Lesson Overview}

Booting a Linux system involves multiple stages:
\begin{enumerate}
    \item \textbf{Firmware Load:} BIOS or UEFI initializes basic hardware.
    \item \textbf{Bootloader:} Typically \textbf{GRUB}, which locates and loads the kernel.
    \item \textbf{Kernel \& initramfs:} Kernel initializes hardware and reads modules from the initramfs.
    \item \textbf{System Initialization:} \textbf{init} (SysVinit, systemd, Upstart) starts services and completes the boot process.
\end{enumerate}

\subsection*{1. BIOS vs. UEFI}
\begin{itemize}
\item\textbf{BIOS}  
\begin{itemize}
    \item Uses MBR (first 512 bytes) to load boot code (GRUB stage 1).
    \item Relies on a DOS partition scheme and the Master Boot Record.
    \item Boots the second stage of the bootloader, which in turn loads the kernel.
\end{itemize}

\item\textbf{UEFI}
\begin{itemize}
    \item Looks at entries in \textbf{NVRAM} to find an \textbf{EFI application} (usually GRUB).
    \item Loads the EFI application from a dedicated \textbf{EFI System Partition (ESP)}.
    \item Supports \textbf{Secure Boot} to allow only signed EFI applications.
\end{itemize}
\end{itemize}

\subsection*{2. Bootloader (GRUB)}
\begin{itemize}
    \item Presents a menu of installed kernels or operating systems.
    \item Enables passing \textbf{kernel parameters} (e.g., \texttt{quiet}, \texttt{acpi=off}, \texttt{root=/dev/sdaX}, etc.).
    \item Kernel parameters can be made persistent in \texttt{/etc/default/grub} and then updated with:
\end{itemize}

\begin{lstlisting}[language=bash]
grub-mkconfig -o /boot/grub/grub.cfg
\end{lstlisting}

\begin{itemize}
    \item Current kernel parameters are visible in \texttt{/proc/cmdline}.
\end{itemize}

\subsection*{3. System Initialization}

\begin{enumerate}
    \item \textbf{initramfs}
    \begin{itemize}
        \item Temporary root filesystem with essential drivers/modules.
        \item Lets the kernel mount the actual root filesystem.
    \end{itemize}
    \item \textbf{init}
    \begin{itemize}
        \item The “first process” in user space.
        \item \textbf{SysVinit:} uses \textbf{runlevels} (0–6).
        \item \textbf{systemd:} uses \textbf{targets}, concurrency, D-Bus, cgroups. Most common in modern distros.
        \item \textbf{Upstart:} parallel boot focusing on faster startup. Largely replaced by systemd.
    \end{itemize}
\end{enumerate}

\subsection*{4. Boot Logging and Inspection}
\begin{itemize}
    \item \textbf{dmesg}
    \begin{itemize}
        \item Displays the \textbf{kernel ring buffer} (including boot messages).
        \item Clears with \texttt{dmesg --clear}.
    \end{itemize}
    \item \textbf{journalctl}
    \begin{itemize}
        \item Systemd-based logging tool.
        \item \texttt{journalctl -b} shows current boot messages.
        \item \texttt{journalctl --list-boots} lists previous boots.
    \end{itemize}
    \item Traditional log files also found in \texttt{/var/log/}, e.g., \texttt{/var/log/messages} or \texttt{/var/log/syslog}.
\end{itemize}

\section*{Workbook Exercises}

\begin{enumerate}
    \item \textbf{Firmware Awareness}
    \begin{itemize}
        \item Reboot a test machine.
        \item Determine whether it uses \textbf{BIOS} or \textbf{UEFI}.
        \item In BIOS: Find where the boot order is set.
        \item In UEFI: Locate the ESP partition and explore contents if possible.
    \end{itemize}
    \item \textbf{GRUB Menu and Kernel Parameters}
    \begin{itemize}
        \item Boot into the GRUB menu by pressing \textbf{Shift} (BIOS) or \textbf{Esc} (UEFI).
        \item Edit a menu entry to add or change a kernel parameter (e.g., \texttt{init=/bin/bash}, \texttt{acpi=off}).
        \item After boot, check \texttt{/proc/cmdline} to confirm your changes.
    \end{itemize}
    \item \textbf{System Initialization Tools}
    \begin{itemize}
        \item Identify which init system your distribution uses (\texttt{ps -p 1 -o comm=}).
        \item If it’s systemd, compare output of these commands:
\end{itemize}

\begin{lstlisting}[language=bash]
systemctl list-units --type=service
journalctl -b
\end{lstlisting}

\begin{itemize}
        \item If SysVinit is present, inspect runlevel scripts in \texttt{/etc/rc.d/} or \texttt{/etc/init.d/}.
    \end{itemize}
    \item \textbf{Inspecting Boot Logs}
    \begin{itemize}
        \item Run \texttt{dmesg | less} to page through the kernel ring buffer.
        \item If using systemd, run \texttt{journalctl --list-boots} to see previous boots.
        \item View the logs for the current boot with \texttt{journalctl -b 0}.
    \end{itemize}
    \item \textbf{initramfs Exploration}
    \begin{itemize}
        \item Locate your initramfs file (commonly in \texttt{/boot}, e.g., \texttt{initramfs-<version>.img}).
        \item List contents using \texttt{lsinitrd} or \texttt{unmkinitramfs} (may require additional packages).
        \item Identify which modules are included for the root filesystem.
    \end{itemize}
\end{enumerate}

\section*{Summary}
\begin{itemize}
    \item The boot process starts with \textbf{BIOS/UEFI} firmware, which calls \textbf{GRUB} to load the \textbf{kernel}.
    \item The \textbf{initramfs} contains essential modules and mounts the real root filesystem.
    \item An \textbf{init} system (SysVinit, systemd, Upstart) then starts daemons and services.
    \item \textbf{dmesg} and \textbf{journalctl} provide essential logs for troubleshooting.
    \item Understanding these steps ensures you can troubleshoot common startup issues and manage kernel parameters effectively.
\end{itemize}


\newpage

\section*{101.3 Change Runlevels / Boot Targets and Shutdown or Reboot System}
\addcontentsline{toc}{section}{101.3 Change Runlevels / Boot Targets and Shutdown or Reboot System}

\textbf{Reference to LPI Objectives:}  
\begin{itemize}
    \item \textbf{LPIC-1 v5, Exam 101, Objective 101.3}
    \item \textbf{Weight:} 3
\end{itemize}

\subsection*{Key Knowledge Areas}
\begin{itemize}
    \item Setting the default runlevel/boot target.
    \item Changing between runlevels/targets, including single-user mode.
    \item Shutting down and rebooting from the command line.
    \item Alerting users before switching runlevels/boot targets or major system events.
    \item Properly terminating processes.
    \item Awareness of \textbf{acpid} (power management).
\end{itemize}

\subsection*{Important Files, Terms, and Utilities}
\begin{itemize}
    \item \textbf{/etc/inittab} (SysVinit)
    \item \textbf{shutdown}
    \item \textbf{init}, \textbf{telinit} (SysVinit)
    \item \textbf{/etc/init.d/} (SysVinit scripts)
    \item \textbf{systemd}, \textbf{systemctl}
    \item \textbf{/etc/systemd/}, \textbf{/usr/lib/systemd/}
    \item \textbf{wall} (send messages to all logged-in users)
\end{itemize}

\section*{Lesson Overview}

Linux can operate in different “states” or “modes” called \textbf{runlevels} in SysVinit or \textbf{targets} in systemd. Being able to switch between them and perform system shutdowns or reboots is essential for system administration.

\section*{1. SysVinit Runlevels}

\subsection*{1. Runlevels}
\begin{itemize}
    \item \textbf{0} – Shutdown
    \item \textbf{1 (single), s} – Single-user (maintenance) mode
    \item \textbf{2, 3, 4} – Multi-user modes (3 is typical, 2/4 vary by distro)
    \item \textbf{5} – Multi-user plus graphical mode
    \item \textbf{6} – Reboot
\end{itemize}

\subsection*{2. Configuration}
\begin{itemize}
    \item \textbf{/etc/inittab} defines default runlevel (\texttt{id:x:initdefault:})
    \item Each runlevel has a dedicated directory: \textbf{/etc/rc0.d/}, \textbf{/etc/rc1.d/}, etc.
    \item Scripts in \textbf{/etc/init.d/} are symlinked to these runlevel directories.
    \begin{itemize}
        \item Names starting with \textbf{S} start services.
        \item Names starting with \textbf{K} kill services.
    \end{itemize}
\end{itemize}

\subsection*{3. Switching Runlevels}
\begin{itemize}
    \item \textbf{init} or \textbf{telinit} commands set the current runlevel.
    \item \texttt{telinit 1}: move to runlevel 1 (maintenance mode).
    \item \texttt{runlevel}: shows current and previous runlevel (e.g., \texttt{N 3} means currently 3 and no prior change).
\end{itemize}

\subsection*{4. Reloading \textbf{/etc/inittab}}
\begin{itemize}
    \item After editing \textbf{/etc/inittab}, run \texttt{telinit q} to re-read the config.
\end{itemize}

\section*{2. systemd Targets}

\subsection*{1. systemd Concepts}
\begin{itemize}
    \item \textbf{Units} represent services, sockets, devices, mounts, automounts, targets, and snapshots.
    \item \textbf{systemctl} is the primary command to manage these units (start, stop, enable, etc.).
\end{itemize}

\subsection*{2. Targets}
\begin{itemize}
    \item systemd uses \textbf{targets} to group units. Examples:
    \begin{itemize}
        \item \textbf{multi-user.target} – analogous to runlevel 3 (no GUI).
        \item \textbf{graphical.target} – analogous to runlevel 5 (GUI mode).
    \end{itemize}
    \item You can isolate a target:
\end{itemize}

\begin{lstlisting}[language=bash]
systemctl isolate multi-user.target
\end{lstlisting}

\subsection*{3. Default Target}
\begin{itemize}
    \item Change default target:
\end{itemize}

\begin{lstlisting}[language=bash]
systemctl set-default multi-user.target
\end{lstlisting}

\begin{itemize}
    \item View current default:
\end{itemize}

\begin{lstlisting}[language=bash]
systemctl get-default
\end{lstlisting}

\begin{itemize}
    \item Avoid pointing to \textbf{shutdown.target} or \textbf{reboot.target}.
\end{itemize}

\subsection*{4. Service Management}
\begin{itemize}
    \item \texttt{systemctl start/stop/restart} \texttt{<service>.service}
    \item \texttt{systemctl enable/disable} \texttt{<service>.service} (at boot)
    \item \texttt{systemctl status} \texttt{<service>.service}
    \item \texttt{systemctl list-unit-files --type=service} – list available services
    \item \texttt{systemctl list-units --type=service} – list loaded/running services
\end{itemize}

\subsection*{5. Power Management}
\begin{itemize}
    \item \texttt{systemctl suspend}, \texttt{systemctl hibernate}
    \item For finer power-event control (e.g., lid close), \textbf{acpid} can be used instead of systemd’s built-in power management.
\end{itemize}

\section*{3. Upstart (Historical)}

\begin{enumerate}
    \item \textbf{Upstart} was used in older Ubuntu-based systems before switching to systemd.
    \item \textbf{Commands}:
    \begin{itemize}
        \item \texttt{initctl list} – list services and states
        \item \texttt{start / stop / status <service>} – control services
        \item Initialization scripts: \textbf{/etc/init/}
    \end{itemize}
    \item \texttt{runlevel} and \texttt{telinit} still work for basic runlevel tasks.
\end{enumerate}

\section*{4. Shutting Down and Rebooting}

\subsection*{1. shutdown}
\begin{itemize}
    \item Syntax:
\end{itemize}

\begin{lstlisting}[language=bash]
shutdown [option] time [message]
\end{lstlisting}

\begin{itemize}
    \item \textbf{time} can be \texttt{now}, \texttt{+m} (minutes from now), or \texttt{hh:mm} (absolute time).
    \item Common options:
    \begin{itemize}
        \item \textbf{-h} – halt/power off
        \item \textbf{-r} – reboot
    \end{itemize}
    \item Notifies logged-in users and prevents new logins (unless overridden).
\end{itemize}

\subsection*{2. systemctl (systemd)}
\begin{itemize}
    \item \texttt{systemctl reboot} – reboot system
    \item \texttt{systemctl poweroff} – power off system
    \item Sometimes distros alias \texttt{poweroff} and \texttt{reboot} to systemd commands.
\end{itemize}

\subsection*{3. wall}
\begin{itemize}
    \item Sends a message to all logged-in users’ terminals (similar to \texttt{shutdown}’s broadcast).
    \item Useful for manual warnings before switching to single-user mode or shutting down.
\end{itemize}

\section*{Workbook Exercises}

\begin{enumerate}
    \item \textbf{Identify Your Init System}
    \begin{itemize}
        \item Run \texttt{ps -p 1 -o comm=} to see if your system uses \textbf{systemd}, \textbf{init}, or \textbf{Upstart}.
    \end{itemize}
    \item \textbf{Practice Switching Runlevels (SysV)}
    \begin{itemize}
        \item On a SysVinit system, edit \textbf{/etc/inittab} to set default runlevel to \textbf{3}.
        \item Run \texttt{telinit q} and verify with \texttt{runlevel}.
        \item Switch to single-user mode: \texttt{telinit 1}.
    \end{itemize}
    \item \textbf{Practice Managing systemd Targets}
    \begin{itemize}
        \item Show the current default target: \texttt{systemctl get-default}.
        \item Switch from \textbf{graphical.target} to \textbf{multi-user.target} using:
\end{itemize}

\begin{lstlisting}[language=bash]
systemctl isolate multi-user.target
\end{lstlisting}

\begin{itemize}
        \item Confirm the change: \texttt{systemctl status multi-user.target}.
    \end{itemize}
    \item \textbf{Service Control with systemd}
    \begin{itemize}
        \item Start a service (e.g., \texttt{ssh.service}):
\end{itemize}

\begin{lstlisting}[language=bash]
sudo systemctl start ssh
\end{lstlisting}

\begin{itemize}
        \item Check service status: \texttt{systemctl status ssh}.
        \item Enable service at boot: \texttt{systemctl enable ssh}.
    \end{itemize}
    \item \textbf{Shutdown Commands}
    \begin{itemize}
        \item Schedule a reboot in 10 minutes, sending a warning message:
\end{itemize}

\begin{lstlisting}[language=bash]
sudo shutdown -r +10 "System will reboot in 10 minutes."
\end{lstlisting}

\begin{itemize}
        \item Cancel a scheduled shutdown with:
\end{itemize}

\begin{lstlisting}[language=bash]
sudo shutdown -c
\end{lstlisting}

\begin{itemize}
        \item Use \textbf{systemctl} to reboot immediately: \texttt{systemctl reboot}.
    \end{itemize}
    \item \textbf{Sending Warnings}
    \begin{itemize}
        \item Open a second terminal and log in as a test user.
        \item From the admin terminal, run:
\end{itemize}

\begin{lstlisting}[language=bash]
wall "Warning! System moving to single-user mode in 1 minute."
\end{lstlisting}

\begin{itemize}
        \item Confirm the message appears in the other terminal.
    \end{itemize}
\end{enumerate}

\section*{Summary}
\begin{itemize}
    \item \textbf{SysVinit} uses numbered runlevels (0–6), configured via \textbf{/etc/inittab}, and manages services in \textbf{/etc/init.d/}.
    \item \textbf{systemd} uses \textbf{targets} and \textbf{units}, with \textbf{systemctl} providing service control and target isolation.
    \item \textbf{Upstart} (historical) uses \textbf{initctl} and scripts in \textbf{/etc/init/}.
    \item Shutting down, rebooting, or switching modes should alert current users (via \textbf{wall} or \textbf{shutdown}’s broadcast).
    \item Proper runlevel/target configuration ensures the correct set of services starts at boot, maximizing system stability and user support.
\end{itemize}

%=======================================================
% TOPIC 102: LINUX INSTALLATION AND PACKAGE MANAGEMENT
%=======================================================

\chapter{Topic 102: Linux Installation and Package Management}
\section{102.1 Design hard disk layout}
\textit{[Brief syllabus and questions to be added here]}

\section{102.2 Install a boot manager}
\textit{[Brief syllabus and questions to be added here]}

\section{102.3 Manage shared libraries}
\textit{[Brief syllabus and questions to be added here]}

\section{102.4 Use Debian package management}
\textit{[Brief syllabus and questions to be added here]}

\section{102.5 Use RPM and YUM package management}
\textit{[Brief syllabus and questions to be added here]}

\section{102.6 Linux as a virtualization guest}
\textit{[Brief syllabus and questions to be added here]}

%=======================================================
% TOPIC 103: GNU AND UNIX COMMANDS
%=======================================================
\chapter{Topic 103: GNU and UNIX Commands}
\section{103.1 Work on the command line}
\textit{[Brief syllabus and questions to be added here]}

\section{103.2 Process text streams using filters}
\textit{[Brief syllabus and questions to be added here]}

\section{103.3 Perform basic file management}
\textit{[Brief syllabus and questions to be added here]}

\section{103.4 Use streams, pipes and redirects}
\textit{[Brief syllabus and questions to be added here]}

\subsection*{103.4 Lesson 1}
\textit{[Brief syllabus and questions to be added here]}

\section{103.5 Create, monitor and kill processes}
\textit{[Brief syllabus and questions to be added here]}

\section{103.6 Modify process execution priorities}
\textit{[Brief syllabus and questions to be added here]}

\subsection*{103.6 Lesson 1}
\textit{[Brief syllabus and questions to be added here]}

\section{103.7 Search text files using regular expressions}
\textit{[Brief syllabus and questions to be added here]}

\section{103.8 Basic file editing}
\textit{[Brief syllabus and questions to be added here]}

%=======================================================
% TOPIC 104: DEVICES, LINUX FILESYSTEMS, FHS
%=======================================================
\chapter{Topic 104: Devices, Linux Filesystems, Filesystem Hierarchy Standard}

\section{104.1 Create partitions and filesystems}
\textit{[Brief syllabus and questions to be added here]}

\section{104.2 Maintain the integrity of filesystems}
\textit{[Brief syllabus and questions to be added here]}

\section{104.3 Control mounting and unmounting of filesystems}
\textit{[Brief syllabus and questions to be added here]}

\section{104.5 Manage file permissions and ownership}
\textit{[Brief syllabus and questions to be added here]}

\subsection*{104.5 Lesson 1}
\textit{[Brief syllabus and questions to be added here]}

\section{104.6 Create and change hard and symbolic links}
\textit{[Brief syllabus and questions to be added here]}

\section{104.7 Find system files and place files in the correct location}
\textit{[Brief syllabus and questions to be added here]}

%=======================================================
% TOPIC 105: SHELLS AND SHELL SCRIPTING
%=======================================================
\chapter{Topic 105: Shells and Shell Scripting}
\section{105.1 Customize and use the shell environment}
\textit{[Brief syllabus and questions to be added here]}

\section{105.2 Customize or write simple scripts}
\textit{[Brief syllabus and questions to be added here]}

%=======================================================
% TOPIC 106: USER INTERFACES AND DESKTOPS
%=======================================================
\chapter{Topic 106: User Interfaces and Desktops}
\section{106.1 Install and configure X11}
\textit{[Brief syllabus and questions to be added here]}

\section{106.2 Graphical Desktops}
\textit{[Brief syllabus and questions to be added here]}

\section{106.3 Accessibility}
\textit{[Brief syllabus and questions to be added here]}

%=======================================================
% TOPIC 107: ADMINISTRATIVE TASKS
%=======================================================
\chapter{Topic 107: Administrative Tasks}
\section{107.1 Manage user and group accounts and related system files}
\textit{[Brief syllabus and questions to be added here]}

\section{107.2 Automate system administration tasks by scheduling jobs}
\textit{[Brief syllabus and questions to be added here]}

\section{107.3 Localisation and internationalisation}
\textit{[Brief syllabus and questions to be added here]}

%=======================================================
% TOPIC 108: ESSENTIAL SYSTEM SERVICES
%=======================================================
\chapter{Topic 108: Essential System Services}
\section{108.1 Maintain system time}
\textit{[Brief syllabus and questions to be added here]}

\section{108.2 System logging}
\textit{[Brief syllabus and questions to be added here]}

\section{108.3 Mail Transfer Agent (MTA) basics}
\textit{[Brief syllabus and questions to be added here]}

\section{108.4 Manage printers and printing}
\textit{[Brief syllabus and questions to be added here]}

%=======================================================
% TOPIC 109: NETWORKING FUNDAMENTALS
%=======================================================
\chapter{Topic 109: Networking Fundamentals}
\section{109.1 Fundamentals of internet protocols}
\textit{[Brief syllabus and questions to be added here]}

\section{109.2 Persistent network configuration}
\textit{[Brief syllabus and questions to be added here]}

\section{109.3 Basic network troubleshooting}
\textit{[Brief syllabus and questions to be added here]}

\section{109.4 Configure client side DNS}
\textit{[Brief syllabus and questions to be added here]}

%=======================================================
% TOPIC 110: SECURITY
%=======================================================
\chapter{Topic 110: Security}
\section{110.1 Perform security administration tasks}
\textit{[Brief syllabus and questions to be added here]}

\section{110.2 Setup host security}
\textit{[Brief syllabus and questions to be added here]}

\section{110.3 Securing data with encryption}
\textit{[Brief syllabus and questions to be added here]}

%-------------------------------------------------------
% ANSWERS SECTION
%-------------------------------------------------------
\clearpage

\chapter*{Answers} 
\addcontentsline{toc}{chapter}{Answers} 

%=======================================================
% ANSWERS FOR TOPIC 101
%=======================================================
\section*{Topic 101: System Architecture}
\addcontentsline{toc}{section}{Topic 101: System Architecture}

\subsection*{101.1 Determine and Configure Hardware Settings}
\subsubsection*{Multiple-Choice Questions (101.1)}
\begin{enumerate}[1.]
    \item A
    \item C
    \item D
    \item B
    \item A
    \item B
    \item C
    \item A
    \item C
    \item B
    \item B
    \item C
    \item C
    \item D
    \item B
    \item C
    \item C
    \item C
    \item D
    \item C
    \end{enumerate}

\subsubsection*{Fill-in-the-Blank Questions (101.1)}
\begin{enumerate}[1.]
    \item BIOS
    \item lsmod
    \item driver
    \item udev
    \item /proc
    \item BIOS
    \item SCSI
    \item modprobe
    \item /etc/modprobe.d
    \item lsusb
    \end{enumerate}

\subsection*{101.2 Boot the System}
\subsubsection*{Multiple-Choice Questions (101.2)}

\begin{enumerate}[1.]
    \item \textbf{D}
    \item \textbf{A}
    \item \textbf{C}
    \item \textbf{B}
    \item \textbf{A}
\end{enumerate}

\subsubsection*{Fill-in-the-Blank Questions (101.2)}
\begin{enumerate}[1.]
    \item \textbf{kernel}
    \item \textbf{grub}
    \item \textbf{/boot}
    \item \textbf{initrd}
    \item \textbf{systemd}
\end{enumerate}

\end{document}
