\documentclass[a4paper]{report}
\usepackage[margin=1in]{geometry}
\usepackage[T1]{fontenc}
\usepackage[utf8]{inputenc}
\usepackage[hidelinks]{hyperref}
\usepackage{amsmath}
\usepackage{amssymb}
\usepackage{enumerate}
\usepackage{setspace}
\usepackage{listings}
\usepackage{xcolor}
\usepackage{amsthm}
\newtheorem*{noteenv}{Note}


% Configuration pour les listings de code
\definecolor{codebg}{RGB}{245,245,245}
\lstset{
    language=bash,
    backgroundcolor=\color{codebg},
    basicstyle=\ttfamily\small,
    frame=single,
    breaklines=true,
    columns=fullflexible,
    keywordstyle=\color{blue},
    commentstyle=\color{gray},
    stringstyle=\color{orange},
    showstringspaces=false
}
\setstretch{1.15}

\begin{document}

%-------------------------------------------------------
% Title Page
%-------------------------------------------------------
\begin{titlepage}
    \centering
    \vspace*{3cm}
    {\Huge \textbf{LPIC-1 Exam Workbook}}\\
    \vspace{1cm}
    {\large \textit{A Chapter-by-Chapter Syllabus with Practice Questions}}\\
    \vfill
    {\large \textbf{Version 1.0}}\\
    \vspace{2cm}
    \vfill
    \textbf{Author:} Mehdi JAFARIZADEH \\
    \textbf{Date:} January 1, 2025
    \vspace{2cm}
\end{titlepage}

\tableofcontents
\newpage

%=======================================================
% TOPIC 101: SYSTEM ARCHITECTURE
%=======================================================
\chapter{Topic 101: System Architecture}




%-------------------------------------------------------
% 101.1 Determine and Configure Hardware Settings
%-------------------------------------------------------
\section*{101.1 Determine and Configure Hardware Settings}
\addcontentsline{toc}{section}{101.1 Determine and Configure Hardware Settings}

\textbf{Reference to LPI Objectives:}  
\begin{itemize}
    \item \textbf{LPIC-1 v5, Exam 101, Objective 101.1}
    \item \textbf{Weight:} 2
\end{itemize}

\subsection*{Key Knowledge Areas}
\begin{itemize}
    \item Enabling/disabling integrated peripherals (BIOS/UEFI).
    \item Identifying different types of mass storage devices.
    \item Determining hardware resources for devices (IRQ, DMA, etc.).
    \item Using tools (\texttt{lsusb}, \texttt{lspci}, \texttt{lsmod}) for hardware inspection.
    \item Manipulating USB devices.
    \item Understanding \texttt{sysfs}, \texttt{udev}, and \texttt{dbus} concepts.
\end{itemize}

\subsection*{Important Files, Terms, and Utilities}
\begin{itemize}
    \item \textbf{/sys/}
    \item \textbf{/proc/}
    \item \textbf{/dev/}
    \item \texttt{modprobe}
    \item \texttt{lsmod}
    \item \texttt{lspci}
    \item \texttt{lsusb}
\end{itemize}

\section*{Lesson Overview}

Modern computers rely on standards for firmware and hardware interaction. On x86 platforms, the firmware could be traditional \textbf{BIOS} or newer \textbf{UEFI}. Both allow for configuring hardware resources (e.g., integrated peripherals, IRQs, DMA settings) even before the operating system loads.

Once Linux is running, device detection and configuration rely on the kernel and support from user-space utilities such as \texttt{lspci}, \texttt{lsusb}, \texttt{lsmod}, and various pseudo-filesystems in \textbf{/proc} and \textbf{/sys}.

\subsection*{1. BIOS and UEFI Configuration}
\begin{itemize}
    \item \textbf{Accessing Firmware:} Typically press \texttt{Del}, \texttt{F2}, or \texttt{F12} at startup.
    \item \textbf{Common Configurations:}
    \begin{itemize}
        \item Enable/disable integrated peripherals (USB ports, onboard audio, etc.).
        \item Set boot order and define the primary device for the bootloader.
        \item Adjust CPU features or RAM parameters if needed.
    \end{itemize}
    \item \textbf{Impact:} Misconfiguration (e.g., wrong boot device) can prevent the OS from loading.
\end{itemize}

\subsection*{2. Device Detection in Linux}
\begin{itemize}
    \item \textbf{Goal:} Match hardware parts to the correct driver (\textbf{kernel module}).
    \item \textbf{Basic Workflow:}
    \begin{enumerate}
        \item \textbf{Check if hardware is detected} (e.g., \texttt{lspci}, \texttt{lsusb}).
        \item \textbf{Verify if a driver is loaded} (e.g., \texttt{lsmod}, \texttt{lspci -k}).
        \item \textbf{Confirm functionality} via logs, testing, or additional tools.
    \end{enumerate}
\end{itemize}

\subsection*{3. Commands for Hardware Inspection}

\begin{enumerate}
    \item \textbf{\texttt{lspci}}
    \begin{itemize}
        \item Lists PCI devices (graphics cards, network interfaces, etc.).
        \item Use \texttt{-v} for more detail and \texttt{-k} to see which kernel modules are in use.
        \item Example:
        \begin{lstlisting}[language=bash]
lspci -s 04:02.0 -v
lspci -s 01:00.0 -k
        \end{lstlisting}
    \end{itemize}

    \item \textbf{\texttt{lsusb}}
    \begin{itemize}
        \item Lists USB devices (keyboards, mice, USB hubs, etc.).
        \item Use \texttt{-v} for verbose output and \texttt{-d <vendor:product>} to focus on a specific device.
        \item Example:
        \begin{lstlisting}[language=bash]
lsusb -v -d 1781:0c9f
lsusb -t  # Show devices in a tree structure
        \end{lstlisting}
    \end{itemize}

    \item \textbf{\texttt{lsmod}}
    \begin{itemize}
        \item Shows loaded kernel modules.
        \item Columns: \textbf{Module}, \textbf{Size}, \textbf{Used by} (dependency information).
        \item Example:
        \begin{lstlisting}[language=bash]
lsmod | grep snd_hda_intel
        \end{lstlisting}
    \end{itemize}

    \item \textbf{\texttt{modprobe}}
    \begin{itemize}
        \item Loads or unloads modules (with dependencies).
        \item \texttt{modprobe -r <module>} removes a module if not in use.
        \item \texttt{modinfo <module>} shows module details (author, license, parameters, etc.).
        \item Configuration files in \texttt{/etc/modprobe.d/} can blacklist or set module parameters.
    \end{itemize}
\end{enumerate}

\subsection*{4. Hardware Information Files}
\begin{itemize}
    \item \textbf{/proc} (pseudo-filesystem for processes and hardware info)
    \begin{itemize}
        \item \texttt{/proc/cpuinfo}, \texttt{/proc/interrupts}, \texttt{/proc/ioports}, \texttt{/proc/dma}
    \end{itemize}
    \item \textbf{/sys} (\texttt{sysfs} for device and kernel data)
    \item \textbf{/dev} (device files)
    \begin{itemize}
        \item Each entry represents a device (e.g., \texttt{/dev/sda1}, \texttt{/dev/fd0}).
        \item \textbf{\texttt{udev}} dynamically creates/removes these files as devices connect or disconnect.
    \end{itemize}
\end{itemize}

\subsection*{5. Storage Devices}
\begin{itemize}
    \item \textbf{Block Devices:} Accessed in fixed-size blocks (hard disks, SSDs, etc.).
    \item \textbf{Naming Conventions:}
    \begin{itemize}
        \item Newer kernels use \texttt{sd} prefix for most disks; partitions are numbered (\texttt{/dev/sda1}).
        \item \textbf{IDE} devices also appear as \texttt{sd} on modern kernels
        \item \textbf{NVMe} devices get names like \texttt{/dev/nvme0n1p1}.
        \item \textbf{SD Cards} often appear as \texttt{/dev/mmcblk0p1}.
    \end{itemize}

    \item \textbf{Hotplug and Coldplug:}
    \begin{itemize}
        \item \textbf{Hotplug:} device recognized after boot (e.g., USB).
        \item \textbf{Coldplug:} device recognized during boot (built-in or already connected).
    \end{itemize}
\end{itemize}

\section*{Workbook Exercises}

\begin{enumerate}
    \item \textbf{Accessing BIOS/UEFI}
    \begin{itemize}
        \item Reboot a test machine and enter BIOS/UEFI.
        \item Locate the sections that let you enable/disable integrated peripherals.
        \item Identify the menu where boot order is set.
    \end{itemize}

    \item \textbf{Listing Hardware}
    \begin{itemize}
        \item On a Linux system, run \texttt{lspci -k}.
        \begin{itemize}
            \item Identify which driver is used by the video card.
        \end{itemize}
        \item Run \texttt{lsusb -t}.
        \begin{itemize}
            \item Check which USB driver modules are in use (e.g., \texttt{btusb, usbhid}).
        \end{itemize}
    \end{itemize}

    \item \textbf{Exploring /proc and /sys}
    \begin{itemize}
        \item View CPU details with \texttt{cat /proc/cpuinfo}.
        \item Inspect interrupts with \texttt{cat /proc/interrupts}.
        \item Explore \texttt{/sys/class} and \texttt{/sys/block} to see how devices are represented.
    \end{itemize}

    \item \textbf{Managing Kernel Modules}
    \begin{itemize}
        \item Use  \texttt{lsmod} to list all loaded modules.
        \item Pick a module (e.g., a sound driver) and unload it with \texttt{sudo modprobe -r <module>}.
        \begin{itemize}
            \item Check if removal is allowed (the module should not be in use).
        \end{itemize}
        \item Use \texttt{modinfo -p <module>} to see possible parameters, and note how you might apply them in \texttt{/etc/modprobe.d/}.
        
    \end{itemize}

    \item \textbf{Blacklisting a Module}
    \begin{itemize}
        \item Create a test file in \texttt{/etc/modprobe.d/} to blacklist an unwanted module (e.g., \texttt{nouveau}).
        \item Reboot and confirm it is not loaded by checking \texttt{lsmod}.
    \end{itemize}
\end{enumerate}

\section*{Summary}
\begin{itemize}
    \item Modern systems rely on firmware (BIOS/UEFI) for initial hardware configuration.
    \item Linux identifies devices via kernel modules; tools like \texttt{lspci}, \texttt{lsusb}, \texttt{lsmod}, and \texttt{modprobe} allow you to inspect and manage hardware.
    \item \texttt{/proc} and \texttt{/sys} provide detailed, real-time system information, while \texttt{udev} dynamically manages device nodes in \texttt{/dev}.
    \item Storage device naming conventions follow standard patterns such as \texttt{sd}, \texttt{nvme}, \texttt{mmcblk}, and partition numbers like \texttt{/dev/sda1}.
    \item Understanding how to enable/disable devices, load/unload modules, and explore hardware information files is crucial for effective system administration and LPIC-1 success.
\end{itemize}




%-------------------------------------------------------
% Multiple-Choice Questions (101.1)
%-------------------------------------------------------
\newpage
\section*{Multiple-Choice Questions for 101.1}

\begin{enumerate}[1.]
\item When trying to enable or disable motherboard-integrated peripherals, which component of the system is typically used?
  \begin{enumerate}[A)]
    \item The BIOS or UEFI configuration utility
    \item The Linux kernel’s initrd
    \item The \texttt{/boot} partition
    \item The \texttt{lsusb} command

  \end{enumerate}

\item Which command lists devices currently connected to the PCI bus?
  \begin{enumerate}[A)]
    \item \texttt{modprobe}
    \item \texttt{lsmod}
    \item \texttt{lspci}
    \item \texttt{lshw}
  \end{enumerate}

\item Which of the following commands helps you list USB devices in a tree-like hierarchy?
  \begin{enumerate}[A)]
    \item lsusb -a
    \item lsusb -s
    \item lsusb -f
    \item lsusb -t
  \end{enumerate}

\item To remove a kernel module (along with its dependencies) while the system is running, which command should be used?
  \begin{enumerate}[A)]
    \item modinfo -r
    \item modprobe -r
    \item rmmod --all
    \item lsmod -r
  \end{enumerate}

\item On modern Linux systems, SATA disks are generally identified as which kind of device name?
  \begin{enumerate}[A)]
    \item /dev/sdX
    \item /dev/hdX
    \item /dev/nvmeXnY
    \item /dev/fdX
  \end{enumerate}

\item Which file below would you edit to permanently blacklist a problematic kernel module such that it doesn’t load automatically?
  \begin{enumerate}[A)]
    \item /etc/rc.local
    \item /etc/modprobe.d/blacklist.conf
    \item /boot/grub/grub.cfg
    \item /proc/blacklist/modules
  \end{enumerate}

\item Which pseudo-filesystem is most specifically devoted to storing device and kernel data related to hardware?
  \begin{enumerate}[A)]
    \item /dev
    \item /proc
    \item /sys
    \item /home
  \end{enumerate}

\item Which command line will show a specific USB device’s verbose information using its vendor:product ID (e.g., 1781:0c9f)?
  \begin{enumerate}[A)]
    \item lsusb -d 1781:0c9f -v
    \item lsusb -p 1781:0c9f -v
    \item lsusb -i 1781:0c9f
    \item lsusb -v -s 01:02
  \end{enumerate}

\item In the output of lsmod, the “Used by” column indicates:
  \begin{enumerate}[A)]
    \item the file size of the module on disk
    \item the user-level applications that installed the module
    \item the modules or processes depending on that module
    \item kernel version compatibility for that module
  \end{enumerate}

\item If you need to confirm which kernel driver is in use by a particular PCI device, which \texttt{lspci} option combination is most helpful on recent distributions?
  \begin{enumerate}[A)]
    \item lspci -m
    \item lspci -k
    \item lspci -D
    \item lspci --driver
  \end{enumerate}

\item What does the output of \texttt{lsusb -t} specifically highlight that differs from plain \texttt{lsusb}?
  \begin{enumerate}[A)]
    \item The exact partition layout of attached USB drives
    \item A hierarchical (tree-like) representation of USB devices and drivers
    \item The SCSI ID mappings of USB-attached devices
    \item A summary of device’s kernel modules only
  \end{enumerate}

\item Which best describes the function of the \texttt{modinfo} command?
  \begin{enumerate}[A)]
    \item It removes the specified module from the kernel
    \item It displays all processes currently using a kernel module
    \item It lists detailed information about a specified module, including parameters
    \item It inserts the specified module and resolves dependencies
  \end{enumerate}

\item What is the role of \texttt{udev} on a modern Linux system?
  \begin{enumerate}[A)]
    \item It is a pseudo-filesystem used to track hardware devices in \texttt{/sys}
    \item It permanently stores device drivers in \texttt{/boot}
    \item It manages device nodes in \texttt{/dev}, handling hotplug/coldplug events
    \item It only configures CPU frequency scaling
  \end{enumerate}

\item Which file inside \texttt{/proc} would you inspect to see how many interrupts have occurred for each device?
  \begin{enumerate}[A)]
    \item \texttt{/proc/ioports}
    \item \texttt{/proc/dma}
    \item \texttt{/proc/cpuinfo}
    \item \texttt{/proc/interrupts}
  \end{enumerate}

\item If a device is recognized by the kernel but not functioning correctly, which of the following is the most likely underlying cause?
  \begin{enumerate}[A)]
    \item The BIOS is not set to read the device’s firmware
    \item The associated kernel module (driver) is not loaded or is misconfigured
    \item The CPU lacks the required SSE instruction set
    \item The device was not assigned a correct IRQ in the \texttt{/etc/fstab}
  \end{enumerate}

\item Which file is typically used to pass persistent module load options like \texttt{options nouveau modeset=0}?
  \begin{enumerate}[A)]
    \item \texttt{/etc/udev/rules.d/99-custom.rules}
    \item \texttt{/proc/meminfo}
    \item \texttt{/etc/modprobe.d/<module>.conf}
    \item \texttt{/etc/modules-load.d/module.options}
  \end{enumerate}

\item What is the main purpose of SysFS (\texttt{/sys}) in a Linux system?
  \begin{enumerate}[A)]
    \item Stores process information like CPU usage
    \item Holds user configuration data for \texttt{/home}
    \item Exports device and driver information from the kernel to user space
    \item Contains scripts to mount all system filesystems
  \end{enumerate}

\item Which command is most appropriate for listing all currently loaded kernel modules?
  \begin{enumerate}[A)]
    \item \texttt{ls -la /lib/modules/\$(uname -r)}
    \item \texttt{depmod -a}
    \item \texttt{lsmod}
    \item \texttt{insmod}
  \end{enumerate}

\item To selectively unload the \texttt{snd-hda-intel} module along with related dependent modules, which command would you use?
  \begin{enumerate}[A)]
    \item \texttt{modinfo snd-hda-intel --remove}
    \item \texttt{lsmod --unload snd-hda-intel}
    \item \texttt{depmod -r snd-hda-intel}
    \item \texttt{modprobe -r snd-hda-intel}
  \end{enumerate}

\item If you see a disk labeled as \texttt{/dev/mmcblk0p1}, which type of physical device is this likely referring to?
  \begin{enumerate}[A)]
    \item A SATA SSD
    \item An older IDE HDD
    \item An SD card or MMC device
    \item A USB DVD drive
  \end{enumerate}
\end{enumerate}

%-------------------------------------------------------
% Fill-in-the-Blank Questions (101.1)
%-------------------------------------------------------
\newpage
\section*{Fill-in-the-Blank Questions for 101.1}

\begin{enumerate}[1.]

\item The older firmware commonly used before the UEFI standard is called \underline{\hspace{2cm}}.

\item The \underline{\hspace{2cm}} command lists all kernel modules currently loaded into the system.

\item A kernel module responsible for controlling hardware in Linux is often referred to as a \underline{\hspace{2cm}}.

\item The Linux subsystem that manages device node creation in \texttt{/dev} and handles hotplug/coldplug events is called \underline{\hspace{2cm}}.

\item The special, memory-based filesystem used for storing process and hardware information is the \underline{\hspace{2cm}} directory.

\item To configure boot device priority and enable or disable onboard peripherals, a user must typically access the \underline{\hspace{2cm}} or UEFI setup utility.

\item In Linux, disks commonly appear under \texttt{/dev} as \underline{\hspace{2cm}} devices (e.g., \texttt{/dev/sda}, \texttt{/dev/sdb}) on modern systems.

\item The \underline{\hspace{2cm}} command is used to insert or remove kernel modules and their dependencies.

\item When blacklisting a kernel module to prevent it from loading automatically, the configuration file is often placed in \underline{\hspace{2cm}}.

\item To see a hierarchical (tree-like) view of USB devices and the drivers handling them, you can run \underline{\hspace{2cm}} with the \texttt{-t} option.
 
\end{enumerate}

%-------------------------------------------------------
% 101.2 Boot the System
%-------------------------------------------------------
\newpage

\section*{101.2 Boot the System}
\addcontentsline{toc}{section}{101.2 Boot the System}

\textbf{Reference to LPI Objectives:}  
\begin{itemize}
    \item \textbf{LPIC-1 v5, Exam 101, Objective 101.2}
    \item \textbf{Weight:} 3
\end{itemize}

\subsection*{Key Knowledge Areas}
\begin{itemize}
    \item Providing common bootloader commands and kernel options at boot.
    \item Understanding the boot sequence (BIOS/UEFI through OS startup).
    \item Familiarity with SysVinit, systemd, and Upstart.
    \item Checking boot events and logs (\texttt{dmesg}, \texttt{journalctl}).
\end{itemize}

\subsection*{Important Files, Terms, and Utilities}
\begin{itemize}
    \item \textbf{dmesg}
    \item \textbf{journalctl}
    \item \textbf{BIOS} / \textbf{UEFI}
    \item \textbf{bootloader} (GRUB)
    \item \textbf{kernel}
    \item \textbf{initramfs}
    \item \textbf{init} (SysVinit, systemd, Upstart)
    \item \textbf{/proc/cmdline}
    \item \textbf{/var/log/}
\end{itemize}

\section*{Lesson Overview}

Booting a Linux system involves multiple stages:
\begin{enumerate}
    \item \textbf{Firmware Load:} BIOS or UEFI initializes basic hardware.
    \item \textbf{Bootloader:} Typically \textbf{GRUB}, which locates and loads the kernel.
    \item \textbf{Kernel \& initramfs:} Kernel initializes hardware and reads modules from the initramfs.
    \item \textbf{System Initialization:} \textbf{init} (SysVinit, systemd, Upstart) starts services and completes the boot process.
\end{enumerate}

\subsection*{1. BIOS vs. UEFI}
\begin{itemize}
\item\textbf{BIOS}  
\begin{itemize}
    \item Uses MBR (first 512 bytes) to load boot code (GRUB stage 1).
    \item Relies on a DOS partition scheme and the Master Boot Record.
    \item Boots the second stage of the bootloader, which in turn loads the kernel.
\end{itemize}

\item\textbf{UEFI}
\begin{itemize}
    \item Looks at entries in \textbf{NVRAM} to find an \textbf{EFI application} (usually GRUB).
    \item Loads the EFI application from a dedicated \textbf{EFI System Partition (ESP)}.
    \item Supports \textbf{Secure Boot} to allow only signed EFI applications.
\end{itemize}
\end{itemize}

\subsection*{2. Bootloader (GRUB)}
\begin{itemize}
    \item Presents a menu of installed kernels or operating systems.
    \item Enables passing \textbf{kernel parameters} (e.g., \texttt{quiet}, \texttt{acpi=off}, \texttt{root=/dev/sdaX}, etc.).
    \item Kernel parameters can be made persistent in \texttt{/etc/default/grub} and then updated with:
\end{itemize}

\begin{lstlisting}[language=bash]
grub-mkconfig -o /boot/grub/grub.cfg
\end{lstlisting}

\begin{itemize}
    \item Current kernel parameters are visible in \texttt{/proc/cmdline}.
\end{itemize}

\subsection*{3. System Initialization}

\begin{enumerate}
    \item \textbf{initramfs}
    \begin{itemize}
        \item Temporary root filesystem with essential drivers/modules.
        \item Lets the kernel mount the actual root filesystem.
    \end{itemize}
    \item \textbf{init}
    \begin{itemize}
        \item The “first process” in user space.
        \item \textbf{SysVinit:} uses \textbf{runlevels} (0–6).
        \item \textbf{systemd:} uses \textbf{targets}, concurrency, D-Bus, cgroups. Most common in modern distros.
        \item \textbf{Upstart:} parallel boot focusing on faster startup. Largely replaced by systemd.
    \end{itemize}
\end{enumerate}

\subsection*{4. Boot Logging and Inspection}
\begin{itemize}
    \item \textbf{dmesg}
    \begin{itemize}
        \item Displays the \textbf{kernel ring buffer} (including boot messages).
        \item Clears with \texttt{dmesg --clear}.
    \end{itemize}
    \item \textbf{journalctl}
    \begin{itemize}
        \item Systemd-based logging tool.
        \item \texttt{journalctl -b} shows current boot messages.
        \item \texttt{journalctl --list-boots} lists previous boots.
    \end{itemize}
    \item Traditional log files also found in \texttt{/var/log/}, e.g., \texttt{/var/log/messages} or \texttt{/var/log/syslog}.
\end{itemize}

\section*{Workbook Exercises}

\begin{enumerate}
    \item \textbf{Firmware Awareness}
    \begin{itemize}
        \item Reboot a test machine.
        \item Determine whether it uses \textbf{BIOS} or \textbf{UEFI}.
        \item In BIOS: Find where the boot order is set.
        \item In UEFI: Locate the ESP partition and explore contents if possible.
    \end{itemize}
    \item \textbf{GRUB Menu and Kernel Parameters}
    \begin{itemize}
        \item Boot into the GRUB menu by pressing \textbf{Shift} (BIOS) or \textbf{Esc} (UEFI).
        \item Edit a menu entry to add or change a kernel parameter (e.g., \texttt{init=/bin/bash}, \texttt{acpi=off}).
        \item After boot, check \texttt{/proc/cmdline} to confirm your changes.
    \end{itemize}
    \item \textbf{System Initialization Tools}
    \begin{itemize}
        \item Identify which init system your distribution uses (\texttt{ps -p 1 -o comm=}).
        \item If it’s systemd, compare output of these commands:
\end{itemize}

\begin{lstlisting}[language=bash]
systemctl list-units --type=service
journalctl -b
\end{lstlisting}

\begin{itemize}
        \item If SysVinit is present, inspect runlevel scripts in \texttt{/etc/rc.d/} or \texttt{/etc/init.d/}.
    \end{itemize}
    \item \textbf{Inspecting Boot Logs}
    \begin{itemize}
        \item Run \texttt{dmesg | less} to page through the kernel ring buffer.
        \item If using systemd, run \texttt{journalctl --list-boots} to see previous boots.
        \item View the logs for the current boot with \texttt{journalctl -b 0}.
    \end{itemize}
    \item \textbf{initramfs Exploration}
    \begin{itemize}
        \item Locate your initramfs file (commonly in \texttt{/boot}, e.g., \texttt{initramfs-<version>.img}).
        \item List contents using \texttt{lsinitrd} or \texttt{unmkinitramfs} (may require additional packages).
        \item Identify which modules are included for the root filesystem.
    \end{itemize}
\end{enumerate}

\section*{Summary}
\begin{itemize}
    \item The boot process starts with \textbf{BIOS/UEFI} firmware, which calls \textbf{GRUB} to load the \textbf{kernel}.
    \item The \textbf{initramfs} contains essential modules and mounts the real root filesystem.
    \item An \textbf{init} system (SysVinit, systemd, Upstart) then starts daemons and services.
    \item \textbf{dmesg} and \textbf{journalctl} provide essential logs for troubleshooting.
    \item Understanding these steps ensures you can troubleshoot common startup issues and manage kernel parameters effectively.
\end{itemize}


%-------------------------------------------------------
% Multiple-Choice Questions for (101.2)
%-------------------------------------------------------
\newpage
\section*{Multiple-Choice Questions for 101.2}

\begin{enumerate}[1.]
    \item Which of the following best describes the role of the \textbf{kernel ring buffer} during the boot process?
      \begin{enumerate}[A)]
        \item It stores a copy of the MBR after BIOS initialization.
        \item It holds user processes’ initialization scripts during startup.
        \item It temporarily stores kernel messages, including boot messages.
        \item It provides secure boot verification for the EFI System Partition.
      \end{enumerate}
    
    \item On a typical Linux system with GRUB, which file should be edited to \textbf{persistently} add kernel boot parameters?
      \begin{enumerate}[A)]
        \item \texttt{/etc/default/grub}
        \item \texttt{/etc/systemd/system.conf}
        \item \texttt{/boot/vmlinuz}
        \item \texttt{/proc/cmdline}
      \end{enumerate}
    
    \item Which bootloader is most commonly associated with modern x86-based Linux systems?
      \begin{enumerate}[A)]
        \item LILO
        \item SYSLINUX
        \item BURG
        \item GRUB
      \end{enumerate}
    
    \item Which of the following statements about \textbf{Secure Boot} is \textbf{true}?
      \begin{enumerate}[A)]
        \item It forces the user to boot only from a local disk rather than USB devices.
        \item It requires EFI applications to be signed/authorized by the hardware vendor or a trusted party.
        \item It loads the SysVinit scripts in parallel to reduce the boot time of the OS.
        \item It uses MBR partition tables exclusively and disables GPT.
      \end{enumerate}
    
    \item The BIOS in a legacy (non-UEFI) x86 system typically reads and executes boot code from what specific location?
      \begin{enumerate}[A)]
        \item The first 440 bytes of the MBR on the primary boot device
        \item The second stage of GRUB in \texttt{/boot/grub}
        \item The NVRAM partition labeled \texttt{/efi/boot}
        \item \texttt{/boot} partition
      \end{enumerate}
    
    \item What is the \textbf{primary purpose} of \texttt{initramfs} during the boot process?
      \begin{enumerate}[A)]
        \item To store the kernel ring buffer.
        \item To provide early user accounts for system security.
        \item To load required kernel modules so the real root filesystem can be mounted.
        \item To replace the BIOS firmware in older systems.
      \end{enumerate}
    
    \item You want to limit a Linux guest system to a maximum of 1 GB of RAM at boot time. Which kernel parameter should be used?
      \begin{enumerate}[A)]
        \item \texttt{nosmp=1G}
        \item \texttt{mem=1G}
        \item \texttt{ram=1G}
        \item \texttt{maxcpus=1G}
      \end{enumerate}
    
    \item Which of the following is a feature of \textbf{systemd}?
      \begin{enumerate}[A)]
        \item Entirely depends on runlevels 0–6 and SysV scripts.
        \item Uses sockets and D-Bus for on-demand service activation.
        \item Must be installed as a kernel module.
        \item It can only run one service at a time to avoid concurrency issues.
      \end{enumerate}
    
    \item While troubleshooting a boot issue, you want to see \textbf{previous} system boots’ log messages. Which systemd-related command enables you to do this?
      \begin{enumerate}[A)]
        \item \texttt{dmesg --previous}
        \item \texttt{journalctl --list-boots}
        \item \texttt{systemctl --history}
        \item \texttt{logrotate -b}
      \end{enumerate}
    
    \item After you edit \texttt{/etc/default/grub} to add a new kernel parameter, which command is typically used to \textbf{update} the GRUB configuration on many distributions?
      \begin{enumerate}[A)]
        \item \texttt{cp /etc/default/grub /boot/grub/grub.conf}
        \item \texttt{touch /boot/grub/grub.cfg}
        \item \texttt{grub-install /boot}
        \item \texttt{grub-mkconfig -o /boot/grub/grub.cfg}
      \end{enumerate}
    
    \item What does the kernel parameter \texttt{acpi=off} do?
      \begin{enumerate}[A)]
        \item Disables multi-processor support, similar to \texttt{nosmp}.
        \item Disables BIOS POST checks and loads the kernel directly.
        \item Disables ACPI functions to troubleshoot power management or ACPI-related issues.
        \item Forces the root filesystem to be mounted as read-only.
      \end{enumerate}
    
    \item In a SysVinit-based system, which file primarily determines which \textbf{runlevel} the system will go to when it finishes booting?
      \begin{enumerate}[A)]
        \item \texttt{/etc/fstab}
        \item \texttt{/boot/initramfs-<version>.img}
        \item \texttt{/etc/inittab}
        \item \texttt{/var/log/boot.log}
      \end{enumerate}
    
    \item When using UEFI, which partition \textbf{must} contain the bootloader or EFI applications?
      \begin{enumerate}[A)]
        \item The root (\texttt{/}) filesystem partition
        \item A dedicated GPT partition labeled \textquotedbl{}MBR\textquotedbl{}
        \item An NVRAM-based partition called \texttt{/var/lib/EFI}
        \item The EFI System Partition (ESP)
      \end{enumerate}
    
    \item Which kernel parameter instructs the system to \textbf{start} a different \textbf{initial process} instead of the default \texttt{/sbin/init} or systemd?
      \begin{enumerate}[A)]
        \item \texttt{init=/bin/bash}
        \item \texttt{systemd.unit=multi-user.target}
        \item \texttt{noapic}
        \item \texttt{ro}
      \end{enumerate}
    
    \item The term \textbf{daemon} is typically used to describe which kind of program in a Linux system?
      \begin{enumerate}[A)]
        \item A program that only runs once at boot and then terminates.
        \item A service that remains \textbf{running} in the background.
        \item Any script that an administrator invokes manually from the command line.
        \item A background service process (e.g. system or network) that runs indefinitely.
      \end{enumerate}
    
    \item Which of the following is \textbf{not} a valid kernel parameter for controlling the amount of displayed boot information?
      \begin{enumerate}[A)]
        \item \texttt{verbose=0}
        \item \texttt{quiet}
        \item \texttt{vga=ask}
        \item \texttt{maxcpus=1}
      \end{enumerate}
    
    \item If a critical system service fails to start during boot and the system uses \textbf{systemd}, where would you most likely check \textbf{first} for the relevant error messages?
      \begin{enumerate}[A)]
        \item \texttt{/proc/cmdline}
        \item \texttt{/etc/default/grub}
        \item \texttt{systemctl list-jobs}
        \item \texttt{journalctl -b} or \texttt{journalctl --boot}
      \end{enumerate}
    
    \item In a system that uses SysVinit, which runlevel is \textbf{commonly} used for \textbf{single-user mode} (maintenance mode)?
      \begin{enumerate}[A)]
        \item 2
        \item 5
        \item 1
        \item 3
      \end{enumerate}
    
    \item Which of the following statements about \textbf{Upstart} is correct?
      \begin{enumerate}[A)]
        \item It can parallelize the initialization of services but has largely been replaced by systemd.
        \item It replaces the BIOS in older systems.
        \item It is strictly a tool for reading the kernel ring buffer.
        \item It is used to sign EFI applications for Secure Boot.
      \end{enumerate}
    
    \item The BIOS POST (Power-On Self-Test) primarily checks for:
      \begin{enumerate}[A)]
        \item Valid ext4 partitions on the system’s boot drive.
        \item Basic hardware components and any major hardware failures.
        \item Corrupted kernel parameters in \texttt{/proc/cmdline}.
        \item Upstart jobs that should be started first.
      \end{enumerate}
    
    \end{enumerate}

  
  %-------------------------------------------------------
  % Fill-in-the-Blank Questions (101.2)
  %-------------------------------------------------------
  \newpage
  \section*{Fill-in-the-Blank Questions for 101.2}
  
  \begin{enumerate}[1.]
    \item The firmware on modern x86 systems can be either traditional \underline{\hspace{2cm}} or the more advanced \underline{\hspace{2cm}}.
    \item On legacy BIOS-based systems, the first stage of the bootloader is typically located in the first \underline{\hspace{2cm}} bytes of the \underline{\hspace{2cm}}.
    \item When using UEFI, the bootloader or EFI applications are stored in a dedicated partition called the \underline{\hspace{2cm}}, often formatted with a FAT filesystem.
    \item The kernel parameter \texttt{\underline{\hspace{2cm}}=/bin/bash} causes the system to start a Bash shell as the first user-space process instead of the standard init system.
    \item The file \texttt{/etc/default/grub} contains the directive \texttt{GRUB\_CMDLINE\_LINUX}, which is used to specify \underline{\hspace{2cm}} passed to the kernel at boot time.
    \item The command \texttt{grub-mkconfig -o /boot/grub/grub.cfg} is needed after modifying \texttt{/etc/default/grub} to \underline{\hspace{2cm}} the bootloader configuration.
    \item The memory area that stores kernel messages, including boot information, is called the \underline{\hspace{2cm}}, which can be viewed with the \texttt{dmesg} command.
    \item The \underline{\hspace{2cm}} process runs basic hardware checks (like checking memory) as soon as the machine is powered on, before loading the bootloader.
    \item In a SysVinit-based system, the file \texttt{/etc/\underline{\hspace{2cm}}} typically defines which runlevel the system will enter when it finishes booting.
    \item A(n) \underline{\hspace{2cm}} is a background service or process that remains running to provide system or network functionality.
\end{enumerate}
\newpage


%-------------------------------------------------------
% 101.3 Change Runlevels / Boot Targets and Shutdown or Reboot System
%-------------------------------------------------------

\section*{101.3 Change Runlevels / Boot Targets and Shutdown or Reboot System}
\addcontentsline{toc}{section}{101.3 Change Runlevels / Boot Targets and Shutdown or Reboot System}

\textbf{Reference to LPI Objectives:}  
\begin{itemize}
    \item \textbf{LPIC-1 v5, Exam 101, Objective 101.3}
    \item \textbf{Weight:} 3
\end{itemize}

\subsection*{Key Knowledge Areas}
\begin{itemize}
    \item Setting the default runlevel/boot target.
    \item Changing between runlevels/targets, including single-user mode.
    \item Shutting down and rebooting from the command line.
    \item Alerting users before switching runlevels/boot targets or major system events.
    \item Properly terminating processes.
    \item Awareness of \textbf{acpid} (power management).
\end{itemize}

\subsection*{Important Files, Terms, and Utilities}
\begin{itemize}
    \item \textbf{/etc/inittab} (SysVinit)
    \item \textbf{shutdown}
    \item \textbf{init}, \textbf{telinit} (SysVinit)
    \item \textbf{/etc/init.d/} (SysVinit scripts)
    \item \textbf{systemd}, \textbf{systemctl}
    \item \textbf{/etc/systemd/}, \textbf{/usr/lib/systemd/}
    \item \textbf{wall} (send messages to all logged-in users)
\end{itemize}

\section*{Lesson Overview}

Linux can operate in different “states” or “modes” called \textbf{runlevels} in SysVinit or \textbf{targets} in systemd. Being able to switch between them and perform system shutdowns or reboots is essential for system administration.

\section*{1. SysVinit Runlevels}

\subsection*{1. Runlevels}
\begin{itemize}
    \item \textbf{0} – Shutdown
    \item \textbf{1 (single), s} – Single-user (maintenance) mode
    \item \textbf{2, 3, 4} – Multi-user modes (3 is typical, 2/4 vary by distro)
    \item \textbf{5} – Multi-user plus graphical mode
    \item \textbf{6} – Reboot
\end{itemize}

\subsection*{2. Configuration}
\begin{itemize}
    \item \textbf{/etc/inittab} defines default runlevel (\texttt{id:x:initdefault:})
    \item Each runlevel has a dedicated directory: \textbf{/etc/rc0.d/}, \textbf{/etc/rc1.d/}, etc.
    \item Scripts in \textbf{/etc/init.d/} are symlinked to these runlevel directories.
    \begin{itemize}
        \item Names starting with \textbf{S} start services.
        \item Names starting with \textbf{K} kill services.
    \end{itemize}
\end{itemize}

\subsection*{3. Switching Runlevels}
\begin{itemize}
    \item \textbf{init} or \textbf{telinit} commands set the current runlevel.
    \item \texttt{telinit 1}: move to runlevel 1 (maintenance mode).
    \item \texttt{runlevel}: shows current and previous runlevel (e.g., \texttt{N 3} means currently 3 and no prior change).
\end{itemize}

\subsection*{4. Reloading \textbf{/etc/inittab}}
\begin{itemize}
    \item After editing \textbf{/etc/inittab}, run \texttt{telinit q} to re-read the config.
\end{itemize}

\section*{2. systemd Targets}

\subsection*{1. systemd Concepts}
\begin{itemize}
    \item \textbf{Units} represent services, sockets, devices, mounts, automounts, targets, and snapshots.
    \item \textbf{systemctl} is the primary command to manage these units (start, stop, enable, etc.).
\end{itemize}

\subsection*{2. Targets}
\begin{itemize}
    \item systemd uses \textbf{targets} to group units. Examples:
    \begin{itemize}
        \item \textbf{multi-user.target} – analogous to runlevel 3 (no GUI).
        \item \textbf{graphical.target} – analogous to runlevel 5 (GUI mode).
    \end{itemize}
    \item You can isolate a target:
\end{itemize}

\begin{lstlisting}[language=bash]
systemctl isolate multi-user.target
\end{lstlisting}

\subsection*{3. Default Target}
\begin{itemize}
    \item Change default target:
\end{itemize}

\begin{lstlisting}[language=bash]
systemctl set-default multi-user.target
\end{lstlisting}

\begin{itemize}
    \item View current default:
\end{itemize}

\begin{lstlisting}[language=bash]
systemctl get-default
\end{lstlisting}

\begin{itemize}
    \item Avoid pointing to \textbf{shutdown.target} or \textbf{reboot.target}.
\end{itemize}

\subsection*{4. Service Management}
\begin{itemize}
    \item \texttt{systemctl start/stop/restart} \texttt{<service>.service}
    \item \texttt{systemctl enable/disable} \texttt{<service>.service} (at boot)
    \item \texttt{systemctl status} \texttt{<service>.service}
    \item \texttt{systemctl list-unit-files --type=service} – list available services
    \item \texttt{systemctl list-units --type=service} – list loaded/running services
\end{itemize}

\subsection*{5. Power Management}
\begin{itemize}
    \item \texttt{systemctl suspend}, \texttt{systemctl hibernate}
    \item For finer power-event control (e.g., lid close), \textbf{acpid} can be used instead of systemd’s built-in power management.
\end{itemize}

\section*{3. Upstart (Historical)}

\begin{enumerate}
    \item \textbf{Upstart} was used in older Ubuntu-based systems before switching to systemd.
    \item \textbf{Commands}:
    \begin{itemize}
        \item \texttt{initctl list} – list services and states
        \item \texttt{start / stop / status <service>} – control services
        \item Initialization scripts: \textbf{/etc/init/}
    \end{itemize}
    \item \texttt{runlevel} and \texttt{telinit} still work for basic runlevel tasks.
\end{enumerate}

\section*{4. Shutting Down and Rebooting}

\subsection*{1. shutdown}
\begin{itemize}
    \item Syntax:
\end{itemize}

\begin{lstlisting}[language=bash]
shutdown [option] time [message]
\end{lstlisting}

\begin{itemize}
    \item \textbf{time} can be \texttt{now}, \texttt{+m} (minutes from now), or \texttt{hh:mm} (absolute time).
    \item Common options:
    \begin{itemize}
        \item \textbf{-h} – halt/power off
        \item \textbf{-r} – reboot
    \end{itemize}
    \item Notifies logged-in users and prevents new logins (unless overridden).
\end{itemize}

\subsection*{2. systemctl (systemd)}
\begin{itemize}
    \item \texttt{systemctl reboot} – reboot system
    \item \texttt{systemctl poweroff} – power off system
    \item Sometimes distros alias \texttt{poweroff} and \texttt{reboot} to systemd commands.
\end{itemize}

\subsection*{3. wall}
\begin{itemize}
    \item Sends a message to all logged-in users’ terminals (similar to \texttt{shutdown}’s broadcast).
    \item Useful for manual warnings before switching to single-user mode or shutting down.
\end{itemize}

\section*{Workbook Exercises}

\begin{enumerate}
    \item \textbf{Identify Your Init System}
    \begin{itemize}
        \item Run \texttt{ps -p 1 -o comm=} to see if your system uses \textbf{systemd}, \textbf{init}, or \textbf{Upstart}.
    \end{itemize}
    \item \textbf{Practice Switching Runlevels (SysV)}
    \begin{itemize}
        \item On a SysVinit system, edit \textbf{/etc/inittab} to set default runlevel to \textbf{3}.
        \item Run \texttt{telinit q} and verify with \texttt{runlevel}.
        \item Switch to single-user mode: \texttt{telinit 1}.
    \end{itemize}
    \item \textbf{Practice Managing systemd Targets}
    \begin{itemize}
        \item Show the current default target: \texttt{systemctl get-default}.
        \item Switch from \textbf{graphical.target} to \textbf{multi-user.target} using:
\end{itemize}

\begin{lstlisting}[language=bash]
systemctl isolate multi-user.target
\end{lstlisting}

\begin{itemize}
        \item Confirm the change: \texttt{systemctl status multi-user.target}.
    \end{itemize}
    \item \textbf{Service Control with systemd}
    \begin{itemize}
        \item Start a service (e.g., \texttt{ssh.service}):
\end{itemize}

\begin{lstlisting}[language=bash]
sudo systemctl start ssh
\end{lstlisting}

\begin{itemize}
        \item Check service status: \texttt{systemctl status ssh}.
        \item Enable service at boot: \texttt{systemctl enable ssh}.
    \end{itemize}
    \item \textbf{Shutdown Commands}
    \begin{itemize}
        \item Schedule a reboot in 10 minutes, sending a warning message:
\end{itemize}

\begin{lstlisting}[language=bash]
sudo shutdown -r +10 "System will reboot in 10 minutes."
\end{lstlisting}

\begin{itemize}
        \item Cancel a scheduled shutdown with:
\end{itemize}

\begin{lstlisting}[language=bash]
sudo shutdown -c
\end{lstlisting}

\begin{itemize}
        \item Use \textbf{systemctl} to reboot immediately: \texttt{systemctl reboot}.
    \end{itemize}
    \item \textbf{Sending Warnings}
    \begin{itemize}
        \item Open a second terminal and log in as a test user.
        \item From the admin terminal, run:
\end{itemize}

\begin{lstlisting}[language=bash]
wall "Warning! System moving to single-user mode in 1 minute."
\end{lstlisting}

\begin{itemize}
        \item Confirm the message appears in the other terminal.
    \end{itemize}
\end{enumerate}

\section*{Summary}
\begin{itemize}
    \item \textbf{SysVinit} uses numbered runlevels (0–6), configured via \textbf{/etc/inittab}, and manages services in \textbf{/etc/init.d/}.
    \item \textbf{systemd} uses \textbf{targets} and \textbf{units}, with \textbf{systemctl} providing service control and target isolation.
    \item \textbf{Upstart} (historical) uses \textbf{initctl} and scripts in \textbf{/etc/init/}.
    \item Shutting down, rebooting, or switching modes should alert current users (via \textbf{wall} or \textbf{shutdown}’s broadcast).
    \item Proper runlevel/target configuration ensures the correct set of services starts at boot, maximizing system stability and user support.
\end{itemize}


%-------------------------------------------------------
% Multiple-Choice Questions for (101.3)
%-------------------------------------------------------
\newpage
\section*{Multiple-Choice Questions for 101.3}

\begin{enumerate}[1.]

    \item Which file traditionally defines the default runlevel in a SysVinit system?
    \begin{enumerate}[A)]
        \item \texttt{/etc/inittab}
        \item \texttt{/etc/rc.conf}
        \item \texttt{/etc/systemd/system.conf}
        \item \texttt{/etc/default/runlevel}
    \end{enumerate}
    
    \item In SysVinit, which runlevel usually corresponds to \textbf{system restart}?
    \begin{enumerate}[A)]
        \item Runlevel 1
        \item Runlevel 3
        \item Runlevel 5
        \item Runlevel 6
    \end{enumerate}
    
    \item Which command is used on a SysVinit system to \textbf{check the current runlevel}?
    \begin{enumerate}[A)]
        \item \texttt{who -r}
        \item \texttt{runlevel}
        \item \texttt{init}
        \item \texttt{sysvcheck}
    \end{enumerate}
    
    \item On a SysVinit system, which \textbf{runlevel} is typically reserved for \textbf{multi-user mode without a graphical environment}?
    \begin{enumerate}[A)]
        \item Runlevel 0
        \item Runlevel 1
        \item Runlevel 3
        \item Runlevel 6
    \end{enumerate}
    
    \item Which command \textbf{reloads} the \texttt{/etc/inittab} file after changes are made (on a SysVinit system)?
    \begin{enumerate}[A)]
        \item \texttt{telinit q}
        \item \texttt{init reload}
        \item \texttt{systemctl daemon-reload}
        \item \texttt{reload runlevel}
    \end{enumerate}
    
    \item Which \textbf{systemd unit type} is used for grouping other units so they can be controlled as a single entity?
    \begin{enumerate}[A)]
        \item service
        \item automount
        \item target
        \item socket
    \end{enumerate}
    
    \item On a \textbf{systemd} system, which command would you use to \textbf{switch} the system to \texttt{multi-user.target} immediately?
    \begin{enumerate}[A)]
        \item \texttt{systemctl default multi-user.target}
        \item \texttt{systemctl multi-user.target}
        \item \texttt{systemctl reload multi-user.target}
        \item \texttt{systemctl isolate multi-user.target}
    \end{enumerate}
    
    \item Which command is commonly used on SysVinit systems to \textbf{change} the current runlevel \textbf{without} rebooting?
    \begin{enumerate}[A)]
        \item \texttt{systemctl isolate}
        \item \texttt{telinit}
        \item \texttt{initctrl}
        \item \texttt{switchrun}
    \end{enumerate}
    
    \item In a SysVinit layout, scripts in directories like \texttt{/etc/rc3.d/} typically \textbf{start} with what letter if they are launched upon entering that runlevel?
    \begin{enumerate}[A)]
        \item R
        \item G
        \item S
        \item T
    \end{enumerate}
    
    \item Which \textbf{runlevel} or mode is typically used for \textbf{maintenance} when the system is only available to the administrator (no network services)?
    \begin{enumerate}[A)]
        \item Single-user (Runlevel 1)
        \item Graphical mode (Runlevel 5)
        \item Multi-user mode (Runlevel 3)
        \item Runlevel 2
    \end{enumerate}

    \item Which \textbf{SysVinit} command can be used to \textbf{halt} the system, after modifying the \texttt{/etc/inittab} entry for Ctrl+Alt+Del with the \texttt{-a} option?
    \begin{enumerate}[A)]
        \item \texttt{halt -a}
        \item \texttt{shutdown}
        \item \texttt{poweroff}
        \item \texttt{stop system}
    \end{enumerate}
    
    \item Which \textbf{systemctl} command would you use to \textbf{turn off} the system immediately on a \textbf{systemd} host?
    \begin{enumerate}[A)]
        \item \texttt{systemctl shutdown}
        \item \texttt{systemctl down}
        \item \texttt{systemctl isolate runlevel0.target}
        \item \texttt{systemctl poweroff}
    \end{enumerate}
    
    \item Which \textbf{systemd} unit type is used for hardware devices identified by the kernel?
    \begin{enumerate}[A)]
        \item target
        \item service
        \item device
        \item mount
    \end{enumerate}
    
    \item Which file is \textbf{not} used by \textbf{systemd} to set the default system target?
    \begin{enumerate}[A)]
        \item \texttt{/etc/systemd/system/default.target}
        \item \texttt{/lib/systemd/system/multi-user.target}
        \item \texttt{/lib/systemd/system/graphical.target}
        \item \texttt{/etc/inittab}
    \end{enumerate}
    
    \item If you see the output \texttt{tty5 start/running, process 1764} on an Ubuntu system, which \textbf{init system} is likely in use?
    \begin{enumerate}[A)]
        \item SysVinit
        \item Upstart
        \item systemd
        \item OpenRC
    \end{enumerate}
    
    \item On a \textbf{systemd} system, which command \textbf{reboots} the machine?
    \begin{enumerate}[A)]
        \item \texttt{systemctl shutdown -r}
        \item \texttt{systemctl kill}
        \item \texttt{systemctl isolate reboot.target}
        \item \texttt{systemctl reboot}
    \end{enumerate}
    
    \item Which \textbf{systemd} unit type is used to define an on-demand mount point?
    \begin{enumerate}[A)]
        \item device
        \item service
        \item socket
        \item automount
    \end{enumerate}
    
    \item Which \textbf{Upstart} command is used to \textbf{stop} a currently running job or service?
    \begin{enumerate}[A)]
        \item \texttt{upstartctl kill}
        \item \texttt{stop}
        \item \texttt{service halt}
        \item \texttt{haltjob}
    \end{enumerate}
    
    \item Which command is typically used to \textbf{send a message} to all logged-in users’ terminals?
    \begin{enumerate}[A)]
        \item \texttt{wall}
        \item \texttt{announce}
        \item \texttt{globalmsg}
        \item \texttt{bcast}
    \end{enumerate}
    
    \item In the \textbf{SysVinit} scheme, which directory contains startup scripts (symbolic links) specifically for \textbf{runlevel 2}?
    \begin{enumerate}[A)]
        \item \texttt{/etc/init.d2/}
        \item \texttt{/etc/rc.d/2/}
        \item \texttt{/etc/rc2.d/}
        \item \texttt{/etc/sysvinit/2/}
    \end{enumerate}
    
\end{enumerate}


%-------------------------------------------------------
% Fill-in-the-Blank Questions (101.3)
%-------------------------------------------------------
\newpage
\section*{Fill-in-the-Blank Questions for 101.3}

\begin{enumerate}[1.]

\item In a \textbf{SysVinit} system, the default runlevel is configured in the file \textbf{\underline{\hspace{2cm}}}.

\item To switch the system to \textbf{single-user mode} (runlevel 1) on a SysVinit system, you can type \textbf{\underline{\hspace{2cm}} 1} or \textbf{\underline{\hspace{2cm}} s}.

\item The command \textbf{\underline{\hspace{2cm}} q} is used to make \textbf{init} re-read the \textbf{/etc/inittab} file after changes are made.

\item In \textbf{System V} style initialization, scripts controlling services are located in \textbf{\underline{\hspace{2cm}}}, while each runlevel (e.g., runlevel 3, 5) has its own subdirectory like \textbf{/etc/rc3.d/} or \textbf{/etc/rc5.d/}.

\item Under \textbf{systemd}, each background process or subsystem is referred to as a \textbf{\underline{\hspace{2cm}}} (e.g., \textbf{httpd.service}).

\item To change the \textbf{default target} in \textbf{systemd} without editing kernel parameters directly, you can use the command \textbf{systemctl set-default \underline{\hspace{2cm}}.target}.

\item In \textbf{systemd}, if you want to switch to \textbf{multi-user mode} without rebooting, you can execute \textbf{systemctl \underline{\hspace{2cm}} multi-user.target}.

\item When switching from \textbf{Upstart}, Ubuntu replaced its init system with \textbf{\underline{\hspace{2cm}}}.

\item The \textbf{\underline{\hspace{2cm}}} command sends a message to the terminal sessions of all logged-in users and is useful before shutting down or switching runlevels.

\item In a \textbf{SysVinit} system, \textbf{Runlevel 0} corresponds to \textbf{\underline{\hspace{2cm}}}, while \textbf{Runlevel 6} corresponds to a \textbf{restart} of the system.

\end{enumerate}

%=======================================================
% TOPIC 102: LINUX INSTALLATION AND PACKAGE MANAGEMENT
%=======================================================
\chapter{Topic 102: Linux Installation and Package Management}

%-------------------------------------------------------
% 102.1 Design hard disk layout
%-------------------------------------------------------
\newpage

\section*{102.1 Design Hard Disk Layout}
\addcontentsline{toc}{section}{102.1 Design Hard Disk Layout}


\textbf{Reference to LPI Objectives:}
\begin{itemize}
    \item \textbf{LPIC-1 v5, Exam 102, Objective 102.1}
    \item \textbf{Weight:} 2
\end{itemize}

\subsection*{Key Knowledge Areas}
\begin{itemize}
    \item Allocating filesystems and swap space to separate partitions or disks.
    \item Tailoring the partitioning scheme to system usage.
    \item Understanding \texttt{/boot} or EFI System Partition requirements for booting.
    \item Basic familiarity with LVM (Logical Volume Manager).
\end{itemize}

\subsection*{Important Terms and Utilities}
\begin{itemize}
    \item \texttt{/ (root)}, \texttt{/boot}, \texttt{/home}, \texttt{/var}
    \item \textbf{EFI System Partition (ESP)}
    \item \textbf{swap}
    \item \textbf{mount points} (e.g., \texttt{/mnt}, \texttt{/media/USER/LABEL})
    \item \textbf{partitions} and \textbf{logical volumes}
    \item \textbf{LVM} (Physical Volumes, Volume Groups, Logical Volumes)
\end{itemize}

\section*{Lesson Overview}

Designing an effective disk layout is critical for system stability, performance, and ease of administration. You must understand partitions, filesystems, mount points, swap, and how LVM can simplify storage allocation.

\section*{1. Partitions, Filesystems, and Mount Points}

\begin{enumerate}
    \item \textbf{Partitions}
    \begin{itemize}
        \item Logical “fences” on a disk; each partition has its own filesystem.
        \item Partition information is stored in the \textbf{partition table}.
        \item Partitions \textbf{cannot} span multiple disks (unless using LVM or RAID).
    \end{itemize}

    \item \textbf{Filesystems}
    \begin{itemize}
        \item Define how data is organized in directories, files, and metadata.
        \item Must be \textbf{mounted} on a mount point (e.g., \texttt{/mnt/mydata}).
    \end{itemize}

    \item \textbf{Mount Points}
    \begin{itemize}
        \item Directory where a filesystem is attached.
        \item Common directories:
        \begin{itemize}
            \item \texttt{/mnt/} – traditional manual mount point.
            \item \texttt{/media/} – automatic mounting of removable media.
        \end{itemize}
        \item Existing contents of a mount point become hidden while another filesystem is mounted.
    \end{itemize}
\end{enumerate}

\section*{2. Recommended Partitions and Their Uses}

\begin{enumerate}
    \item \textbf{Root Partition (\texttt{/})}
    \begin{itemize}
        \item Base of the Linux directory structure.
        \item Typically holds OS binaries and system config if not separated elsewhere.
    \end{itemize}

    \item \textbf{\texttt{/boot} or EFI System Partition (ESP)}
    \begin{itemize}
        \item \texttt{/boot} stores bootloader files (kernel images, initramfs, GRUB).
        \item ESP is used on UEFI systems (formatted as FAT).
        \item Usually 200–300 MB in size is sufficient for either.
        \item Keeping boot files separate can help ensure the system can still boot if root is damaged.
    \end{itemize}

    \item \textbf{\texttt{/home}}
    \begin{itemize}
        \item Houses users’ personal files and preferences.
        \item Separating \texttt{/home} allows OS reinstallation without erasing user data.
        \item Size depends on user data and expected usage.
    \end{itemize}

    \item \textbf{\texttt{/var}}
    \begin{itemize}
        \item Contains variable data: logs (\texttt{/var/log}), caches (\texttt{/var/cache}), temp data (\texttt{/var/tmp}), etc.
        \item On servers, \texttt{/var} can grow significantly (e.g., web or database data).
        \item Putting \texttt{/var} on a separate partition (or disk) improves stability and prevents root from filling up.
    \end{itemize}

    \item \textbf{Swap}
    \begin{itemize}
        \item Extension of RAM to disk; cannot be mounted as a normal directory.
        \item Often sized according to usage (e.g., old rule was 2×RAM; modern guidelines vary).
        \item Consider \textbf{hibernation} requirements (swap $\geq$ RAM if hibernation is used).
    \end{itemize}
\end{enumerate}

\section*{3. LVM (Logical Volume Manager)}

\begin{enumerate}
    \item \textbf{Overview}
    \begin{itemize}
        \item Provides flexible “virtual” partitions called \textbf{Logical Volumes (LVs)}.
        \item \textbf{Physical Volumes (PVs)} $\to$ grouped into \textbf{Volume Groups (VGs)} $\to$ split into \textbf{Logical Volumes (LVs)}.
        \item LVM allows resizing or adding storage more easily than traditional partitions.
    \end{itemize}

    \item \textbf{Advantages}
    \begin{itemize}
        \item \textbf{Ease of extension:} add space without reformatting or migrating data.
        \item \textbf{Abstracts} underlying physical disks.
        \item Logical volumes appear in \texttt{/dev/VG\_NAME/LV\_NAME}.
    \end{itemize}

    \item \textbf{Basic Workflow (High-level)}
    \begin{enumerate}
        \item Create or identify a \textbf{partition} (or entire disk) as a PV (\texttt{pvcreate /dev/sdaX}).
        \item Combine PVs into a \textbf{Volume Group} (\texttt{vgcreate MYVG /dev/sdaX}).
        \item Create a \textbf{Logical Volume} (\texttt{lvcreate -L 20G -n MYSERVICELV MYVG}).
        \item Format LV with a filesystem (\texttt{mkfs.ext4 /dev/MYVG/MYSERVICELV}).
        \item Mount where desired (\texttt{/etc/fstab} entry or \texttt{mount} command).
    \end{enumerate}
\end{enumerate}

\section*{Workbook Exercises}

\begin{enumerate}
    \item \textbf{Plan a Basic Partition Scheme}
    \begin{itemize}
        \item Imagine you have a 500 GB disk for a personal workstation.
        \item Sketch out your proposed partition table: \texttt{/boot} (300 MB), root (\texttt{/}), \texttt{/home}, \texttt{/var}, and swap.
        \item Consider sizes for each partition and justify your choices.
    \end{itemize}

    \item \textbf{Identify ESP/BIOS Partitions}
    \begin{itemize}
        \item On a UEFI-based system, locate and identify the \textbf{EFI System Partition} (\texttt{/boot/efi}).
        \item Check partition type using \texttt{gdisk -l /dev/sda} or \texttt{fdisk -l /dev/sda}.
        \item Verify its filesystem (FAT-based) with \texttt{lsblk -f} or \texttt{blkid}.
    \end{itemize}

    \item \textbf{Decide on Swap Size}
    \begin{itemize}
        \item If your system has 8 GB of RAM, use Red Hat’s guidelines to propose a recommended swap size.
        \item If planning hibernation, recalculate.
    \end{itemize}

    \item \textbf{Mount Points}
    \begin{itemize}
        \item Create a directory \texttt{/mnt/testmount}.
        \item Using an existing spare partition (or loopback device), manually mount it on \texttt{/mnt/testmount}.
        \item Verify it is mounted with \texttt{mount | grep /mnt/testmount}.
    \end{itemize}

    \item \textbf{LVM Planning}
    \begin{itemize}
        \item Using a virtual environment with two disks, plan an LVM layout:
        \begin{enumerate}
            \item Convert one partition from each disk into PVs.
            \item Create a Volume Group that spans both.
            \item Create one or more Logical Volumes for \texttt{/data}.
        \end{enumerate}
        \item Write down how you will format and mount \texttt{/data}.
    \end{itemize}

    \item \textbf{Storage Scenarios}
    \begin{itemize}
        \item You run out of disk space on \texttt{/home}. What steps can you take with LVM to add more space?
        \item If \texttt{/var} was not separated and you frequently run out of space due to logs, how might you redesign?
    \end{itemize}
\end{enumerate}

\section*{Summary}

\begin{itemize}
    \item \textbf{Partitions} define logical divisions of a disk, while \textbf{filesystems} define how data is stored.
    \item Strategic partitioning improves \textbf{stability, security, and maintenance} (e.g., \texttt{/boot}, \texttt{/home}, \texttt{/var} separate).
    \item \textbf{UEFI} systems require an \textbf{EFI System Partition (ESP)}; BIOS systems may benefit from a separate \texttt{/boot}.
    \item Adequate \textbf{swap} is essential; guidelines depend on RAM, system usage, and whether hibernation is used.
    \item \textbf{LVM} adds flexibility for resizing and pooling storage among multiple physical disks.
\end{itemize}


%-------------------------------------------------------
% Multiple-Choice Questions for (102.1)
%-------------------------------------------------------

\newpage
\section*{Multiple-Choice Questions for 102.1}

\begin{enumerate}[1.]

    \item Which statement best describes the purpose of a \textbf{partition table} on a disk?  
    \begin{enumerate}[A)]
        \item It is the directory on the filesystem that contains user data  
        \item It is the bootloader used to start the operating system  
        \item It contains information about the sectors and types of partitions on the disk  
        \item It is the tool used to create logical volumes  
    \end{enumerate}

    \item On most Linux distributions, \textbf{removable media} (e.g., USB drives, memory cards) are:  
    \begin{enumerate}[A)]
        \item Automatically mounted under \texttt{/media/USER/LABEL}  
        \item Expected to be manually mounted under \texttt{/opt}  
        \item Required to be unmounted only after a reboot  
        \item Limited to read-only access by default  
    \end{enumerate}

    \item What is one \textbf{benefit} of creating a dedicated \texttt{/boot} partition?  
    \begin{enumerate}[A)]
        \item It automatically encrypts the disk to improve security  
        \item It merges multiple storage devices into a single volume  
        \item It allows \texttt{/home} to remain untouched during upgrades  
        \item It ensures the system can still boot if the root filesystem is corrupted  
    \end{enumerate}

    \item Which of the following is a \textbf{directory} that frequently benefits from being put on a separate partition, due to potential growth of log files and database data?  
    \begin{enumerate}[A)]
        \item \texttt{/bin}  
        \item \texttt{/var}  
        \item \texttt{/root}  
        \item \texttt{/tmp}  
    \end{enumerate}

    \item The \textbf{EFI System Partition (ESP)} on a UEFI-based system is typically:  
    \begin{enumerate}[A)]
        \item Formatted with an ext4 filesystem  
        \item Formatted with a FAT-based filesystem  
        \item Placed at the end of the disk to maximize disk space  
        \item Labeled as \texttt{/swap} in the \texttt{/etc/fstab} file  
    \end{enumerate}

    \item What is the primary \textbf{purpose} of the \textbf{swap partition}?  
    \begin{enumerate}[A)]
        \item To store user home directories  
        \item To store kernel bootloader files for GRUB  
        \item To hold system logs and database files  
        \item To provide virtual memory (swap out pages from RAM)  
    \end{enumerate}

    \item Which of the following is \textbf{NOT} a reason to create a separate \texttt{/home} partition?  
    \begin{enumerate}[A)]
        \item To prevent boot issues on legacy BIOS systems  
        \item To simplify system reinstallation while keeping user data  
        \item To allow for different filesystem choices for user data  
        \item To avoid overwriting user data during an OS upgrade  
    \end{enumerate}

    \item A \textbf{mount point} is best described as:  
    \begin{enumerate}[A)]
        \item A command used to set the filesystem read-only  
        \item A method to copy data from one partition to another  
        \item A directory where a filesystem is attached to the directory tree  
        \item A device driver for SATA disks  
    \end{enumerate}

    \item Which directory commonly holds the \textbf{Apache Web Server} data by default?  
    \begin{enumerate}[A)]
        \item \texttt{/home/apache}  
        \item \texttt{/var/www/html}  
        \item \texttt{/srv/httpd}  
        \item \texttt{/apache/www}  
    \end{enumerate}

    \item Under \textbf{Logical Volume Management (LVM)}, which statement is correct about \textbf{Volume Groups (VGs)}?  
    \begin{enumerate}[A)]
        \item VGs cannot be reduced in size after creation  
        \item VGs are used only for encrypting partitions  
        \item VGs aggregate multiple physical volumes into one large pool of storage  
        \item VGs must reside on a separate \texttt{/vg} partition  
    \end{enumerate}

    \item Which of the following directories is typically \textbf{not} located under \texttt{/home} on a Linux system?  
    \begin{enumerate}[A)]
        \item \texttt{/root} (the home directory for the root user)  
        \item \texttt{/john} (a normal user’s home directory)  
        \item \texttt{/jack} (a normal user’s home directory)  
        \item \texttt{/carol} (a normal user’s home directory)  
    \end{enumerate}

    \item Which \textbf{bootloader} is most commonly used on modern Linux systems for loading the operating system?  
    \begin{enumerate}[A)]
        \item syslinux  
        \item LILO  
        \item rEFIt  
        \item GRUB2  
    \end{enumerate}

    \item The EFI System Partition (ESP) is generally \textbf{mounted} under which directory on a Linux system?  
    \begin{enumerate}[A)]
        \item \texttt{/boot}  
        \item \texttt{/ESP}  
        \item \texttt{/boot/efi}  
        \item \texttt{/usr/local/esp}  
    \end{enumerate}

    \item Which directory contains “variable data” and is known to grow due to logs and other services?  
    \begin{enumerate}[A)]
        \item \texttt{/etc}  
        \item \texttt{/var}  
        \item \texttt{/mnt}  
        \item \texttt{/bin}  
    \end{enumerate}

    \item Which of the following statements correctly describes \textbf{Physical Volumes (PVs)} in LVM?  
    \begin{enumerate}[A)]
        \item They are the mount points for the swap partition  
        \item They are user directories that can be encrypted  
        \item They are used only for backing up the master boot record  
        \item They are block devices (partitions, disks) that become part of a volume group  
    \end{enumerate}

    \item Why might a system administrator place \texttt{/home} and \texttt{/var} on \textbf{separate physical disks}?  
    \begin{enumerate}[A)]
        \item To reduce the impact of one disk’s failure on another  
        \item To comply with GPT partition ID requirements  
        \item To ensure that \texttt{/boot} remains fully accessible  
        \item To use a unique journaling filesystem only on \texttt{/home}  
    \end{enumerate}

    \item \textbf{Logical Volumes (LVs)} in an LVM setup appear to the operating system as:  
    \begin{enumerate}[A)]
        \item Symbolic links pointing to \texttt{/boot}  
        \item BIOS-level CHS addresses  
        \item A directory tree with unlimited capacity  
        \item Normal block devices (e.g., \texttt{/dev/VGNAME/LVNAME})  
    \end{enumerate}

    \item If you need to \textbf{manually} mount an additional filesystem that is \textbf{not} removable media, the best practice is typically to mount it under which directory?  
    \begin{enumerate}[A)]
        \item \texttt{/var}  
        \item \texttt{/home}  
        \item \texttt{/mnt}  
        \item \texttt{/opt/extra}  
    \end{enumerate}

    \item According to common recommendations, placing system logs and database files under a dedicated \texttt{/var} partition helps:  
    \begin{enumerate}[A)]
        \item Prevent swap space from being used  
        \item Protect the root filesystem if \texttt{/var} fills up  
        \item Force all logs to be read-only  
        \item Make the system boot faster  
    \end{enumerate}

    \item Which of the following describes a \textbf{swap file}?  
    \begin{enumerate}[A)]
        \item A regular file residing on an existing filesystem, configured to provide additional swap space  
        \item A partition created at the beginning of the disk and labeled as \texttt{0xEF}  
        \item A mount point used only for kernel boot files  
        \item A special utility run by the BIOS to load the operating system  
    \end{enumerate}

\end{enumerate}
%-------------------------------------------------------
% Fill-in-the-Blank Questions (102.1)
%-------------------------------------------------------
\newpage
\section*{Fill-in-the-Blank Questions for 102.1}

\begin{enumerate}[1.]

\item The folder where user data files and preferences are typically stored is \underline{\hspace{2cm}}.

\item The BIOS limitation of 1024 cylinders initially required a dedicated \underline{\hspace{2cm}} partition to be placed at the beginning of the disk for bootloader files.

\item A \underline{\hspace{2cm}} is the logical subdivision of a physical disk, storing data separately from other subdivisions on the same disk.

\item On systems with UEFI firmware, the \underline{\hspace{2cm}} partition stores boot loader and kernel files in a FAT-based filesystem.

\item The command that prepares a partition for use as swap space is \underline{\hspace{2cm}}.

\item The directory \underline{\hspace{2cm}} is used for “variable data” such as logs, database files, caches, and spool directories.

\item A filesystem must be “attached” to the system via a process called \underline{\hspace{2cm}} before you can access its contents.

\item In LVM, multiple Physical Volumes can be combined to form a larger \underline{\hspace{2cm}}.

\item A \underline{\hspace{2cm}} is a file that can serve as additional swap space without requiring a dedicated swap partition.

\item Traditional partitioning can be inflexible when allocating disk space. One way to overcome this limitation is by using \underline{\hspace{2cm}}.

\end{enumerate}

%-------------------------------------------------------
% 102.2 Install a boot manager
%-------------------------------------------------------

\newpage

\section*{102.2 Install a Boot Manager}
\addcontentsline{toc}{section}{102.2 Install a Boot Manager}

\textbf{Reference to LPI Objectives:}
\begin{itemize}
    \item \textbf{LPIC-1 v5, Exam 102, Objective 102.2}
    \item \textbf{Weight:} 2
\end{itemize}

\subsection*{Key Knowledge Areas}
\begin{itemize}
    \item Providing alternate or backup boot options.
    \item Installing and configuring boot loaders (GRUB Legacy, GRUB 2).
    \item Performing basic GRUB 2 configuration changes.
    \item Interacting with the boot loader at startup.
\end{itemize}

\subsection*{Important Files, Terms, and Utilities}
\begin{itemize}
    \item \textbf{MBR} (Master Boot Record)
    \item \texttt{/boot} directory or partition (often containing GRUB files, kernels, initrd)
    \item \texttt{menu.lst}, \texttt{grub.cfg}, and \texttt{grub.conf}
    \item \texttt{grub-install}, \texttt{grub-mkconfig} (or \texttt{update-grub})
    \item \textbf{chainloading} (for non-Linux OS, e.g., Windows)
\end{itemize}

\section*{Lesson Overview}

A system’s boot loader is the first software executed when a machine powers on. On Linux, this is typically \textbf{GRUB} (either Legacy or GRUB 2). GRUB loads the kernel and passes control to it. Having a working knowledge of installing and configuring GRUB is essential for system recovery and customizing boot behavior.

\section*{1. GRUB Legacy vs. GRUB 2}

\begin{enumerate}
    \item \textbf{GRUB Legacy}
    \begin{itemize}
        \item Older, no longer actively developed (last release 0.97 from 2005).
        \item Configuration file: \texttt{/boot/grub/menu.lst} (sometimes \texttt{grub.conf}).
        \item Simpler configuration, fewer features.
    \end{itemize}

    \item \textbf{GRUB 2}
    \begin{itemize}
        \item Complete rewrite, default on most modern distributions.
        \item Configuration files:
        \begin{itemize}
            \item \texttt{/etc/default/grub} (main user-editable file)
            \item \texttt{/boot/grub/grub.cfg} (auto-generated, do not edit manually)
        \end{itemize}
        \item More modular, supports more filesystems, better scripting, theming, etc.
    \end{itemize}
\end{enumerate}

\section*{2. Bootloader Locations and Partitions}

\begin{enumerate}
    \item \textbf{MBR Partition Scheme}
    \begin{itemize}
        \item Legacy layout where the first 512 bytes of the disk contain the MBR (boot code + partition table).
        \item Boot loader code often placed in MBR + post-MBR gap (32 KB) before the first partition.
    \end{itemize}

    \item \textbf{GPT (GUID Partition Table)}
    \begin{itemize}
        \item Modern layout for large disks ($>$2 TB).
        \item Requires a \textbf{BIOS boot partition} (for BIOS systems) or uses \textbf{EFI System Partition (ESP)} on UEFI systems.
    \end{itemize}

    \item \textbf{/boot Partition}
    \begin{itemize}
        \item Often first partition on the disk, historically to avoid BIOS cylinder limits.
        \item Typically $\sim$300 MB in size, containing kernel images (\texttt{vmlinuz}), initrd, GRUB files, etc.
        \item Helps ensure boot files remain accessible (especially if \texttt{/} uses encryption or an unsupported filesystem).
    \end{itemize}
\end{enumerate}

\section*{3. Installing GRUB 2}

\begin{enumerate}
    \item \textbf{\texttt{grub-install}}
    \begin{itemize}
        \item Installs GRUB 2 boot code onto a disk (e.g., \texttt{/dev/sda}) or EFI partition.
        \item Syntax examples:
        \begin{lstlisting}[language=bash]
grub-install --boot-directory=/boot /dev/sda
# or for a dedicated /boot partition mounted at /mnt/tmp:
grub-install --boot-directory=/mnt/tmp /dev/sda
        \end{lstlisting}
        \item Must point to the \textbf{disk} (e.g., \texttt{/dev/sda}), not a specific partition (unless UEFI requires otherwise).
    \end{itemize}

    \item \textbf{Configuration}
    \begin{itemize}
        \item \texttt{/etc/default/grub} – main file for user edits. Common parameters:
        \begin{itemize}
            \item \texttt{GRUB\_DEFAULT}: default menu entry (0-based index, or \texttt{saved}).
            \item \texttt{GRUB\_SAVEDEFAULT}: if set to \texttt{true} with \texttt{GRUB\_DEFAULT=saved}, boots the last-chosen entry.
            \item \texttt{GRUB\_TIMEOUT}: seconds before auto-booting the default. \texttt{-1} waits indefinitely.
            \item \texttt{GRUB\_CMDLINE\_LINUX}: universal kernel parameters (e.g., \texttt{quiet}, \texttt{splash}).
        \end{itemize}
        \item \texttt{grub-mkconfig} (or \texttt{update-grub}) generates \texttt{/boot/grub/grub.cfg} from the above file and scripts in \texttt{/etc/grub.d/}:
        \begin{lstlisting}[language=bash]
grub-mkconfig -o /boot/grub/grub.cfg
# or:
update-grub
        \end{lstlisting}
    \end{itemize}

    \item \textbf{Menu Entries}
    \begin{itemize}
        \item Auto-discovered for Linux, other OS, or kernels.
        \item Custom entries often added to \texttt{/etc/grub.d/40\_custom}, then re-run \texttt{update-grub}.
    \end{itemize}

    \item \textbf{Interacting with GRUB 2}
    \begin{itemize}
        \item \textbf{Boot Menu:} highlight an entry with arrow keys, press \textbf{e} to edit before booting.
        \item \textbf{Shell Mode:} press \textbf{c} to access \texttt{grub>} shell.
        \item \textbf{Rescue Shell} (\texttt{grub rescue>}): minimal commands, must \texttt{insmod} needed modules (e.g., \texttt{normal}, \texttt{linux}) if GRUB config is broken.
    \end{itemize}
\end{enumerate}

\section*{4. GRUB Legacy (for Reference)}

\begin{enumerate}
    \item \textbf{Installing}
    \begin{itemize}
        \item Via \texttt{grub-install /dev/sda} (must specify the disk, not a partition).
        \item From GRUB Legacy shell:
        \begin{lstlisting}[language=bash]
grub> root (hd0,0)
grub> setup (hd0)
        \end{lstlisting}
        \item \texttt{root (hd0,0)} means the first disk (\texttt{hd0}), first partition (\texttt{0}), if \texttt{/boot} is there.
    \end{itemize}

    \item \textbf{Configuration: \texttt{/boot/grub/menu.lst}}
    \begin{itemize}
        \item Example menu entry:
        \begin{lstlisting}
title My Linux
root (hd0,0)
kernel /vmlinuz root=/dev/hda1
initrd /initrd.img
        \end{lstlisting}
        \item \texttt{chainloader +1} used to boot Windows or other OS by loading their own bootloader code.
    \end{itemize}
\end{enumerate}

\section*{5. Booting from the GRUB Shell}

\begin{enumerate}
    \item \textbf{Identify Partitions}:
    \begin{lstlisting}[language=bash]
grub> ls
(hd0) (hd0,msdos1)
    \end{lstlisting}

    \item \textbf{Set root} (example):
    \begin{lstlisting}[language=bash]
grub> set root=(hd0,msdos1)
    \end{lstlisting}

    \item \textbf{Load Kernel \& Initrd} (GRUB 2 example):
    \begin{lstlisting}[language=bash]
grub> linux /vmlinuz root=/dev/sda1
grub> initrd /initrd.img
grub> boot
    \end{lstlisting}

    \item \textbf{Rescue Mode}: need to \texttt{set prefix=(hd0,msdos1)/boot/grub} and \texttt{insmod normal}, \texttt{insmod linux} before proceeding.
\end{enumerate}

\section*{Workbook Exercises}

\begin{enumerate}
    \item \textbf{Identify Boot Device}
    \begin{itemize}
        \item Run \texttt{fdisk -l /dev/sda} or \texttt{lsblk -f} and find your \textbf{boot partition}.
        \item Note which partition is marked as bootable.
    \end{itemize}

    \item \textbf{Install GRUB 2}
    \begin{itemize}
        \item Mount your \texttt{/boot} (or boot partition) if needed at \texttt{/mnt/tmp}.
        \item Run:
        \begin{lstlisting}[language=bash]
grub-install --boot-directory=/mnt/tmp /dev/sda
        \end{lstlisting}
        \item Verify GRUB files are placed in \texttt{/mnt/tmp/boot/grub}.
    \end{itemize}

    \item \textbf{Customize GRUB Timeout}
    \begin{itemize}
        \item Edit \texttt{/etc/default/grub} and set \texttt{GRUB\_TIMEOUT=5}.
        \item Run \texttt{update-grub} (or \texttt{grub-mkconfig -o /boot/grub/grub.cfg}).
        \item Reboot and confirm you see the menu for 5 seconds.
    \end{itemize}

    \item \textbf{Add a Kernel Parameter}
    \begin{itemize}
        \item In \texttt{/etc/default/grub}, add an option to \texttt{GRUB\_CMDLINE\_LINUX="quiet splash"}.
        \item Update GRUB and reboot. Check \texttt{/proc/cmdline} to confirm the new parameter.
    \end{itemize}

    \item \textbf{Practice Chainloading}
    \begin{itemize}
        \item If you have a Windows install at \texttt{(hd0,2)}, add a custom entry in \texttt{/etc/grub.d/40\_custom} (or in GRUB Legacy’s \texttt{menu.lst}):
        \begin{lstlisting}
menuentry "Windows" {
    set root=(hd0,2)
    chainloader +1
}
        \end{lstlisting}
        \item Update GRUB and verify you can boot into Windows.
    \end{itemize}

    \item \textbf{GRUB Rescue Simulation}
    \begin{itemize}
        \item Temporarily rename \texttt{/boot/grub/grub.cfg} to break GRUB.
        \item Reboot to force the \texttt{grub rescue>} prompt.
        \item Use \texttt{ls}, \texttt{set prefix=}, \texttt{insmod normal}, etc., to recover manually.
        \item Restore \texttt{grub.cfg} after testing.
    \end{itemize}
\end{enumerate}

\section*{Summary}

\begin{itemize}
    \item \textbf{GRUB 2} is the modern bootloader on most Linux systems, replacing \textbf{GRUB Legacy}.
    \item \texttt{grub-install} places boot code on the MBR (BIOS) or ESP (UEFI).
    \item \texttt{/etc/default/grub} and scripts in \texttt{/etc/grub.d/} define the GRUB 2 menu.
    \item Use \texttt{update-grub} (or \texttt{grub-mkconfig}) to regenerate \texttt{/boot/grub/grub.cfg}.
    \item In emergencies, the \textbf{GRUB shell} (normal or rescue) can manually load kernel and initrd to boot.
\end{itemize}




%-------------------------------------------------------
% Multiple-Choice Questions for (102.2)
%-------------------------------------------------------
\newpage
\section*{Multiple-Choice Questions for 102.2}

\begin{enumerate}[1.]

    \item Which of the following statements correctly describes the purpose of the Master Boot Record (MBR)?  
    \begin{enumerate}[A)]
        \item It is a legacy BIOS setting used to store GPU firmware information.  
        \item It is a reserved partition for the \texttt{/boot} directory on GPT disks.  
        \item It contains a high-level overview of the Linux file system hierarchy.  
        \item It contains a partition table and minimal bootstrap code used to start the boot process.  
    \end{enumerate}

    \item Which command is commonly used to generate a new GRUB 2 configuration file?  
    \begin{enumerate}[A)]
        \item \texttt{grub-init /dev/sda}  
        \item \texttt{update-grub} (or \texttt{grub-mkconfig -o /boot/grub/grub.cfg})  
        \item \texttt{mkconfig -g}  
        \item \texttt{grub-legacy -o /boot/grub/menu.lst}  
    \end{enumerate}

    \item In GRUB 2, which file is recommended for manual editing to permanently change bootloader settings?  
    \begin{enumerate}[A)]
        \item \texttt{/boot/grub/menu.lst}  
        \item \texttt{/etc/grub.d/custom.cfg}  
        \item \texttt{/etc/default/grub}  
        \item \texttt{/boot/grub/grub.cfg}  
    \end{enumerate}

    \item Which directive in \texttt{/etc/default/grub} allows the last chosen boot option to become the default on the next reboot?  
    \begin{enumerate}[A)]
        \item \texttt{GRUB\_SAVEDEFAULT=}  
        \item \texttt{GRUB\_ENABLE\_CRYPTODISK=}  
        \item \texttt{GRUB\_TIMEOUT=}  
        \item \texttt{GRUB\_CMDLINE\_LINUX=}  
    \end{enumerate}

    \item What is the main purpose of an initial RAM disk (initrd)?  
    \begin{enumerate}[A)]
        \item It is another name for the GRUB 2 configuration file.  
        \item It is a dedicated swap partition to store kernel crash dumps.  
        \item It compresses the kernel to reduce its size in \texttt{/boot}.  
        \item It provides a minimal root filesystem so the kernel can mount the real root filesystem.  
    \end{enumerate}

    \item Which GRUB 2 command is used in a manual (interactive) session to load the Linux kernel?  
    \begin{enumerate}[A)]
        \item \texttt{linux}  
        \item \texttt{initrd}  
        \item \texttt{chainloader}  
        \item \texttt{set root=}  
    \end{enumerate}

    \item Which of the following is \textbf{not} true regarding GPT?  
    \begin{enumerate}[A)]
        \item It is part of the UEFI specification.  
        \item It is incompatible with BIOS-based systems.  
        \item It supports disks larger than 2 TB.  
        \item It uses GUIDs to identify partitions.  
    \end{enumerate}

    \item When chainloading Windows from GRUB Legacy or GRUB 2, which command loads the Windows bootloader from the first sector of its partition?  
    \begin{enumerate}[A)]
        \item \texttt{initrd /windows.img}  
        \item \texttt{search --set=root --fs-uuid <uuid>}  
        \item \texttt{chainloader +1}  
        \item \texttt{module /boot/grub/i386-pc/chain.mod}  
    \end{enumerate}

    \item In GRUB Legacy, which file typically stores the boot menu configuration?  
    \begin{enumerate}[A)]
        \item \texttt{/boot/grub/menu.lst}  
        \item \texttt{/boot/grub/grub.cfg}  
        \item \texttt{/etc/default/grub}  
        \item \texttt{/etc/grub.d/menu.cfg}  
    \end{enumerate}

    \item You are using a Live CD to rescue a system. You have mounted the dedicated boot partition at \texttt{/mnt/bootpartition}. Which command correctly installs GRUB 2 to the MBR on \texttt{/dev/sda}?  
    \begin{enumerate}[A)]
        \item \texttt{grub-install --root-directory=/mnt/bootpartition /dev/sda}  
        \item \texttt{update-grub -d /dev/sda /mnt/bootpartition}  
        \item \texttt{grub-mkconfig -o /dev/sda}  
        \item \texttt{grub-install --boot-directory=/mnt/bootpartition /dev/sda}  
    \end{enumerate}

    \item In the context of GRUB 2, which file is \textbf{not} recommended for direct editing?  
    \begin{enumerate}[A)]
        \item \texttt{/etc/grub.d/40\_custom}  
        \item \texttt{/boot/grub/grub.cfg}  
        \item \texttt{/boot/grub/fonts/unicode.pf2}  
        \item \texttt{/etc/default/grub}  
    \end{enumerate}

    \item Which of the following describes a scenario where a separate \texttt{/boot} partition is particularly useful?  
    \begin{enumerate}[A)]
        \item When you want to mount \texttt{/boot} as a swap partition.  
        \item When you plan to store the entire root filesystem on MBR with no separate partitions.  
        \item When you want all boot files in \texttt{/bin/boot} for convenience.  
        \item When you use encryption or compression methods not supported by the bootloader.  
    \end{enumerate}

    \item Which GRUB 2 command can help you identify partitions and disks from the GRUB shell?  
    \begin{enumerate}[A)]
        \item \texttt{ls}  
        \item \texttt{disklist}  
        \item \texttt{list-devices}  
        \item \texttt{find /boot/grub/stage1}  
    \end{enumerate}

    \item What is the typical name of the Linux kernel file found in \texttt{/boot}?  
    \begin{enumerate}[A)]
        \item \texttt{System.map-VERSION}  
        \item \texttt{initrd.img-VERSION}  
        \item \texttt{vmlinuz-VERSION}  
        \item \texttt{config-VERSION}  
    \end{enumerate}

    \item Which of the following lines in a GRUB 2 \texttt{menuentry} block is responsible for loading the initial RAM disk?  
    \begin{enumerate}[A)]
        \item \texttt{chainloader +1}  
        \item \texttt{initrd /initrd.img}  
        \item \texttt{set root=(hd0,1)}  
        \item \texttt{linux /vmlinuz root=/dev/sda1 ro}  
    \end{enumerate}

    \item When using GRUB Legacy from a bootable rescue image, which command installs GRUB to the MBR \textbf{after} setting the correct root device?  
    \begin{enumerate}[A)]
        \item \texttt{update-grub (hd0)}  
        \item \texttt{mkconfig -o (hd0)}  
        \item \texttt{install (hd0,msdos1)}  
        \item \texttt{setup (hd0)}  
    \end{enumerate}

    \item Which line in \texttt{/etc/default/grub} disables the boot menu countdown and waits indefinitely for user input?  
    \begin{enumerate}[A)]
        \item \texttt{GRUB\_DEFAULT=saved}  
        \item \texttt{GRUB\_ENABLE\_CRYPTODISK=y}  
        \item \texttt{GRUB\_TIMEOUT=-1}  
        \item \texttt{GRUB\_CMDLINE\_LINUX\_DEFAULT=""}  
    \end{enumerate}

    \item Which component of the Linux filesystem organizes kernel symbols (variables, functions) with their memory addresses for debugging?  
    \begin{enumerate}[A)]
        \item \texttt{vmlinuz-VERSION}  
        \item \texttt{config-VERSION}  
        \item \texttt{initrd-VERSION}  
        \item \texttt{System.map-VERSION}  
    \end{enumerate}

    \item Which GRUB 2 command would you use within the shell to specify the disk or partition containing the root filesystem?  
    \begin{enumerate}[A)]
        \item \texttt{chainloader +1}  
        \item \texttt{set root=(hd0,1)}  
        \item \texttt{root (hd0,0)}  
        \item \texttt{initrd /initrd.img}  
    \end{enumerate}

    \item If a misconfiguration causes GRUB 2 to drop into a rescue shell (\texttt{grub rescue>}), which modules often need to be loaded manually to regain the normal GRUB shell functionality?  
    \begin{enumerate}[A)]
        \item \texttt{normal} and \texttt{linux}  
        \item \texttt{fs\_uuid} and \texttt{configfile}  
        \item \texttt{chainloader} and \texttt{search}  
        \item \texttt{ls} and \texttt{initrd}  
    \end{enumerate}

\end{enumerate}


%-------------------------------------------------------
% Fill-in-the-Blank Questions (102.2)
%-------------------------------------------------------
\newpage
\section*{Fill-in-the-Blank Questions for 102.2}

\begin{enumerate}[1.]

\item When a computer is powered on, the first software component to run is the \underline{\hspace{2cm}}.

\item On a BIOS system using MBR partitioning, the very first 512-byte sector of the disk is called the \underline{\hspace{2cm}}.

\item The Linux kernel is typically stored in a compressed file named \underline{\hspace{2cm}}-VERSION in the `/boot` directory.

\item The minimal root filesystem required to load the real root filesystem is contained in the \underline{\hspace{2cm}} file.

\item The file \underline{\hspace{2cm}}-VERSION contains a look-up table used for debugging kernel panics.

\item In GRUB 2, manual edits to the bootloader settings should be done in \underline{\hspace{2cm}}, rather than directly editing `/boot/grub/grub.cfg`.

\item The \underline{\hspace{2cm}} command scans available disks and partitions to build a list of operating systems for GRUB 2.

\item A dedicated \underline{\hspace{2cm}} partition is often used to store all files needed for the boot process separately from the root filesystem.

\item GRUB Legacy stores its main configuration in the file \underline{\hspace{2cm}}.

\item A technique known as \underline{\hspace{2cm}} is used by GRUB to load other operating systems’ bootloaders (e.g., Windows) from their first sector.

\end{enumerate}

%-------------------------------------------------------
% 102.3 Manage shared libraries
%-------------------------------------------------------


\newpage

\section*{102.3 Manage Shared Libraries}
\addcontentsline{toc}{section}{102.3 Manage Shared Libraries}

\textbf{Reference to LPI Objectives:}
\begin{itemize}
    \item \textbf{LPIC-1 v5, Exam 101, Objective 102.3}
    \item \textbf{Weight:} 1
\end{itemize}

\subsection*{Key Knowledge Areas}
\begin{itemize}
    \item Identifying shared libraries.
    \item Understanding typical locations of system libraries.
    \item Loading and configuring shared libraries at runtime.
\end{itemize}

\subsection*{Important Commands and Files}
\begin{itemize}
    \item \texttt{ldd} – shows shared library dependencies.
    \item \texttt{ldconfig} – updates library cache and symbolic links.
    \item \texttt{/etc/ld.so.conf} and \texttt{/etc/ld.so.conf.d/} – configuration for library paths.
    \item \texttt{LD\_LIBRARY\_PATH} – environment variable to temporarily add library paths.
\end{itemize}

\section*{Lesson Overview}

Shared libraries (\texttt{.so} files) allow multiple executables to reuse common code, reducing memory usage and disk size. Administrators must know how to locate libraries, configure library paths, and troubleshoot missing dependencies.

\section*{1. Concept of Shared Libraries}

\begin{enumerate}
    \item \textbf{Static Libraries (\texttt{.a})}
    \begin{itemize}
        \item Code is \textbf{copied} into an executable at compile/link time.
        \item Larger file size; no external dependencies at runtime.
    \end{itemize}

    \item \textbf{Dynamic (Shared) Libraries (\texttt{.so})}
    \begin{itemize}
        \item Code is \textbf{not} copied into the executable.
        \item Must be present at runtime.
        \item More efficient memory usage (shared among processes).
    \end{itemize}
\end{enumerate}

\section*{2. Typical Library Naming and Locations}

\begin{enumerate}
    \item \textbf{Shared Library Naming}
    \begin{itemize}
        \item Usually \texttt{libXYZ.so.major.minor}.
        \item Example: \texttt{libc.so.6 → libc-2.24.so}.
        \item Symbolic links often point from a generic name to a versioned file.
    \end{itemize}

    \item \textbf{Locations}
    \begin{itemize}
        \item \texttt{/lib}, \texttt{/lib64}, \texttt{/usr/lib}, \texttt{/usr/local/lib}, and architecture-specific directories like \texttt{/lib/x86\_64-linux-gnu}.
    \end{itemize}

    \item \textbf{Dynamic Linker}
    \begin{itemize}
        \item \texttt{ld.so} or \texttt{ld-linux.so} handles runtime loading of \texttt{.so} files.
    \end{itemize}
\end{enumerate}

\section*{3. Configuring Library Paths}

\begin{enumerate}
    \item \textbf{\texttt{/etc/ld.so.conf} and \texttt{/etc/ld.so.conf.d/*.conf}}
    \begin{itemize}
        \item Lists directories to be searched by the dynamic linker.
        \item Usually references sub-files in \texttt{/etc/ld.so.conf.d/}.
    \end{itemize}

    \item \textbf{\texttt{ldconfig}}
    \begin{itemize}
        \item Reads config files, creates symbolic links, updates \texttt{/etc/ld.so.cache}.
        \item Run after installing new libraries or editing config.
        \item Common options:
        \begin{itemize}
            \item \textbf{-v}: verbose mode.
            \item \textbf{-p}: print current cache contents.
        \end{itemize}
    \end{itemize}

    \item \textbf{\texttt{LD\_LIBRARY\_PATH}}
    \begin{itemize}
        \item Environment variable to \textbf{temporarily} add library directories.
        \item Example:
        \begin{lstlisting}[language=bash]
export LD_LIBRARY_PATH=/usr/local/mylib
        \end{lstlisting}
        \item Similar to \texttt{PATH}, but for shared libraries.
    \end{itemize}
\end{enumerate}

\section*{4. Checking Dependencies with \texttt{ldd}}

\begin{enumerate}
    \item \textbf{\texttt{ldd /path/to/executable}}
    \begin{itemize}
        \item Shows which \texttt{.so} files an executable needs and where they’re loaded from.
        \item Example:
        \begin{lstlisting}[language=bash]
ldd /usr/bin/git
        \end{lstlisting}
    \end{itemize}

    \item \textbf{\texttt{ldd /path/to/library.so}}
    \begin{itemize}
        \item Also works on \texttt{.so} files themselves.
    \end{itemize}

    \item \textbf{-u (unused)}
    \begin{itemize}
        \item Shows libraries listed as dependencies but not actually used.
    \end{itemize}
\end{enumerate}

\section*{Workbook Exercises}

\begin{enumerate}
    \item \textbf{List All Shared Libraries}
    \begin{itemize}
        \item Inspect \texttt{/lib}, \texttt{/usr/lib}, and \texttt{/usr/local/lib}.
        \item Observe versioned vs. unversioned symbolic links (e.g., \texttt{libm.so.6 → libm-2.31.so}).
    \end{itemize}

    \item \textbf{Update Library Cache}
    \begin{itemize}
        \item Create a directory \texttt{/opt/customlib} and put a dummy \texttt{.so} file (or symbolic link) there.
        \item Add \texttt{/opt/customlib} to \texttt{/etc/ld.so.conf.d/custom.conf}.
        \item Run \texttt{sudo ldconfig -v} and verify the new library is recognized (\texttt{ldconfig -p | grep customlib}).
    \end{itemize}

    \item \textbf{Use \texttt{LD\_LIBRARY\_PATH}}
    \begin{itemize}
        \item Temporarily set \texttt{LD\_LIBRARY\_PATH=/opt/customlib}.
        \item Run an executable depending on the custom library.
        \item Confirm it finds the library without editing \texttt{/etc/ld.so.conf}.
    \end{itemize}

    \item \textbf{Check Dependencies}
    \begin{itemize}
        \item Use \texttt{ldd /bin/ls} to see the libraries it requires.
        \item Use \texttt{ldd} on a custom binary if available.
        \item (Optional) Try the \texttt{-u} option to see if any direct dependencies are unused.
    \end{itemize}

    \item \textbf{Investigate a Broken App}
    \begin{itemize}
        \item Intentionally remove or rename a \texttt{.so} file that an application needs (e.g., \texttt{mv libXYZ.so.1 libXYZ.so.1.bak}).
        \item Attempt to run the application and note the error.
        \item Restore the file or fix the library path to resolve the error.
    \end{itemize}
\end{enumerate}

\section*{Summary}

\begin{itemize}
    \item Linux uses \textbf{shared libraries} (\texttt{.so}) to avoid embedding common code in each executable, saving resources.
    \item The \textbf{dynamic linker} finds libraries via paths defined in \texttt{/etc/ld.so.conf} (and sub-files in \texttt{ld.so.conf.d}) and updates a cache with \texttt{ldconfig}.
    \item \texttt{LD\_LIBRARY\_PATH} can override these directories temporarily for testing or specialized setups.
    \item Tools like \texttt{ldd} help identify which libraries an executable (or another library) needs, aiding in troubleshooting.
\end{itemize}



%-------------------------------------------------------
% Multiple-Choice Questions for (102.3)
%-------------------------------------------------------
\newpage
\section*{Multiple-Choice Questions for 102.3}
\begin{enumerate}[1.]

    \item What is the purpose of having a \textbf{dynamic} library as opposed to a \textbf{static} one?  
    \begin{enumerate}[A)]
        \item It embeds all external library code into the final executable.  
        \item It decreases the chance of version conflicts between libraries.  
        \item It simplifies the debugging process by combining all symbols in one file.  
        \item It allows multiple programs to share the same library file in memory at runtime.  
    \end{enumerate}

    \item Which directory is \textbf{commonly} used by Linux systems to store 64-bit libraries?  
    \begin{enumerate}[A)]
        \item \texttt{/etc/x86\_64-linux-gnu}  
        \item \texttt{/lib64}  
        \item \texttt{/bin64}  
        \item \texttt{/usr/local/opt/lib64}  
    \end{enumerate}

    \item When you install a new library in a custom location, which \textbf{environment variable} can you set to let the system know about the additional library path?  
    \begin{enumerate}[A)]
        \item \texttt{LIB\_EXTRA\_PATH}  
        \item \texttt{LD\_DEBUG\_PATH}  
        \item \texttt{LD\_LIBRARY\_PATH}  
        \item \texttt{PATHLIB}  
    \end{enumerate}

    \item After adding a new \texttt{.conf} file under \texttt{/etc/ld.so.conf.d/}, which \textbf{command} should you usually run to ensure the changes take effect?  
    \begin{enumerate}[A)]
        \item \texttt{sudo ldconfig}  
        \item \texttt{echo \$LD\_LIBRARY\_PATH}  
        \item \texttt{ldd -u /etc/ld.so.conf}  
        \item \texttt{touch /etc/ld.so.cache}  
    \end{enumerate}

    \item A \textbf{static library} is characterized by:  
    \begin{enumerate}[A)]
        \item Having no effect on program size.  
        \item Being dynamically loaded at runtime.  
        \item Residing exclusively under \texttt{/usr/local/lib}.  
        \item Being fully integrated into the binary during link time.  
    \end{enumerate}

    \item Which file typically holds \textbf{symbolic links} to the actual versioned shared library files and speeds up library lookups?  
    \begin{enumerate}[A)]
        \item \texttt{/etc/bash.bashrc}  
        \item \texttt{/etc/ld.so.conf.d/}  
        \item \texttt{/etc/ld.so.cache}  
        \item \texttt{/etc/profile}  
    \end{enumerate}

    \item What is the \textbf{main} function of the command \texttt{ldd /usr/bin/git}?  
    \begin{enumerate}[A)]
        \item It loads any missing libraries into memory for the executable.  
        \item It displays which shared libraries and memory addresses the program will use.  
        \item It modifies \texttt{/etc/ld.so.conf} to remove outdated references.  
        \item It compiles the binary from source code.  
    \end{enumerate}

    \item Which statement is \textbf{true} regarding the file \texttt{/etc/ld.so.conf} on many modern Linux distributions?  
    \begin{enumerate}[A)]
        \item It directly lists all directories containing library files without any inclusion mechanisms.  
        \item It typically has an \texttt{include} line that references additional \texttt{.conf} files in \texttt{/etc/ld.so.conf.d/}.  
        \item It is unrelated to dynamic library configuration.  
        \item It must only contain symbolic links, not absolute paths.  
    \end{enumerate}

    \item Which command option would show \textbf{unused} direct library dependencies for an executable?  
    \begin{enumerate}[A)]
        \item \texttt{ldconfig -v}  
        \item \texttt{ldconfig -p}  
        \item \texttt{ldd --verbose}  
        \item \texttt{ldd -u}  
    \end{enumerate}

    \item The \textbf{logical name} given to a shared library (like \texttt{libm.so.6}) is called the:  
    \begin{enumerate}[A)]
        \item Full path reference.  
        \item Statically linked file name.  
        \item Base library handle.  
        \item Soname.  
    \end{enumerate}

    \item Which directory is \textbf{not} typically part of the default library search path?  
    \begin{enumerate}[A)]
        \item \texttt{/opt/libraries/}  
        \item \texttt{/lib/}  
        \item \texttt{/usr/lib/}  
        \item \texttt{/usr/local/lib/}  
    \end{enumerate}

    \item In a typical modern Linux system, if you run \texttt{ldconfig -p}, what does the \texttt{-p} option do?  
    \begin{enumerate}[A)]
        \item It permanently deletes old cache entries from \texttt{/etc/ld.so.cache}.  
        \item It lists the directories and candidate libraries in the \textbf{current} cache.  
        \item It prints out any undefined symbols in the loaded libraries.  
        \item It prompts the user for additional library paths.  
    \end{enumerate}

    \item During the build process, an executable may mark certain libraries as \textbf{NEEDED} even if they aren’t used at runtime. This often happens because of:  
    \begin{enumerate}[A)]
        \item Accidental corruption in the library file.  
        \item Missing environment variables in the developer’s system.  
        \item Linker flags that reference multiple libraries.  
        \item Repeated calls to \texttt{ldconfig}.  
    \end{enumerate}

    \item Which of the following \textbf{best} describes what \texttt{ld-linux.so} (or \texttt{ld.so}) does?  
    \begin{enumerate}[A)]
        \item It locates, loads, and links the needed shared libraries at runtime.  
        \item It compiles source code into object files for linking.  
        \item It performs static linking of libraries during application build time.  
        \item It generates symbolic links in \texttt{/etc/ld.so.conf.d/} automatically.  
    \end{enumerate}

    \item What is a \textbf{key advantage} of using shared libraries on a system with many processes?  
    \begin{enumerate}[A)]
        \item Each application runs in its own memory space without any shared code.  
        \item You never need to update libraries since they are compiled into each executable.  
        \item They can be loaded in user space without root privileges.  
        \item Only one copy of the library is loaded into memory and used by multiple processes.  
    \end{enumerate}

    \item If a program is \textbf{statically} linked against a library, then:  
    \begin{enumerate}[A)]
        \item You do \textbf{not} need the library on the system at runtime for the program to function.  
        \item You must always place a copy of the library in \texttt{/lib/x86\_64-linux-gnu}.  
        \item The linker will load the library into memory each time it’s called.  
        \item There must be an exact match between library version and kernel version.  
    \end{enumerate}

    \item What does the command \texttt{ldconfig -v} do \textbf{in addition} to updating the library cache?  
    \begin{enumerate}[A)]
        \item It permanently locks the library version in place.  
        \item It displays verbose details about the libraries found, links created, and directories scanned.  
        \item It only updates symbolic links, but not the cache.  
        \item It filters out symbolic links that are unused or broken.  
    \end{enumerate}

    \item Which of these files would you \textbf{most likely} edit or create to specify a custom library directory (e.g., \texttt{/opt/mylib})?  
    \begin{enumerate}[A)]
        \item \texttt{/etc/ld.so.conf.d/custom.conf}  
        \item \texttt{/usr/local/lib/custom.ld.so.conf}  
        \item \texttt{/var/run/ldconfig/ld.so.conf}  
        \item \texttt{/etc/bash.bashrc}  
    \end{enumerate}

    \item \texttt{LD\_LIBRARY\_PATH} is similar to \texttt{PATH} in the sense that:  
    \begin{enumerate}[A)]
        \item Both are used solely for system administrators to track dependencies.  
        \item Both contain hashed references to libraries that get pre-loaded.  
        \item Both are environment variables that list directories to search, but for different purposes (executables vs. libraries).  
        \item Neither can be exported in a user shell.  
    \end{enumerate}

    \item Which statement accurately \textbf{describes} the role of symbolic links like \texttt{libpthread.so.0 -> libpthread-2.31.so}?  
    \begin{enumerate}[A)]
        \item They are stored in \texttt{/etc/ld.so.cache} for quick loading of kernel modules.  
        \item They connect older kernel releases to new libraries.  
        \item They enable the system to reference a library by its \textbf{soname} while pointing to the actual versioned file.  
        \item They are used exclusively for statically linking code.  
    \end{enumerate}

\end{enumerate}



%-------------------------------------------------------
% Fill-in-the-Blank Questions (102.3)
%-------------------------------------------------------
\newpage
\section*{Fill-in-the-Blank Questions for 102.3}

\begin{enumerate}[1.]

\item The utility used to check the \textbf{shared libraries} required by a specific program is \underline{\hspace{2cm}}.

\item The \textbf{environment variable} that can be set to add or override library paths at runtime is \underline{\hspace{2cm}}.

\item On many Linux systems, the file `/etc/ld.so.conf` includes a line pointing to configuration files in the \underline{\hspace{2cm}} directory.

\item When building an application, if the library code is \textbf{copied} into the executable at link time, we say the program is using \underline{\hspace{2cm}} linking.

\item If we install a new library in `/usr/local/mylib` and do \textbf{not} want to modify system-wide configurations, we can temporarily set \underline{\hspace{2cm}}=/usr/local/mylib.

\item A library name like `libfuse.so.2` is often a symbolic link pointing to a \textbf{versioned} file such as `libfuse.so.2.9.7`; this more general filename is often called the \underline{\hspace{2cm}}.

\item The command \underline{\hspace{2cm}} -p lists the directories and candidate libraries stored in the current library cache.

\item We typically run \underline{\hspace{2cm}} after modifying or adding a new `.conf` file under `/etc/ld.so.conf.d/` to update the library cache.

\item The file `/etc/ld.so.cache` is used to \textbf{speed up} the lookup of \underline{\hspace{2cm}}.

\item A static library file typically ends in the extension \underline{\hspace{2cm}}.

\end{enumerate}
%-------------------------------------------------------
% 102.4 Use Debian package management
%-------------------------------------------------------

\newpage

\section*{102.4 Use Debian Package Management}
\addcontentsline{toc}{section}{102.4 Use Debian Package Management}

\textbf{Reference to LPI Objectives:}
\begin{itemize}
    \item \textbf{LPIC-1 v5, Exam 101, Objective 102.4}
    \item \textbf{Weight:} 3
\end{itemize}

\subsection*{Key Knowledge Areas}
\begin{itemize}
    \item Installing, upgrading, and uninstalling Debian binary packages.
    \item Finding packages containing specific files or libraries (installed or not).
    \item Obtaining package information (version, contents, dependencies, integrity, status).
    \item Awareness of \texttt{apt} and related commands.
\end{itemize}

\subsection*{Important Files, Terms, and Utilities}
\begin{itemize}
    \item \texttt{/etc/apt/sources.list} (and \texttt{/etc/apt/sources.list.d/}) – repository lists
    \item \texttt{dpkg} – the low-level Debian package tool
    \item \texttt{dpkg-reconfigure} – re-run configuration scripts for installed packages
    \item \texttt{apt-get} (or \texttt{apt}) – higher-level tool for package handling
    \item \texttt{apt-cache} (or \texttt{apt search/show}) – searching in and displaying details about packages
\end{itemize}

\section*{Lesson Overview}

In Debian-based Linux distributions (including Ubuntu and others), packages come in \texttt{.deb} format. The \texttt{dpkg} utility can install and remove \texttt{.deb} files, but does not automatically handle dependencies. For that, tools like \texttt{apt-get} (or the more modern \texttt{apt}) help resolve dependencies, perform upgrades, and search repositories.

\section*{1. Using \texttt{dpkg} (Debian Package Tool)}

\begin{enumerate}
    \item \textbf{Install a \texttt{.deb} Package}
    \begin{lstlisting}[language=bash]
sudo dpkg -i PACKAGE_FILE.deb
    \end{lstlisting}
    \begin{itemize}
        \item Installs or upgrades the package if an older version is detected.
        \item Fails if dependencies are missing.
    \end{itemize}

    \item \textbf{Remove a Package}
    \begin{lstlisting}[language=bash]
sudo dpkg -r PACKAGE_NAME
    \end{lstlisting}
    \begin{itemize}
        \item Leaves config files behind.
        \item \texttt{-P} (purge) removes config files as well.
    \end{itemize}

    \item \textbf{Listing Installed Packages}
    \begin{lstlisting}[language=bash]
dpkg --get-selections
    \end{lstlisting}
    \begin{itemize}
        \item Outputs every installed package.
    \end{itemize}

    \item \textbf{Package Contents}
    \begin{lstlisting}[language=bash]
dpkg -L PACKAGE_NAME
    \end{lstlisting}
    \begin{itemize}
        \item Lists all files installed by that package.
    \end{itemize}

    \item \textbf{Which Package Owns a File?}
    \begin{lstlisting}[language=bash]
dpkg-query -S /path/to/file
    \end{lstlisting}
    \begin{itemize}
        \item Shows the package name that installed the file.
    \end{itemize}

    \item \textbf{Inspect a \texttt{.deb} File}
    \begin{lstlisting}[language=bash]
dpkg -I PACKAGE_FILE.deb
    \end{lstlisting}
    \begin{itemize}
        \item Prints metadata (dependencies, maintainer, version, etc.).
    \end{itemize}

    \item \textbf{Reconfigure Installed Packages}
    \begin{lstlisting}[language=bash]
sudo dpkg-reconfigure PACKAGE_NAME
    \end{lstlisting}
    \begin{itemize}
        \item Reruns post-install scripts, can fix or reset configuration.
    \end{itemize}

    \begin{noteenv}
        Using \texttt{--force} overrides safety checks but risks breaking the system.
    \end{noteenv}

\end{enumerate}

\section*{2. \texttt{apt-get} or \texttt{apt} for Dependency Handling}

\begin{enumerate}
    \item \textbf{Updating Package Index}
    \begin{lstlisting}[language=bash]
sudo apt-get update
    \end{lstlisting}
    \begin{itemize}
        \item Fetches latest package info from repositories.
    \end{itemize}

    \item \textbf{Installing Packages}
    \begin{lstlisting}[language=bash]
sudo apt-get install PACKAGE_NAME
    \end{lstlisting}
    \begin{itemize}
        \item Resolves and installs dependencies automatically.
    \end{itemize}

    \item \textbf{Removing Packages}
    \begin{lstlisting}[language=bash]
sudo apt-get remove PACKAGE_NAME
    \end{lstlisting}
    \begin{itemize}
        \item Leaves config files; use \texttt{--purge} to remove them.
    \end{itemize}

    \item \textbf{Fixing Broken Dependencies}
    \begin{lstlisting}[language=bash]
sudo apt-get install -f
    \end{lstlisting}
    \begin{itemize}
        \item Attempts to fix unmet dependencies.
    \end{itemize}

    \item \textbf{Upgrading Packages}
    \begin{lstlisting}[language=bash]
sudo apt-get upgrade
    \end{lstlisting}
    \begin{itemize}
        \item Upgrades all installed packages to latest versions in the repositories.
        \item Run \texttt{apt-get update} beforehand to refresh index.
    \end{itemize}

    \item \textbf{Cleaning Cache}
    \begin{lstlisting}[language=bash]
sudo apt-get clean
    \end{lstlisting}
    \begin{itemize}
        \item Clears \texttt{.deb} files in \texttt{/var/cache/apt/archives} to free space.
    \end{itemize}
\end{enumerate}

\section*{3. Searching for Packages}

\begin{enumerate}
    \item \textbf{\texttt{apt-cache search} (or \texttt{apt search})}
    \begin{lstlisting}[language=bash]
apt-cache search KEYWORD
    \end{lstlisting}
    \begin{itemize}
        \item Lists packages whose name/description match \texttt{KEYWORD}.
    \end{itemize}

    \item \textbf{\texttt{apt-cache show} (or \texttt{apt show})}
    \begin{lstlisting}[language=bash]
apt-cache show PACKAGE_NAME
    \end{lstlisting}
    \begin{itemize}
        \item Provides detailed info (dependencies, version, maintainers, etc.).
    \end{itemize}

    \item \textbf{\texttt{apt-file}}
    \begin{itemize}
        \item May need \texttt{sudo apt-get install apt-file} first.
        \item Then \texttt{sudo apt-file update} to sync its own index.
    \end{itemize}
    \begin{itemize}
        \item \textbf{Listing contents of a package:}
        \begin{lstlisting}[language=bash]
apt-file list PACKAGE_NAME
        \end{lstlisting}
        \item \textbf{Finding which package provides a file:}
        \begin{lstlisting}[language=bash]
apt-file search FILENAME
        \end{lstlisting}
        \item Unlike \texttt{dpkg-query -S}, works for \textbf{uninstalled} packages as well.
    \end{itemize}
\end{enumerate}

\section*{4. Configuring Repositories (\texttt{sources.list})}

\begin{itemize}
    \item \texttt{/etc/apt/sources.list} or \texttt{/etc/apt/sources.list.d/*.list}
    \item Lines typically look like:
    \begin{lstlisting}
deb http://deb.debian.org/debian buster main contrib non-free
deb-src http://deb.debian.org/debian buster main contrib non-free
    \end{lstlisting}
    \item \textbf{Archive types:} \texttt{deb} (binary packages) or \texttt{deb-src} (source).
    \item \textbf{Distributions:} e.g., \texttt{buster}, \texttt{stable}, \texttt{testing}, or codenames for Ubuntu.
    \item \textbf{Components:} \texttt{main}, \texttt{contrib}, \texttt{non-free}, \texttt{universe}, \texttt{multiverse}, etc.
\end{itemize}

After editing sources, run:
\begin{lstlisting}[language=bash]
sudo apt-get update
\end{lstlisting}

\section*{Workbook Exercises}

\begin{enumerate}
    \item \textbf{Install a \texttt{.deb} File with \texttt{dpkg}}
    \begin{itemize}
        \item Download a \texttt{.deb} (e.g., from a website).
        \item Try to install:
        \begin{lstlisting}[language=bash]
sudo dpkg -i package.deb
        \end{lstlisting}
        \item If dependencies fail, note the error message. Then fix them using either \texttt{dpkg} again or \texttt{apt-get install -f}.
    \end{itemize}

    \item \textbf{Purge an Installed Package}
    \begin{itemize}
        \item Select a small package to remove:
        \begin{lstlisting}[language=bash]
sudo apt-get remove --purge PACKAGE_NAME
        \end{lstlisting}
        \item Confirm config files are removed by checking \texttt{dpkg -L PACKAGE\_NAME} (should say not installed).
    \end{itemize}

    \item \textbf{Reconfigure a Package}
    \begin{itemize}
        \item Example:
        \begin{lstlisting}[language=bash]
sudo dpkg-reconfigure tzdata
        \end{lstlisting}
        \item Verify you can reset or change the time zone.
    \end{itemize}

    \item \textbf{Search and Install with \texttt{apt}}
    \begin{itemize}
        \item Run:
        \begin{lstlisting}[language=bash]
apt-cache search KEYWORD
        \end{lstlisting}
        \item Pick a package from the results and install it with \texttt{apt-get install}.
        \item Check the installed files with:
        \begin{lstlisting}[language=bash]
dpkg -L PACKAGE_NAME
        \end{lstlisting}
    \end{itemize}

    \item \textbf{Repository Configuration}
    \begin{itemize}
        \item Inspect \texttt{/etc/apt/sources.list} and \texttt{/etc/apt/sources.list.d/}.
        \item Optionally add a new repository line (e.g., a backports line).
        \item Run \texttt{sudo apt-get update} and check if new packages are available.
    \end{itemize}

    \item \textbf{List a Package’s Contents}
    \begin{itemize}
        \item Install \texttt{apt-file} if needed:
        \begin{lstlisting}[language=bash]
sudo apt-get install apt-file
sudo apt-file update
        \end{lstlisting}
        \item List contents for a known package:
        \begin{lstlisting}[language=bash]
apt-file list PACKAGE_NAME
        \end{lstlisting}
        \item Search for a file across all packages:
        \begin{lstlisting}[language=bash]
apt-file search /bin/somefile
        \end{lstlisting}
    \end{itemize}
\end{enumerate}

\section*{Summary}

\begin{itemize}
    \item \texttt{dpkg} handles \texttt{.deb} packages at a low level but does \textbf{not} resolve dependencies automatically.
    \item \texttt{apt-get}, \texttt{apt}, and \texttt{apt-cache} provide higher-level features like dependency resolution, searching repositories, and automated upgrades.
    \item \texttt{apt-file} allows searching within packages (even those not installed).
    \item The \texttt{sources.list} (and \texttt{.list} files in \texttt{/etc/apt/sources.list.d}) specify where apt should look for packages.
    \item Knowing these tools is critical for effectively installing, upgrading, or removing software in Debian-based systems, aligning with the LPIC-1 \textbf{102.4} objective.
\end{itemize}


%-------------------------------------------------------
% Multiple-Choice Questions for (102.4)
%-------------------------------------------------------
\newpage
\section*{Multiple-Choice Questions for 102.4}

\begin{enumerate}[1.]

    \item Which parameter in \texttt{dpkg} is used to remove both a package and its configuration files?  
    \begin{enumerate}[A)]
        \item \texttt{-r}  
        \item \texttt{-I}  
        \item \texttt{-S}  
        \item \texttt{-P}  
    \end{enumerate}

    \item Which of the following commands updates the local package index using APT?  
    \begin{enumerate}[A)]
        \item \texttt{apt-get remove}  
        \item \texttt{apt-get install -f}  
        \item \texttt{apt-get update}  
        \item \texttt{apt-get purge}  
    \end{enumerate}

    \item Which \texttt{dpkg} command option allows you to list all files that a package has installed on the system?  
    \begin{enumerate}[A)]
        \item \texttt{dpkg --get-selections}  
        \item \texttt{dpkg -I}  
        \item \texttt{dpkg-reconfigure}  
        \item \texttt{dpkg -L}  
    \end{enumerate}

    \item Which command is used to remove a package but keep its configuration files?  
    \begin{enumerate}[A)]
        \item \texttt{dpkg -P}  
        \item \texttt{apt-get remove}  
        \item \texttt{apt-get install -f}  
        \item \texttt{apt-get purge}  
    \end{enumerate}

    \item What is the correct \texttt{Archive type} that indicates a repository contains ready-to-run packages?  
    \begin{enumerate}[A)]
        \item \texttt{deb-src}  
        \item \texttt{main}  
        \item \texttt{deb}  
        \item \texttt{contrib}  
    \end{enumerate}

    \item What is the default location of the local cache where \texttt{.deb} files are downloaded before installation?  
    \begin{enumerate}[A)]
        \item \texttt{/etc/apt/sources.list}  
        \item \texttt{/var/cache/apt/archives}  
        \item \texttt{/usr/local/cache/dpkg}  
        \item \texttt{/var/dpkg/archives/partial}  
    \end{enumerate}

    \item Which command helps you restore or re-run the initial configuration process of a package?  
    \begin{enumerate}[A)]
        \item \texttt{dpkg --get-selections}  
        \item \texttt{dpkg -L}  
        \item \texttt{dpkg-reconfigure}  
        \item \texttt{dpkg -S}  
    \end{enumerate}

    \item Which \texttt{dpkg} option can show you which package owns a specific file on the filesystem (e.g., \texttt{/usr/bin/example})?  
    \begin{enumerate}[A)]
        \item \texttt{dpkg -I}  
        \item \texttt{dpkg -L}  
        \item \texttt{dpkg -P}  
        \item \texttt{dpkg -S}  
    \end{enumerate}

    \item Which of the following statements is true about \texttt{apt-get install -f}?  
    \begin{enumerate}[A)]
        \item It attempts to fix broken dependencies by installing missing packages.  
        \item It removes all configuration files of broken packages.  
        \item It removes all broken packages from the system.  
        \item It upgrades all packages to the latest version.  
    \end{enumerate}

    \item Which command can be used to search for a package by a keyword in the APT package index?  
    \begin{enumerate}[A)]
        \item \texttt{apt-get show}  
        \item \texttt{apt-cache search}  
        \item \texttt{dpkg -L}  
        \item \texttt{dpkg-query -S}  
    \end{enumerate}

    \item Which parameter of \texttt{dpkg} lists the basic metadata (like version, architecture, dependencies) of a \texttt{.deb} package file?  
    \begin{enumerate}[A)]
        \item \texttt{-I}  
        \item \texttt{-r}  
        \item \texttt{-P}  
        \item \texttt{-L}  
    \end{enumerate}

    \item Which of these lines in \texttt{/etc/apt/sources.list} indicates a repository of source packages rather than binary packages?  
    \begin{enumerate}[A)]
        \item \texttt{deb-src}  
        \item \texttt{deb http://repo.example.com stable main}  
        \item \texttt{deb /var/cache/apt/archives stable main}  
        \item \texttt{deb http://repo.example.com sources main}  
    \end{enumerate}

    \item Which command is used to remove unnecessary \texttt{.deb} files in the local cache under \texttt{/var/cache/apt/archives}?  
    \begin{enumerate}[A)]
        \item \texttt{apt-get remove}  
        \item \texttt{apt-get purge}  
        \item \texttt{apt-get update}  
        \item \texttt{apt-get clean}  
    \end{enumerate}

    \item Which \texttt{dpkg} parameter performs the same function as \texttt{dpkg -r} but leaves configuration files behind?  
    \begin{enumerate}[A)]
        \item \texttt{-r}  
        \item \texttt{-P}  
        \item \texttt{-S}  
        \item \texttt{-I}  
    \end{enumerate}

    \item Which APT command will remove a package along with its configuration files?  
    \begin{enumerate}[A)]
        \item \texttt{apt-get remove}  
        \item \texttt{apt-get update}  
        \item \texttt{apt-get install -f}  
        \item \texttt{apt-get purge}  
    \end{enumerate}

    \item When a Debian-based system warns that certain packages are “kept back,” which command would you generally use to upgrade them?  
    \begin{enumerate}[A)]
        \item \texttt{apt-get dist-upgrade}  
        \item \texttt{apt-file search}  
        \item \texttt{dpkg -P}  
        \item \texttt{dpkg -i}  
    \end{enumerate}

    \item Which Debian repository component includes software that is DFSG-compliant but depends on non-free components?  
    \begin{enumerate}[A)]
        \item \texttt{main}  
        \item \texttt{contrib}  
        \item \texttt{multiverse}  
        \item \texttt{restricted}  
    \end{enumerate}

    \item Which APT utility focuses on searching for package information and displaying metadata about packages?  
    \begin{enumerate}[A)]
        \item \texttt{dpkg}  
        \item \texttt{apt-file}  
        \item \texttt{apt-cache}  
        \item \texttt{dpkg-query}  
    \end{enumerate}

    \item Which of the following \texttt{apt-get} commands will remove a package but leave the configuration files on the system?  
    \begin{enumerate}[A)]
        \item \texttt{apt-get purge}  
        \item \texttt{apt-get remove}  
        \item \texttt{apt-get install -f}  
        \item \texttt{apt-get upgrade}  
    \end{enumerate}

    \item Which main Debian repository contains packages that are compliant with the Debian Free Software Guidelines (DFSG)?  
    \begin{enumerate}[A)]
        \item \texttt{restricted}  
        \item \texttt{non-free}  
        \item \texttt{main}  
        \item \texttt{multiverse}  
    \end{enumerate}

\end{enumerate}



%-------------------------------------------------------
% Fill-in-the-Blank Questions (102.4)
%-------------------------------------------------------

\newpage
\section*{Fill-in-the-Blank Questions for 102.4}

\begin{enumerate}[1.]
\item To list all \textbf{installed packages} on a Debian-based system using dpkg, you can run:\newline
\texttt{dpkg --get-\underline{\hspace{2cm}}}.

\item The \textbf{Advanced Package Tool}, also known as APT, uses repository information from the file:\newline
\texttt{/etc/apt/\underline{\hspace{2cm}}.list}.

\item If you have missing dependencies after a failed install, you can attempt to fix them with:\newline
\texttt{apt-get install \underline{\hspace{2cm}}}.

\item You can use `dpkg` with the \textbf{-I} parameter to get \underline{\hspace{2cm}} about a .deb package file.

\item The package list that APT uses is also known as the APT \underline{\hspace{2cm}}.

\item The parameter `dpkg -L` lets you list the \underline{\hspace{2cm}} installed by a particular package.

\item Lines beginning with a \underline{\hspace{2cm}} character in `/etc/apt/sources.list` are ignored because they are comments.

\item The command:\newline
\texttt{apt-\underline{\hspace{2cm}} search p7zip}\newline
is used to search for a package containing the term “p7zip.”

\item To remove all downloaded package files and reclaim disk space, you run:\newline
\texttt{apt-get \underline{\hspace{2cm}}}.

\item The configuration files are only completely removed when you use the dpkg parameter `-P`, which stands for:\newline
\texttt{dpkg -P \underline{\hspace{2cm}}}.

\end{enumerate}

%-------------------------------------------------------
% 102.5 Use RPM and YUM package management
%-------------------------------------------------------
\newpage

\section*{102.5 Use RPM and YUM Package Management}
\addcontentsline{toc}{section}{102.5 Use RPM and YUM Package Management}

\textbf{Reference to LPI Objectives:}
\begin{itemize}
    \item \textbf{LPIC-1 v5, Exam 101, Objective 102.5}
    \item \textbf{Weight:} 3
\end{itemize}

\subsection*{Key Knowledge Areas}
\begin{itemize}
    \item Installing, re-installing, upgrading, and removing packages with \textbf{rpm}, \textbf{YUM}, and \textbf{Zypper}
    \item Obtaining information on RPM packages (version, dependencies, signatures, etc.)
    \item Determining the files a package provides, and finding which package a specific file comes from
    \item Awareness of \textbf{dnf} (successor to YUM in Fedora-based systems)
\end{itemize}

\subsection*{Important Files, Terms, and Utilities}
\begin{itemize}
    \item \textbf{rpm}, \textbf{rpm2cpio}
    \item \texttt{/etc/yum.conf}, \texttt{/etc/yum.repos.d/}
    \item \textbf{yum}, \textbf{zypper}, \textbf{dnf}
    \item Various \texttt{.repo} configuration files
\end{itemize}

\section*{Lesson Overview}

Linux distributions derived from Red Hat (RHEL, Fedora, CentOS, openSUSE) typically use RPM (\texttt{.rpm} files) for package distribution. The \textbf{rpm} utility handles low-level package operations but does \textbf{not} resolve dependencies automatically. Higher-level tools like \textbf{yum}, \textbf{dnf}, and \textbf{zypper} manage dependencies, perform system upgrades, and handle repository configurations.

\section*{1. Managing Packages with \texttt{rpm}}
\begin{enumerate}
    \item \textbf{Installing a Package}
\begin{lstlisting}[language=bash]
rpm -ivh PACKAGE_FILE.rpm
\end{lstlisting}
\begin{itemize}
    \item \textbf{-i:} install
    \item \textbf{-v:} verbose
    \item \textbf{-h:} show progress with hash marks
\end{itemize}

\item \textbf{Upgrading a Package}
\begin{lstlisting}[language=bash]
rpm -Uvh PACKAGE_FILE.rpm
\end{lstlisting}
\begin{itemize}
    \item Installs if not already present; upgrades if older version is detected.
    \item \textbf{-F:} freshen (upgrade only if installed; skip if not).
\end{itemize}

\item \textbf{Removing (Erasing) a Package}
\begin{lstlisting}[language=bash]
rpm -e PACKAGE_NAME
\end{lstlisting}
\begin{itemize}
    \item Fails if other packages depend on it.
    \item Remove those dependents first or specify them all at once.
\end{itemize}

\item \textbf{Querying Installed Packages}
\begin{itemize}
    \item \textbf{List all packages:}
    \begin{lstlisting}[language=bash]
rpm -qa
    \end{lstlisting}
    \item \textbf{Query a package’s info:}
    \begin{lstlisting}[language=bash]
rpm -qi PACKAGE_NAME
    \end{lstlisting}
    \item \textbf{List files in a package:}
    \begin{lstlisting}[language=bash]
rpm -ql PACKAGE_NAME
    \end{lstlisting}
    \item \textbf{Find which package owns a file:}
    \begin{lstlisting}[language=bash]
rpm -qf /path/to/file
    \end{lstlisting}
\end{itemize}

\item \textbf{Inspecting an Uninstalled Package}
\begin{itemize}
    \item \textbf{Metadata (info):}
    \begin{lstlisting}[language=bash]
rpm -qip PACKAGE_FILE.rpm
    \end{lstlisting}
    \item \textbf{Contents (file list):}
    \begin{lstlisting}[language=bash]
rpm -qlp PACKAGE_FILE.rpm
    \end{lstlisting}
\end{itemize}

\item \textbf{Dependencies}
\begin{itemize}
    \item \textbf{rpm} will list missing dependencies but cannot automatically resolve them.
    \item Use \textbf{yum}, \textbf{dnf}, or \textbf{zypper} to handle dependencies more effectively.
\end{itemize}
\end{enumerate}

\section*{2. YUM (YellowDog Updater Modified)}

\begin{enumerate}
    \item \textbf{Searching for Packages}
    \begin{lstlisting}[language=bash]
yum search KEYWORD
    \end{lstlisting}
    \begin{itemize}
        \item Searches names and summaries for \texttt{KEYWORD}.
    \end{itemize}

    \item \textbf{Installing a Package}
    \begin{lstlisting}[language=bash]
yum install PACKAGE_NAME
    \end{lstlisting}
    \begin{itemize}
        \item Resolves and installs dependencies automatically.
    \end{itemize}

    \item \textbf{Removing a Package}
    \begin{lstlisting}[language=bash]
yum remove PACKAGE_NAME
    \end{lstlisting}
    \begin{itemize}
        \item Also removes packages that depend on it.
    \end{itemize}

    \item \textbf{Upgrading Packages}
    \begin{lstlisting}[language=bash]
yum update PACKAGE_NAME
    \end{lstlisting}
    \begin{itemize}
        \item Without a package name, updates the entire system.
    \end{itemize}

    \item \textbf{Checking for Updates}
    \begin{lstlisting}[language=bash]
yum check-update [PACKAGE_NAME]
    \end{lstlisting}
    \begin{itemize}
        \item Lists available updates; omit package name to check all installed packages.
    \end{itemize}

    \item \textbf{Which Package Provides a File}
    \begin{lstlisting}[language=bash]
yum whatprovides FILENAME
    \end{lstlisting}
    \begin{itemize}
        \item Helps identify the package that contains a needed file or library.
    \end{itemize}

    \item \textbf{Getting Package Info}
    \begin{lstlisting}[language=bash]
yum info PACKAGE_NAME
    \end{lstlisting}
    \begin{itemize}
        \item Shows version, architecture, summary, repo source, etc.
    \end{itemize}

    \item \textbf{Repositories (\texttt{/etc/yum.repos.d/*.repo})}
    \begin{itemize}
        \item \textbf{Add/Remove Repos:} \texttt{yum-config-manager --add-repo URL} / \texttt{yum-config-manager --remove-repo REPO\_ID}
        \item \textbf{Enable/Disable Repos:} \texttt{yum-config-manager --enable REPO\_ID} / \texttt{yum-config-manager --disable REPO\_ID}
        \item \textbf{List Repos:} \texttt{yum repolist all}
    \end{itemize}

    \item \textbf{Cleaning Cache}
    \begin{lstlisting}[language=bash]
yum clean [packages|metadata|all]
    \end{lstlisting}
    \begin{itemize}
        \item Frees disk space by removing cached \texttt{.rpm} files or metadata.
    \end{itemize}
\end{enumerate}

\section*{3. DNF (Dandified YUM)}

\begin{enumerate}
    \item \textbf{Overview}
    \begin{itemize}
        \item Used by Fedora and newer Red Hat-based systems.
        \item Similar commands to \textbf{yum}.
    \end{itemize}

    \item \textbf{Basic Commands}
    \begin{itemize}
        \item \textbf{Search:} \texttt{dnf search KEYWORD}
        \item \textbf{Install:} \texttt{dnf install PACKAGE\_NAME}
        \item \textbf{Remove:} \texttt{dnf remove PACKAGE\_NAME}
        \item \textbf{Upgrade:} \texttt{dnf upgrade [PACKAGE\_NAME]} (upgrade entire system if no package specified)
        \item \textbf{Which Package Provides a File:} \texttt{dnf provides /path/to/file}
        \item \textbf{List Installed Packages:} \texttt{dnf list --installed}
    \end{itemize}

    \item \textbf{Repositories}
    \begin{itemize}
        \item \textbf{List all:} \texttt{dnf repolist [--enabled|--disabled]}
        \item \textbf{Add:} \texttt{dnf config-manager --add-repo URL}
        \item \textbf{Enable/Disable:} \texttt{dnf config-manager --set-enabled REPO\_ID} / \texttt{dnf config-manager --set-disabled REPO\_ID}
    \end{itemize}

    \item \textbf{Cleaning Cache}
    \begin{lstlisting}[language=bash]
dnf clean all
    \end{lstlisting}
    \begin{itemize}
        \item Removes cache data (packages, metadata).
    \end{itemize}
\end{enumerate}

\section*{4. Zypper (openSUSE / SUSE)}

\begin{enumerate}
    \item \textbf{Refreshing Repositories}
    \begin{lstlisting}[language=bash]
zypper refresh
    \end{lstlisting}
    \begin{itemize}
        \item Updates repository metadata.
    \end{itemize}

    \item \textbf{Searching for Packages}
    \begin{lstlisting}[language=bash]
zypper search [--installed-only|--not-installed|--provides /file]
    \end{lstlisting}
    \begin{itemize}
        \item \texttt{zypper se KEYWORD}
        \item \texttt{zypper se -i KEYWORD} (installed only)
        \item \texttt{zypper se --provides /path/to/file} (find package providing a file)
    \end{itemize}

    \item \textbf{Installing Packages}
    \begin{lstlisting}[language=bash]
zypper install PACKAGE_NAME
    \end{lstlisting}
    \begin{itemize}
        \item Or \texttt{zypper in PACKAGE\_NAME}.
    \end{itemize}

    \item \textbf{Upgrading Packages}
    \begin{lstlisting}[language=bash]
zypper update [PACKAGE_NAME]
    \end{lstlisting}
    \begin{itemize}
        \item Without specifying a package, updates all.
    \end{itemize}

    \item \textbf{Removing Packages}
    \begin{lstlisting}[language=bash]
zypper remove PACKAGE_NAME
    \end{lstlisting}
    \begin{itemize}
        \item Or \texttt{zypper rm PACKAGE\_NAME}.
    \end{itemize}

    \item \textbf{Package Info}
    \begin{lstlisting}[language=bash]
zypper info PACKAGE_NAME
    \end{lstlisting}
    \begin{itemize}
        \item Shows version, repository, summary, etc.
    \end{itemize}

    \item \textbf{Listing Package Contents}
    \begin{lstlisting}[language=bash]
zypper search --provides /path/to/file
    \end{lstlisting}
    \begin{itemize}
        \item Or \texttt{zypper info --requires PACKAGE\_NAME} for dependencies.
    \end{itemize}

    \item \textbf{Repositories}
    \begin{itemize}
        \item \textbf{List:} \texttt{zypper repos}
        \item \textbf{Add:} \texttt{zypper addrepo URL ALIAS}
        \item \textbf{Remove:} \texttt{zypper removerepo ALIAS}
        \item \textbf{Enable/Disable:}
        \begin{lstlisting}[language=bash]
zypper modifyrepo -e ALIAS   # enable
zypper modifyrepo -d ALIAS   # disable
        \end{lstlisting}
        \item \textbf{Auto-Refresh:}
        \begin{lstlisting}[language=bash]
zypper modifyrepo -f ALIAS   # enable auto-refresh
zypper modifyrepo -F ALIAS   # disable auto-refresh
        \end{lstlisting}
    \end{itemize}
\end{enumerate}

\section*{Workbook Exercises}

\begin{enumerate}
    \item \textbf{Basic \texttt{rpm} Operations}
    \begin{itemize}
        \item Download an \texttt{.rpm} package (e.g., \texttt{wget http://example.com/somepackage.rpm}).
        \item Install it via:
        \begin{lstlisting}[language=bash]
sudo rpm -ivh somepackage.rpm
        \end{lstlisting}
        \item Query what files it installed (\texttt{rpm -ql PACKAGE\_NAME}).
        \item Remove it (\texttt{rpm -e PACKAGE\_NAME}).
    \end{itemize}

    \item \textbf{Resolve Dependencies with YUM}
    \begin{itemize}
        \item Try installing a package that requires another package.
        \item Notice that \texttt{yum} automatically pulls needed dependencies.
        \item Remove the newly installed package and dependencies if desired:
        \begin{lstlisting}[language=bash]
sudo yum remove PACKAGE_NAME
        \end{lstlisting}
    \end{itemize}

    \item \textbf{Which Package Owns a File?}
    \begin{itemize}
        \item Use \texttt{yum whatprovides /usr/bin/zipinfo} (or a similar file) to see who owns it.
        \item Confirm with \texttt{rpm -qf /usr/bin/zipinfo}.
    \end{itemize}

    \item \textbf{Update the Entire System}
    \begin{itemize}
        \item On a CentOS or RHEL system, run:
        \begin{lstlisting}[language=bash]
sudo yum update
        \end{lstlisting}
        \item Reboot if a new kernel is installed.
    \end{itemize}

    \item \textbf{Add/Enable a New Repository}
    \begin{itemize}
        \item For CentOS, add a repo:
        \begin{lstlisting}[language=bash]
yum-config-manager --add-repo https://example.com/custom.repo
        \end{lstlisting}
        \item Use \texttt{yum repolist all} to confirm it appears, then enable if needed.
    \end{itemize}

    \item \textbf{Zypper Install}
    \begin{itemize}
        \item On an openSUSE system, run:
        \begin{lstlisting}[language=bash]
sudo zypper refresh
sudo zypper search unzip
sudo zypper install unzip
        \end{lstlisting}
        \item Check the installed files via \texttt{rpm -ql unzip} or \texttt{zypper info unzip}.
    \end{itemize}

    \item \textbf{dnf Operations}
    \begin{itemize}
        \item On a Fedora system, search for \texttt{gimp}:
        \begin{lstlisting}[language=bash]
dnf search gimp
        \end{lstlisting}
        \item Install it:
        \begin{lstlisting}[language=bash]
dnf install gimp
        \end{lstlisting}
        \item Remove it:
        \begin{lstlisting}[language=bash]
dnf remove gimp
        \end{lstlisting}
    \end{itemize}
\end{enumerate}

\section*{Summary}

\begin{itemize}
    \item \textbf{rpm} is the low-level tool for installing \texttt{.rpm} packages, but it does \textbf{not} handle dependencies automatically.
    \item \textbf{yum}, \textbf{dnf}, and \textbf{zypper} provide higher-level package management with automatic dependency resolution, repository management, and system-wide updates.
    \item Each tool has commands for searching packages, installing, upgrading, removing, and listing package contents.
    \item Understanding these utilities is critical for effectively managing software on RPM-based Linux distributions—an important skill for LPIC-1 certification and real-world administration.
\end{itemize}



%-------------------------------------------------------
% Multiple-Choice Questions for (102.5)
%-------------------------------------------------------
\newpage
\section*{Multiple-Choice Questions for 102.5}

\begin{enumerate}[1.]

    \item Which \texttt{rpm} parameter is used to remove (erase) an installed package?  
    \begin{enumerate}[A)]
        \item \texttt{-U}  
        \item \texttt{-e}  
        \item \texttt{-F}  
        \item \texttt{-qa}  
    \end{enumerate}

    \item Which \texttt{rpm} command allows you to query an \emph{uninstalled} package file for information (name, version, etc.)?  
    \begin{enumerate}[A)]
        \item \texttt{rpm -qi}  
        \item \texttt{rpm -ql}  
        \item \texttt{rpm -qa}  
        \item \texttt{rpm -qip}  
    \end{enumerate}

    \item Which \texttt{yum} command installs a package named \texttt{vim} from the configured repositories?  
    \begin{enumerate}[A)]
        \item \texttt{yum install vim}  
        \item \texttt{yum remove vim}  
        \item \texttt{yum info vim}  
        \item \texttt{yum repolist vim}  
    \end{enumerate}

    \item Which \texttt{yum} subcommand removes an installed package from your system?  
    \begin{enumerate}[A)]
        \item \texttt{yum whatprovides}  
        \item \texttt{yum info}  
        \item \texttt{yum remove}  
        \item \texttt{yum repolist}  
    \end{enumerate}

    \item Using \texttt{yum}, which command do you run to find the package that provides \texttt{/usr/bin/unzip}?  
    \begin{enumerate}[A)]
        \item \texttt{yum search /usr/bin/unzip}  
        \item \texttt{yum repolist /usr/bin/unzip}  
        \item \texttt{yum list installed /usr/bin/unzip}  
        \item \texttt{yum whatprovides /usr/bin/unzip}  
    \end{enumerate}

    \item Which of the following \texttt{rpm} parameters lists \emph{all} installed packages on the system?  
    \begin{enumerate}[A)]
        \item \texttt{-e}  
        \item \texttt{-qa}  
        \item \texttt{-U}  
        \item \texttt{-ql}  
    \end{enumerate}

    \item What is the main purpose of the \texttt{rpm2cpio} utility?  
    \begin{enumerate}[A)]
        \item It converts an RPM file into a \texttt{.cpio} archive  
        \item It lists installed \texttt{.cpio} packages  
        \item It creates a \texttt{.tar.gz} archive from an RPM  
        \item It checks package signatures in cpio format  
    \end{enumerate}

    \item Which \texttt{rpm} command could forcibly install (ignoring dependencies) a package named \texttt{mypkg.rpm}?  
    \begin{enumerate}[A)]
        \item \texttt{rpm -Uvh --nodeps mypkg.rpm}  
        \item \texttt{rpm -e mypkg.rpm}  
        \item \texttt{rpm -ql mypkg.rpm}  
        \item \texttt{rpm -qa --nodeps mypkg.rpm}  
    \end{enumerate}

    \item Which \texttt{dnf} command updates \emph{all} installed packages on the system to their latest versions?  
    \begin{enumerate}[A)]
        \item \texttt{dnf info}  
        \item \texttt{dnf remove}  
        \item \texttt{dnf upgrade}  
        \item \texttt{dnf list --installed}  
    \end{enumerate}

    \item When using \texttt{dnf}, how do you find which package provides \texttt{/usr/bin/unzip}?  
    \begin{enumerate}[A)]
        \item \texttt{dnf list /usr/bin/unzip}  
        \item \texttt{dnf provides /usr/bin/unzip}  
        \item \texttt{dnf repoquery --installed /usr/bin/unzip}  
        \item \texttt{dnf info /usr/bin/unzip}  
    \end{enumerate}

    \item Which \texttt{zypper} command lets you install an RPM file on disk (e.g., \texttt{/home/user/newpkg.rpm}) while also resolving dependencies from repositories?  
    \begin{enumerate}[A)]
        \item \texttt{zypper update /home/user/newpkg.rpm}  
        \item \texttt{zypper refresh /home/user/newpkg.rpm}  
        \item \texttt{zypper query /home/user/newpkg.rpm}  
        \item \texttt{zypper in /home/user/newpkg.rpm}  
    \end{enumerate}

    \item Which \texttt{zypper} operator should you use to remove a package named \texttt{unzip} from your system?  
    \begin{enumerate}[A)]
        \item \texttt{zypper refresh unzip}  
        \item \texttt{zypper rm unzip}  
        \item \texttt{zypper se -i unzip}  
        \item \texttt{zypper up unzip}  
    \end{enumerate}

    \item Which \texttt{zypper} command syntax is used to see which packages provide a specific file, e.g., \texttt{/usr/lib64/libgimpui-2.0.so.0}?  
    \begin{enumerate}[A)]
        \item \texttt{zypper se --provides /usr/lib64/libgimpui-2.0.so.0}  
        \item \texttt{zypper addrepo --provides /usr/lib64/libgimpui-2.0.so.0}  
        \item \texttt{zypper info --provides /usr/lib64/libgimpui-2.0.so.0}  
        \item \texttt{zypper up --provides /usr/lib64/libgimpui-2.0.so.0}  
    \end{enumerate}

    \item Which \texttt{zypper} operator refreshes all enabled repositories to get the latest metadata?  
    \begin{enumerate}[A)]
        \item \texttt{zypper se}  
        \item \texttt{zypper info}  
        \item \texttt{zypper rm}  
        \item \texttt{zypper refresh}  
    \end{enumerate}

    \item If you only want to \emph{list} available updates (without installing them) using \texttt{zypper}, which command would you use?  
    \begin{enumerate}[A)]
        \item \texttt{zypper up --list}  
        \item \texttt{zypper se updates}  
        \item \texttt{zypper list-updates}  
        \item \texttt{zypper in --updates-only}  
    \end{enumerate}

    \item How do you disable an existing repository named \texttt{repo-non-oss} using \texttt{zypper}?  
    \begin{enumerate}[A)]
        \item \texttt{zypper addrepo -d repo-non-oss}  
        \item \texttt{zypper rm repo-non-oss}  
        \item \texttt{zypper se -d repo-non-oss}  
        \item \texttt{zypper modifyrepo -d repo-non-oss}  
    \end{enumerate}

    \item What does the \texttt{yum-config-manager --add-repo <URL>} command do?  
    \begin{enumerate}[A)]
        \item It removes a repository from \texttt{/etc/yum.conf}  
        \item It adds a new \texttt{.repo} file in \texttt{/etc/yum.repos.d/} based on the specified URL  
        \item It automatically upgrades \texttt{yum} to the latest version  
        \item It disables all repositories except the one specified  
    \end{enumerate}

    \item Which \texttt{dnf} command removes an installed package from your system?  
    \begin{enumerate}[A)]
        \item \texttt{dnf remove PACKAGENAME}  
        \item \texttt{dnf fetch PACKAGENAME}  
        \item \texttt{dnf localinstall PACKAGENAME}  
        \item \texttt{dnf whatprovides PACKAGENAME}  
    \end{enumerate}

    \item Which \texttt{yum} command checks if a new version of a package (e.g., \texttt{wget}) is available, \emph{without} installing it?  
    \begin{enumerate}[A)]
        \item \texttt{yum whatprovides wget}  
        \item \texttt{yum info wget}  
        \item \texttt{yum check-update wget}  
        \item \texttt{yum clean metadata wget}  
    \end{enumerate}

    \item Which file stores the primary configuration for \texttt{yum} by default?  
    \begin{enumerate}[A)]
        \item \texttt{/etc/rpm.conf}  
        \item \texttt{/var/log/yum.conf}  
        \item \texttt{/etc/yum.conf}  
        \item \texttt{/etc/dnf.conf}  
    \end{enumerate}

\end{enumerate}


%-------------------------------------------------------
% Fill-in-the-Blank Questions (102.5)
%-------------------------------------------------------

\newpage
\section*{Fill-in-the-Blank Questions for 102.5}

\begin{enumerate}[1.]

\item To remove a package using `rpm`, we use:\newline
\texttt{rpm \underline{\hspace{2cm}} PACKAGENAME}.

\item On Debian-based systems, the tool analogous to `yum` (mentioned in the lesson) is:\newline
\underline{\hspace{2cm}}.

\item To search for a package with `zypper`, you can use either:\newline
\texttt{zypper \underline{\hspace{2cm}} or zypper \underline{\hspace{2cm}}}.

\item When using `dnf`, the command to uninstall a package named `curl` is:\newline
\texttt{dnf \underline{\hspace{2cm}} curl}.

\item The main configuration file for yum is located at:
\underline{\hspace{2cm}}.

\item On RPM-based systems, the command `rpm -qa` means “query \underline{\hspace{2cm}}.”

\item If you want to list all available updates using `yum` without installing them, you can run:\newline
\texttt{yum \underline{\hspace{2cm}}}.

\item To view the metadata of the `gimp` package using `zypper`, type:\newline
\texttt{zypper \underline{\hspace{2cm}} gimp}.

\item The tool that is considered a “fork” or newer version of YUM (primarily used in Fedora) is called:
\underline{\hspace{2cm}}.

\item To list the files installed by a package named `wget` using `rpm`, you would use:\newline
\texttt{rpm -\underline{\hspace{2cm}} wget}.

\end{enumerate}
%-------------------------------------------------------
% 102.6 Linux as a virtualization guest
%-------------------------------------------------------
\newpage

\section*{102.6 Linux as a virtualization guest}
\addcontentsline{toc}{section}{102.6 Linux as a virtualization guest}



\textbf{Reference to LPI Objectives:}
\begin{itemize}
    \item \textbf{LPIC-1 v5, Exam 101, Objective 102.6}
    \item \textbf{Weight:} 1
\end{itemize}

\subsection*{Key Knowledge Areas}
\begin{itemize}
    \item General concept of virtual machines (VMs) and containers
    \item Key elements of Infrastructure as a Service (IaaS), such as compute instances, block storage, networking
    \item Changing Linux-specific system properties when cloning or templating a VM (e.g., host keys, D-Bus machine ID)
    \item Using system images to deploy VMs, cloud instances, and containers
    \item Guest drivers and integration features for Linux VMs
    \item Awareness of \textbf{cloud-init} for automated provisioning
\end{itemize}

\subsection*{Important Files, Terms, and Utilities}
\begin{itemize}
    \item \textbf{Virtual machine}, \textbf{Linux container}, \textbf{application container}
    \item \textbf{Guest drivers} (e.g., Virtio, VirtualBox Guest Additions)
    \item \textbf{SSH host keys}, \textbf{D-Bus machine ID}
    \item \textbf{cloud-init}
\end{itemize}

\section*{1. Virtualization Overview}

\begin{enumerate}
    \item \textbf{Hypervisor}
    \begin{itemize}
        \item Software layer allowing multiple \textbf{guest} operating systems to run on a single host.
        \item Manages physical resources (CPU, memory, storage).
    \end{itemize}

    \item \textbf{Types of Hypervisors}
    \begin{itemize}
        \item \textbf{Type-1 (Bare-metal):} Runs directly on hardware (e.g., \textbf{Xen}, some KVM implementations).
        \item \textbf{Type-2 (Hosted):} Runs on top of a host OS (e.g., \textbf{VirtualBox}).
    \end{itemize}

    \item \textbf{Common Hypervisors}
    \begin{itemize}
        \item \textbf{Xen} (Type-1, open source).
        \item \textbf{KVM} (kernel module in Linux; used with \textbf{libvirt}, \textbf{QEMU}).
        \item \textbf{VirtualBox} (cross-platform, Type-2).
    \end{itemize}

    \item \textbf{Migration}
    \begin{itemize}
        \item \textbf{Cold migration:} Move VM when powered off.
        \item \textbf{Live migration:} Move a running VM to another hypervisor. Useful for maintenance/resiliency.
    \end{itemize}
\end{enumerate}

\section*{2. Types of Virtual Machines}

\begin{enumerate}
    \item \textbf{Fully Virtualized (Hardware VM)}
    \begin{itemize}
        \item Guest OS is unmodified and unaware it’s virtualized.
        \item CPU extensions (Intel VT-x, AMD-V) often required.
    \end{itemize}

    \item \textbf{Paravirtualized (PVM)}
    \begin{itemize}
        \item Guest OS is aware it’s running in a VM.
        \item Uses special drivers for improved performance (e.g., \textbf{Virtio} in KVM, Xen drivers).
    \end{itemize}

    \item \textbf{Hybrid}
    \begin{itemize}
        \item Fully virtualized OS that uses paravirtualized drivers for I/O performance boosts (disk, network).
    \end{itemize}
\end{enumerate}

\section*{3. Guest Drivers and Tools}

\begin{itemize}
    \item \textbf{KVM} → \textbf{Virtio} drivers for network/storage.
    \item \textbf{VirtualBox} → \textbf{Guest Additions} (mounted via ISO).
    \item Provide near-native performance for I/O operations.
\end{itemize}

\section*{4. Virtual Machine Definition Example (libvirt + KVM)}

\begin{itemize}
    \item \texttt{/etc/libvirt/qemu/} contains XML config files describing VMs:
    \begin{itemize}
        \item Memory, CPUs, disk images, network interfaces, etc.
    \end{itemize}
    \item Example snippet:
    \begin{lstlisting}[language=xml]
<domain type='kvm'>
  <name>rhel8.0</name>
  <memory unit='KiB'>4194304</memory>
  <vcpu>2</vcpu>
  <devices>
    <disk type='file' device='disk'>
      <source file='/var/lib/libvirt/images/rhel8'/>
      <target dev='vda' bus='virtio'/>
    </disk>
    <interface type='network'>
      <source network='default'/>
      <model type='virtio'/>
    </interface>
    ...
  </devices>
</domain>
    \end{lstlisting}
    \item \textbf{Networking} can be NAT-based via \texttt{virbr0} or bridged to the host network.
\end{itemize}

\section*{5. VM Disk Storage Formats}

\begin{enumerate}
    \item \textbf{QCOW2 (Copy-on-write)}
    \begin{itemize}
        \item Thin-provisioned (sparse), only consumes physical space for actual data.
        \item Can expand up to a max size.
    \end{itemize}

    \item \textbf{RAW}
    \begin{itemize}
        \item Pre-allocated, full-size image.
        \item Slight performance advantage.
    \end{itemize}

    \item \textbf{Other Storage Setups}
    \begin{itemize}
        \item Physical LVM volumes, SAN, NAS, or advanced solutions (oVirt, Red Hat Virtualization).
    \end{itemize}
\end{enumerate}

\section*{6. Cloning and Templates}

\begin{enumerate}
    \item \textbf{Templates}
    \begin{itemize}
        \item Pre-built VM images with baseline OS/configuration.
        \item Speeds deployment, reduces repetitive setup steps.
    \end{itemize}

    \item \textbf{Unique System IDs}
    \begin{itemize}
        \item Must regenerate \textbf{SSH host keys}, \textbf{D-Bus machine ID} to avoid duplicates.
        \item Example to regenerate machine ID:
        \begin{lstlisting}[language=bash]
sudo rm -f /etc/machine-id
sudo dbus-uuidgen --ensure=/etc/machine-id
        \end{lstlisting}
    \end{itemize}
\end{enumerate}

\section*{7. Cloud Infrastructure (IaaS)}

\begin{enumerate}
    \item \textbf{Compute Instances}
    \begin{itemize}
        \item Providers bill by CPU/memory usage or by instance count/time.
    \end{itemize}

    \item \textbf{Block Storage}
    \begin{itemize}
        \item Persistent storage volumes attached to VMs; performance tiers vary by cost.
    \end{itemize}

    \item \textbf{Networking}
    \begin{itemize}
        \item Cloud providers offer subnets, routing, firewalls, DNS, or hybrid on-prem/cloud networking (VPN).
    \end{itemize}

    \item \textbf{Access via SSH}
    \begin{itemize}
        \item Typically uses key-based authentication.
        \item Some providers auto-generate keys or let you upload your own.
    \end{itemize}
\end{enumerate}

\section*{8. \texttt{cloud-init} for Automated Provisioning}

\begin{enumerate}
    \item \textbf{\texttt{cloud-init}}
    \begin{itemize}
        \item Tool that runs at boot to configure system settings (network, packages, SSH keys, etc.).
        \item Uses YAML-based \textbf{cloud-config} files.
        \item Example:
        \begin{lstlisting}[language=bash]
#cloud-config
timezone: Africa/Dar_es_Salaam
hostname: test-system
apt_update: true
apt_upgrade: true
packages:
  - nginx
        \end{lstlisting}
        \item Reduces manual setup for new VMs or containers.
    \end{itemize}
\end{enumerate}

\section*{9. Containers}

\begin{enumerate}
    \item \textbf{Container Concepts}
    \begin{itemize}
        \item Isolated environment for an application.
        \item Shares host OS kernel, thus lighter than full VMs.
        \item Faster deployment and scaling, easy migration.
    \end{itemize}

    \item \textbf{\texttt{cgroups} (Control Groups)}
    \begin{itemize}
        \item Linux kernel feature limiting resource usage (CPU, memory, IO).
        \item Container engines (Docker, LXC, Kubernetes) use cgroups under the hood.
    \end{itemize}

    \item \textbf{Use Cases}
    \begin{itemize}
        \item Microservices, ephemeral workloads, dev/test environments.
    \end{itemize}
\end{enumerate}

\section*{Workbook Exercises}

\begin{enumerate}
    \item \textbf{Compare VM Types}
    \begin{itemize}
        \item Write down 3 differences between \textbf{fully virtualized} and \textbf{paravirtualized} VMs.
        \item List examples of \textbf{Type-1} vs. \textbf{Type-2} hypervisors.
    \end{itemize}

    \item \textbf{Inspect a VM Definition (libvirt)}
    \begin{itemize}
        \item On a KVM host, look at \texttt{/etc/libvirt/qemu/VM\_NAME.xml}.
        \item Identify the disk image file, CPU count, and memory assignment.
    \end{itemize}

    \item \textbf{Check Machine ID}
    \begin{itemize}
        \item On a Linux VM, run:
        \begin{lstlisting}[language=bash]
dbus-uuidgen --get
        \end{lstlisting}
        \item If cloned, try regenerating the machine ID.
        \item Discuss why identical IDs can cause conflicts.
    \end{itemize}

    \item \textbf{\texttt{cloud-init} Basics}
    \begin{itemize}
        \item Create a small \texttt{cloud-config} file to set a hostname and install a package.
        \item Discuss how it might be used in a real deployment scenario.
    \end{itemize}

    \item \textbf{Container vs. VM}
    \begin{itemize}
        \item Compare resource usage for a container vs. a full VM (e.g., Docker container vs. KVM instance).
        \item List potential advantages of containers in your environment.
    \end{itemize}
\end{enumerate}

\section*{Summary}

\begin{itemize}
    \item \textbf{Linux} supports various virtualization technologies (KVM, Xen, VirtualBox), each with different performance and integration trade-offs.
    \item Paravirtualization leverages special drivers for higher performance than fully virtualized guests.
    \item \textbf{D-Bus machine ID} and \textbf{SSH keys} must be unique for each cloned VM or template-based deployment.
    \item \textbf{\texttt{cloud-init}} automates initial OS configuration in cloud or container environments.
    \item \textbf{Containers} share the host kernel, providing lighter, faster deployment compared to full VMs, and rely on \textbf{\texttt{cgroups}} for resource isolation.
\end{itemize}

%-------------------------------------------------------
% Multiple-Choice Questions for (102.6)
%-------------------------------------------------------
\newpage
\section*{Multiple-Choice Questions for 102.6}

\begin{enumerate}[1.]

    \item Which hypervisor is described as a Type-1 (bare-metal) hypervisor that does \textbf{not} rely on an underlying operating system?  
    \begin{enumerate}[A)]
        \item VirtualBox  
        \item VMware Workstation  
        \item KVM  
        \item Xen  
    \end{enumerate}

    \item What is the main purpose of a \textbf{guest driver} in a paravirtualized environment?  
    \begin{enumerate}[A)]
        \item They hamper performance by adding extra overhead  
        \item They replace the hypervisor entirely  
        \item They help the guest OS interact with the hypervisor hardware more efficiently  
        \item They prevent kernel modules from loading  
    \end{enumerate}

    \item Which of the following statements about disk images is \textbf{correct}?  
    \begin{enumerate}[A)]
        \item The raw image format is always smaller in physical size  
        \item A 10 GB raw image file only uses 5 GB by default  
        \item Copy-on-write images cannot support snapshots  
        \item \texttt{qcow2} is a copy-on-write disk image format used by QEMU  
    \end{enumerate}

    \item Which statement accurately describes \textbf{containers}?  
    \begin{enumerate}[A)]
        \item Containers require a fully emulated BIOS and disk controllers  
        \item Containers cannot be migrated from one host to another  
        \item Containers are identical to fully virtualized machines  
        \item Containers isolate applications while sharing the host’s operating system kernel  
    \end{enumerate}

    \item Which of the following is an example of a \textbf{Type-2} hypervisor mentioned in the text?  
    \begin{enumerate}[A)]
        \item VirtualBox  
        \item Xen  
        \item KVM  
        \item Docker  
    \end{enumerate}

    \item Which command can be used to ensure a system has a D-Bus machine ID or to generate one if missing?  
    \begin{enumerate}[A)]
        \item \texttt{systemctl machine-id}  
        \item \texttt{uuidgen}  
        \item \texttt{cloud-init --machine-id}  
        \item \texttt{dbus-uuidgen --ensure}  
    \end{enumerate}

    \item Which best describes \textbf{cloud-init} as mentioned in the text?  
    \begin{enumerate}[A)]
        \item A virtualization environment used to create containers  
        \item A network configuration tool for bridging  
        \item A proprietary cloud computing platform  
        \item A vendor-neutral utility for automatically configuring new cloud-based systems at first boot  
    \end{enumerate}

    \item What is the recommended procedure when cloning a Linux VM that needs a \textbf{unique} D-Bus machine ID?  
    \begin{enumerate}[A)]
        \item Reboot the system, and it will generate a new ID automatically  
        \item No action is needed; the hypervisor handles ID generation  
        \item Remove \texttt{/etc/machine-id} and generate a new one with \texttt{dbus-uuidgen}  
        \item Request a new license from LPI  
    \end{enumerate}

    \item Which is \textbf{true} regarding copying SSH public keys with the \texttt{ssh-copy-id} command?  
    \begin{enumerate}[A)]
        \item \texttt{ssh-copy-id} can only be used on local machines, not remote servers  
        \item \texttt{ssh-copy-id} places the public key into the \texttt{authorized\_keys} file on the remote server  
        \item The private key is automatically transferred to the remote server  
        \item \texttt{ssh-copy-id} sets the public key file permission to 700  
    \end{enumerate}

    \item In a libvirt network configuration, which statement is correct regarding \textbf{bridging}?  
    \begin{enumerate}[A)]
        \item Bridging is never used by VMs  
        \item The \texttt{default.xml} might define a bridge interface named \texttt{virbr0}  
        \item NAT is never used with bridging  
        \item The bridging device must have the same name as the hypervisor  
    \end{enumerate}

    \item Which of the following statements is \textbf{true} about NAT in the libvirt \texttt{default} network definition?  
    \begin{enumerate}[A)]
        \item NAT is never used in libvirt  
        \item The default network uses NAT to forward packets to other networks  
        \item NAT requires advanced bridging configuration  
        \item NAT can only be used with a single VM  
    \end{enumerate}

    \item Which file typically stores a \textbf{symbolic link} to \texttt{/etc/machine-id}?  
    \begin{enumerate}[A)]
        \item \texttt{/usr/lib/dbus/machine-id}  
        \item \texttt{/run/machine-id}  
        \item \texttt{/var/lib/dbus/machine-id}  
        \item \texttt{/etc/dbus/machine-id}  
    \end{enumerate}

    \item Which virtualization disk provisioning approach \textbf{only} grows in size as new data is written to the disk image?  
    \begin{enumerate}[A)]
        \item RAW  
        \item Copy-on-write (COW)  
        \item Partition-based allocation  
        \item LVM-based thick provisioning  
    \end{enumerate}

    \item Which hypervisor is described in the text as both Type-1 \textbf{and} Type-2 because it integrates with the Linux kernel but also runs on a host OS?  
    \begin{enumerate}[A)]
        \item KVM  
        \item VirtualBox  
        \item Xen  
        \item VMware ESXi  
    \end{enumerate}

    \item Which of the following are considered \textbf{IaaS computing elements} for cloud-based virtualization?  
    \begin{enumerate}[A)]
        \item Computing instances, block storage, and virtual networking  
        \item Word processors, spreadsheets, and messaging apps  
        \item Email, databases, and printers  
        \item Standard Operating Procedures (SOPs)  
    \end{enumerate}

    \item When using \texttt{ssh-keygen} to generate an SSH key pair, which file extension typically indicates the \textbf{public} key file?  
    \begin{enumerate}[A)]
        \item \texttt{.priv}  
        \item \texttt{.pub}  
        \item \texttt{.asc}  
        \item \texttt{.id}  
    \end{enumerate}

    \item What is the main advantage of paravirtualized drivers (guest drivers) over fully virtualized drivers?  
    \begin{enumerate}[A)]
        \item They are less secure  
        \item They require specialized hardware that is not widely supported  
        \item They typically offer better performance by allowing the guest OS to interact directly with the hypervisor  
        \item They reduce memory usage by 70\%  
    \end{enumerate}

    \item Which of the following statements about \textbf{container technology} is correct?  
    \begin{enumerate}[A)]
        \item It always requires a separate OS kernel per container  
        \item It is always slower than a fully virtualized solution  
        \item It allows applications to run in isolated environments while sharing the host kernel  
        \item Containers cannot be migrated between hosts  
    \end{enumerate}

    \item Which command is used to \textbf{add} a public SSH key to the remote server’s \texttt{authorized\_keys} file automatically?  
    \begin{enumerate}[A)]
        \item \texttt{ssh-copy-id}  
        \item \texttt{scp}  
        \item \texttt{scp-pub}  
        \item \texttt{sftp}  
    \end{enumerate}

    \item Which statement accurately describes \textbf{live migration} in virtualization?  
    \begin{enumerate}[A)]
        \item Live migration is the process of moving a running VM from one hypervisor to another without downtime  
        \item Live migration means the guest OS must be halted first  
        \item Live migration is only possible with container technology  
        \item Live migration requires external storage with no snapshots  
    \end{enumerate}

\end{enumerate}

%-------------------------------------------------------
% Fill-in-the-Blank Questions (102.6)
%-------------------------------------------------------

\newpage
\section*{Fill-in-the-Blank Questions for 102.6}

\begin{enumerate}[1.]
\item The software platform responsible for managing hardware resources for virtual machines is called the \underline{\hspace{2cm}}.

\item When a virtual machine is aware that it is a VM and uses specialized drivers, it is referred to as a \underline{\hspace{2cm}} guest.

\item The \underline{\hspace{2cm}} file format (used by QEMU) supports copy-on-write functionality.

\item In a KVM setup, the XML configuration files for virtual machines are often located under \underline{\hspace{2cm}}.

\item The \underline{\hspace{2cm}} command can generate a new D-Bus machine ID if one does not already exist.

\item A symbolic link for the machine ID is typically found at `/var/lib/dbus/machine-id`, pointing back to \underline{\hspace{2cm}}.

\item When a virtual machine is copied to act as a \underline{\hspace{2cm}}, certain unique properties (like SSH keys or machine IDs) must be changed.

\item \underline{\hspace{2cm}} is a vendor-neutral utility used to automatically configure new cloud-based virtual machines at first boot.

\item An example of a \textbf{Type-2} hypervisor, mentioned in the text, that runs on top of an existing OS is \underline{\hspace{2cm}}.

\item \underline{\hspace{2cm}} is a method that allows a virtual machine to be moved from one hypervisor to another with minimal or no downtime.
\end{enumerate}

%=======================================================
% TOPIC 103: GNU AND UNIX COMMANDS
%=======================================================
\newpage
\chapter{Topic 103: GNU and Unix Commands}

%-------------------------------------------------------
% 103.1 Work on the command line
%-------------------------------------------------------
\newpage
\section*{103.1 Work on the command line}
\addcontentsline{toc}{section}{103.1 Work on the command line}


\textbf{Reference to LPI Objectives}
\begin{itemize}
    \item \textbf{LPIC-1 version 5.0, Exam 101, Objective 103.1}
    \item \textbf{Weight:} 4
\end{itemize}

\section*{Key Knowledge Areas}
\begin{itemize}
    \item Using single shell commands and one-line command sequences.
    \item Managing the shell environment: defining, referencing, and exporting variables.
    \item Using and editing command history.
    \item Invoking commands inside and outside of the PATH.
\end{itemize}

\section*{Important Commands, Files, and Concepts}
\begin{itemize}
    \item \textbf{bash} (shell)
    \item \textbf{echo}, \textbf{env}, \textbf{export}
    \item \textbf{pwd}, \textbf{set}, \textbf{unset}
    \item \textbf{type}, \textbf{which}
    \item \textbf{man}, \textbf{uname}
    \item \textbf{history}, \textbf{.bash\_history}
    \item \textbf{Quoting} (single quotes, double quotes, backslash)
\end{itemize}

\section*{Lesson Overview}
Mastering the command line is foundational for Linux administration. You’ll frequently need to view or modify your environment, recall and repeat past commands, and handle special characters. Below are the essentials of working efficiently from the shell.

\section*{1. Basic System and Command Information}
\begin{enumerate}
    \item \textbf{Where Am I?}: \texttt{pwd}
    \begin{itemize}
        \item Prints your current directory, e.g., \texttt{/home/user}.
        \item Example:
        \begin{lstlisting}[language=bash]
pwd
# /home/frank
        \end{lstlisting}
    \end{itemize}

    \item \textbf{System Information}: \texttt{uname -a}
    \begin{itemize}
        \item Displays kernel name, version, architecture, and more.
        \item Example:
        \begin{lstlisting}[language=bash]
uname -a
# Linux base 4.18.0-18-generic ...
        \end{lstlisting}
    \end{itemize}

    \item \textbf{Manual Pages}: \texttt{man COMMAND}
    \begin{itemize}
        \item Displays documentation for a specified command.
        \item If unsure of the exact command name, use \texttt{apropos KEYWORD}.
    \end{itemize}

    \item \textbf{Command Identification}:
    \begin{itemize}
        \item \texttt{type COMMAND}
        \begin{itemize}
            \item Tells whether it’s a shell builtin, a hashed command, or an external binary.
        \end{itemize}
        \item \texttt{which COMMAND}
        \begin{itemize}
            \item Shows the absolute path (e.g., \texttt{/usr/bin/ls}).
        \end{itemize}
    \end{itemize}
\end{enumerate}

\section*{2. Using Command History}
\begin{enumerate}
    \item \textbf{Listing Past Commands}
    \begin{itemize}
        \item \texttt{history}
        \begin{itemize}
            \item Shows a list of your previously executed commands.
        \end{itemize}
        \item Piping to \texttt{grep KEYWORD} can search through it:
        \begin{lstlisting}[language=bash]
history | grep apt
        \end{lstlisting}
    \end{itemize}

    \item \textbf{.bash\_history}
    \begin{itemize}
        \item Hidden file in your home directory storing commands.
        \item Only updates when you exit a session, so the most recent commands may not appear until logout.
    \end{itemize}

    \item \textbf{Re-executing Commands}
    \begin{itemize}
        \item \textbf{Up/Down Arrow} keys cycle through your history.
        \item Press \textbf{Enter} to execute.
        \item Saves time re-typing complex commands.
    \end{itemize}
\end{enumerate}

\section*{3. Environment Variables}
\begin{enumerate}
    \item \textbf{Listing Environment Variables}
    \begin{itemize}
        \item \texttt{env} shows exported variables (visible to child processes).
        \item \texttt{set} shows all variables and shell functions.
    \end{itemize}

    \item \textbf{Viewing Variable Values}
    \begin{itemize}
        \item \texttt{echo \$VARIABLE\_NAME}
        \item Example:
        \begin{lstlisting}[language=bash]
echo $PATH
        \end{lstlisting}
    \end{itemize}

    \item \textbf{Creating and Exporting Variables}
    \begin{itemize}
        \item \texttt{VARIABLE=value} (local to the current shell).
        \item \texttt{export VARIABLE} makes it inherited by child shells.
        \item Example:
        \begin{lstlisting}[language=bash]
myvar=hello
export myvar
        \end{lstlisting}
    \end{itemize}

    \item \textbf{Removing Variables}
    \begin{itemize}
        \item \texttt{unset VARIABLE} deletes it from the current environment.
    \end{itemize}
\end{enumerate}

\section*{4. Quoting and Special Characters}
\begin{enumerate}
    \item \textbf{Why Quote?}
    \begin{itemize}
        \item Spaces and certain symbols are interpreted by the shell.
        \item Quoting ensures the literal interpretation of special characters/spaces.
    \end{itemize}

    \item \textbf{Methods}
    \begin{itemize}
        \item \textbf{Double quotes} (\texttt{" "}): preserves most characters except \$, \texttt{\`}, \texttt{\textbackslash}, and \texttt{!} in some cases.
        \item \textbf{Single quotes} (\texttt{' '}): preserves all characters literally.
        \item \textbf{Backslash} (\texttt{\textbackslash}): escapes just the next character.
    \end{itemize}

    \item \textbf{Examples}
    \begin{itemize}
        \item Creating a file with spaces:
        \begin{lstlisting}[language=bash]
touch "my big file"
        \end{lstlisting}
        \item Removing it:
        \begin{lstlisting}[language=bash]
rm 'my big file'
        \end{lstlisting}
        \item Escaping spaces:
        \begin{lstlisting}[language=bash]
touch my\ big\ file
        \end{lstlisting}
    \end{itemize}
\end{enumerate}

\section*{Workbook Exercises}
\begin{enumerate}
    \item \textbf{Check Your Current Directory}
    \begin{itemize}
        \item Run \texttt{pwd} and verify the exact path to your home directory.
        \item Create a file there using \texttt{touch <filename>}.
    \end{itemize}

    \item \textbf{Find Your Kernel Version}
    \begin{itemize}
        \item Use \texttt{uname -a} and note the kernel version.
        \item Check \texttt{man uname} to see other possible options.
    \end{itemize}

    \item \textbf{Explore Man Pages}
    \begin{itemize}
        \item Run \texttt{man ls} and look for the \texttt{-l} option description.
        \item Use \texttt{apropos kernel} to see commands/man pages referencing “kernel.”
    \end{itemize}

    \item \textbf{Practice Command History}
    \begin{itemize}
        \item Execute 5–10 random commands (like \texttt{pwd}, \texttt{ls}, \texttt{echo test}).
        \item Run \texttt{history} and filter with \texttt{grep ls}.
        \item Press the \textbf{Up} arrow key to retrieve a previous command and re-run it.
    \end{itemize}

    \item \textbf{Experiment with Environment Variables}
    \begin{itemize}
        \item Create a variable: \texttt{myvar="test123"}.
        \item Echo it: \texttt{echo \$myvar}.
        \item Start a new shell with \texttt{bash}, check if \texttt{myvar} is available.
        \item Go back, export \texttt{myvar}, start another shell, and see if it’s now available.
        \item Remove it with \texttt{unset myvar}.
    \end{itemize}

    \item \textbf{Creating Files with Special Characters}
    \begin{itemize}
        \item Try \texttt{touch my big file} (observe the result).
        \item Now properly create the file: \texttt{touch "my big file"}.
        \item Remove it in three different ways (double quotes, single quotes, backslash-escaped).
    \end{itemize}
\end{enumerate}

\section*{Summary}
\begin{itemize}
    \item The \texttt{pwd} and \texttt{uname} commands help locate you and your system’s details.
    \item \texttt{man}, \texttt{apropos}, \texttt{type}, and \texttt{which} help you find and understand commands.
    \item \texttt{history} and the \texttt{.bash\_history} file let you recall and reuse previous commands.
    \item Environment variables (\texttt{PATH}, etc.) are easy to manage with \texttt{export}, \texttt{unset}, and \texttt{echo}.
    \item Quoting (single quotes, double quotes, or backslashes) is crucial when dealing with spaces or special characters.
\end{itemize}

%-------------------------------------------------------
% Multiple-Choice Questions for (103.1)
%-------------------------------------------------------
\newpage
\section*{Multiple-Choice Questions for 103.1}
\begin{enumerate}[1.]

    \item Which of the following commands quickly displays only the absolute pathname of an executable (without additional information)?  
    \begin{enumerate}[A)]
        \item type  
        \item file  
        \item whereis  
        \item which  
    \end{enumerate}

    \item When using \texttt{bash}, which key sequence allows you to recall and edit previously typed commands?  
    \begin{enumerate}[A)]
        \item Ctrl+H and Ctrl+G  
        \item The Up/Down arrow keys  
        \item Left-clicking on the command line  
        \item The Tab key  
    \end{enumerate}

    \item Which of the following commands can be used to remove an environment variable from the current shell session?  
    \begin{enumerate}[A)]
        \item unsetenv  
        \item erase  
        \item export --remove  
        \item unset  
    \end{enumerate}

    \item Which \texttt{uname} option prints all available system information?  
    \begin{enumerate}[A)]
        \item \texttt{-r}  
        \item \texttt{-o}  
        \item \texttt{-a}  
        \item \texttt{-v}  
    \end{enumerate}

    \item Which command will remove the file \texttt{my big file} if the file name has embedded spaces and you do NOT use quotes or backslashes correctly?  
    \begin{enumerate}[A)]
        \item \texttt{rm my\_big\_file}  
        \item \texttt{rm my big file} (interpreted as removing three separate files)  
        \item \texttt{rm 'my big file'}  
        \item \texttt{rm "my big file"}  
    \end{enumerate}

    \item Which command outputs the environment variables that are exported and accessible to child processes?  
    \begin{enumerate}[A)]
        \item env  
        \item set  
        \item grep  
        \item apropos  
    \end{enumerate}

    \item Which of the following statements is \textbf{true} regarding \texttt{type uname} showing “uname is hashed (/bin/uname)”?  
    \begin{enumerate}[A)]
        \item It means \texttt{uname} was used previously and is cached for faster lookups  
        \item It means the command is a shell builtin  
        \item It means the command no longer exists on disk  
        \item It means there is a conflict with the \texttt{uname} command location  
    \end{enumerate}

    \item By default, new local variables set in the Bash shell are only available:  
    \begin{enumerate}[A)]
        \item In the current shell session  
        \item To all newly created shells and user sessions  
        \item After a reboot only  
        \item To all users on the system  
    \end{enumerate}

    \item When you type \texttt{set | grep myvar} and see a result, but \texttt{env | grep myvar} returns nothing, what does that tell you about \texttt{myvar}?  
    \begin{enumerate}[A)]
        \item \texttt{myvar} is stored in \texttt{.bashrc}  
        \item \texttt{myvar} is a local shell variable (not exported)  
        \item \texttt{myvar} is inherited from a parent environment  
        \item \texttt{myvar} is actually a path variable  
    \end{enumerate}

    \item Which command is used to list all recent commands executed in the current user’s shell session?  
    \begin{enumerate}[A)]
        \item man  
        \item history  
        \item ls -a  
        \item more  
    \end{enumerate}

    \item The file \texttt{\textasciitilde/.bash\_history} typically contains:  
    \begin{enumerate}[A)]
        \item A script that runs every time you open your shell  
        \item A record of previously executed commands in Bash  
        \item User-defined functions  
        \item Environment variables that persist after logout  
    \end{enumerate}

    \item Which command is the quickest way to verify whether the directory \texttt{/usr/local/bin} is in your \texttt{\$PATH}?  
    \begin{enumerate}[A)]
        \item \texttt{echo \$PATH}  
        \item \texttt{man path}  
        \item \texttt{apropos local}  
        \item \texttt{ls /usr/local/bin}  
    \end{enumerate}

    \item Which command is most suitable for searching through the names and descriptions of all installed man pages when you do \textbf{not} remember the exact command name you need?  
    \begin{enumerate}[A)]
        \item more  
        \item info  
        \item apropos  
        \item tail  
    \end{enumerate}

    \item Which of these statements regarding \texttt{man} is \textbf{true}?  
    \begin{enumerate}[A)]
        \item \texttt{man} files are stored in \texttt{/etc/bash\_completion.d/}  
        \item \texttt{man} pages are often organized into separate sections  
        \item \texttt{man} only lists synonyms for commands; it does not provide usage  
        \item \texttt{man} must be run as \texttt{root} to view system documentation  
    \end{enumerate}

    \item If you type \texttt{myvar=hello} (with no spaces), what happens?  
    \begin{enumerate}[A)]
        \item This sets an environment variable globally for all shells  
        \item A local shell variable \texttt{myvar} is created with value \texttt{hello}  
        \item The variable is appended to the path  
        \item You must run \texttt{env} to permanently store that variable  
    \end{enumerate}

    \item Which of the following methods will \textbf{not} preserve special characters in a filename?  
    \begin{enumerate}[A)]
        \item Using double quotes  
        \item Using single quotes  
        \item Escaping them with a backslash (\texttt{\textbackslash})  
        \item Typing them as is, without quotes or escapes  
    \end{enumerate}

    \item If you run \texttt{exit} within a child shell, what happens?  
    \begin{enumerate}[A)]
        \item You return to the parent shell  
        \item The system reboots  
        \item It logs you out completely  
        \item The variable \texttt{\$PATH} is cleared  
    \end{enumerate}

    \item Which command’s output is typically the largest and includes all variables and functions (both local and exported)?  
    \begin{enumerate}[A)]
        \item env  
        \item type  
        \item apropos  
        \item set  
    \end{enumerate}

    \item Which of these commands is a \textbf{shell builtin} by default on most Linux systems using Bash?  
    \begin{enumerate}[A)]
        \item uname  
        \item cp  
        \item which  
        \item kill  
    \end{enumerate}

    \item What does pressing the \textbf{Up Arrow} key multiple times in Bash do?  
    \begin{enumerate}[A)]
        \item Automatically corrects the last typed command  
        \item Displays man pages for previously run commands  
        \item Logs the user out if pressed 3 times in succession  
        \item Cycles through the recent command history  
    \end{enumerate}

\end{enumerate}


%-------------------------------------------------------
% Fill-in-the-Blank Questions (103.1)
%-------------------------------------------------------

%-------------------------------------------------------
% 103.2 Process text streams using filters
%-------------------------------------------------------
\newpage
\section*{103.2 Process text streams using filters}
\addcontentsline{toc}{section}{103.2 Process text streams using filters}

\textbf{Reference to LPI Objectives}
\begin{itemize}
    \item \textbf{LPIC-1 v5, Exam 101, Objective 103.2}
    \item \textbf{Weight:} 2
\end{itemize}

\section*{Key Knowledge Areas}
\begin{itemize}
    \item Sending text files and output streams through standard text utility filters.
    \item Familiarity with GNU textutils (now part of GNU coreutils) and related commands (\texttt{sed}, \texttt{grep}, \texttt{head}, \texttt{tail}, etc.).
\end{itemize}

\section*{Important Commands and Utilities}
\begin{itemize}
    \item \texttt{bzcat}, \texttt{cat}, \texttt{cut}, \texttt{head}, \texttt{less}, \texttt{md5sum}
    \item \texttt{nl}, \texttt{od}, \texttt{paste}, \texttt{sed}, \texttt{sha256sum}, \texttt{sha512sum}
    \item \texttt{sort}, \texttt{split}, \texttt{tail}, \texttt{tr}, \texttt{uniq}, \texttt{wc}
    \item \texttt{xzcat}, \texttt{zcat}
    \item Redirection operators (\texttt{>}, \texttt{>>}) and \textbf{pipes} (\texttt{|}).
\end{itemize}

\section*{1. Quick Review: Redirections and Pipes}
\begin{enumerate}
    \item \textbf{Standard Streams}
    \begin{itemize}
        \item \textbf{stdin} (standard input): file descriptor 0 (keyboard by default).
        \item \textbf{stdout} (standard output): file descriptor 1 (screen by default).
        \item \textbf{stderr} (standard error): file descriptor 2 (screen by default).
    \end{itemize}
    \item \textbf{Redirections}
    \begin{itemize}
        \item \texttt{>} → redirect stdout to a file (overwrite).
        \item \texttt{>>} → redirect stdout to a file (append).
        \item \texttt{<} → redirect a file into stdin.
    \end{itemize}
    \item \textbf{Pipes (\texttt{|})}
    \begin{itemize}
        \item Direct output of one command as input to another.
        \item Example:
        \begin{lstlisting}[language=bash]
cat file.txt | grep "pattern"
        \end{lstlisting}
    \end{itemize}
\end{enumerate}

\section*{2. Basic Usage of \texttt{cat}}
\begin{enumerate}
    \item Concatenate Files: \texttt{cat file1 file2} → writes both files to stdout in sequence.
    \item Standard Input: Just \texttt{cat} (with no arguments) reads from stdin (keyboard).
    \item Copying Files: \texttt{cat source > destination}.
    \item Appending: \texttt{echo "new line" >> file.txt}.
\end{enumerate}

\section*{3. Viewing Compressed Files}
\begin{itemize}
    \item \texttt{bzcat} → for \texttt{.bz2} compressed files.
    \item \texttt{xzcat} → for \texttt{.xz} compressed files.
    \item \texttt{zcat} → for \texttt{.gz} compressed files.
    \item Example:
    \begin{lstlisting}[language=bash]
gzip file.txt  # produces file.txt.gz
zcat file.txt.gz
    \end{lstlisting}
\end{itemize}

\section*{4. Searching Text}
\begin{itemize}
    \item \textbf{\texttt{grep}}:
    \begin{itemize}
        \item Search for lines matching a pattern: \texttt{grep pattern file}.
        \item Common options:
        \begin{itemize}
            \item \texttt{-i} → ignore case.
            \item \texttt{-v} → invert match (show lines \textit{not} matching).
            \item \texttt{-n} → show line numbers.
        \end{itemize}
    \end{itemize}
    \item Example:
    \begin{lstlisting}[language=bash]
grep -i "this" mytextfile
# matches "This" or "this"
    \end{lstlisting}
\end{itemize}

\section*{5. Paging Through Large Output}
\begin{enumerate}
    \item \textbf{\texttt{less}}:
    \begin{itemize}
        \item Interactive pager: scroll with arrow keys, search with \texttt{/pattern}.
        \item Example:
        \begin{lstlisting}[language=bash]
less /var/log/syslog
        \end{lstlisting}
    \end{itemize}
    \item \textbf{\texttt{head}} and \textbf{\texttt{tail}}:
    \begin{itemize}
        \item \texttt{head file} → first 10 lines.
        \item \texttt{tail file} → last 10 lines.
        \item \texttt{-n <count>} → changes how many lines are shown (e.g., \texttt{head -n 5}).
    \end{itemize}
    \item \textbf{\texttt{nl}} and \textbf{\texttt{wc}}:
    \begin{itemize}
        \item \texttt{nl} → numbers each line of input.
        \item \texttt{wc} → word count, line count, etc.
        \item \texttt{wc -l} → line count only.
    \end{itemize}
\end{enumerate}

\section*{6. Editing Text Streams with \texttt{sed}}
\begin{enumerate}
    \item \textbf{Pattern Matching}
    \begin{itemize}
        \item Print only lines matching a regex: \texttt{sed -n '/regex/p' file}.
        \item Delete lines matching: \texttt{sed '/regex/d' file}.
    \end{itemize}
    \item \textbf{Find and Replace}
    \begin{itemize}
        \item \texttt{sed 's/old/new/' file}.
        \item In-place edit: \texttt{sed -i.backup 's/old/new/' file}.
    \end{itemize}
    \item Example:
    \begin{lstlisting}[language=bash]
sed -n /cat/p < ftu.txt  # prints lines containing "cat"
sed /cat/d < ftu.txt     # prints everything except lines containing "cat"
    \end{lstlisting}
\end{enumerate}

\section*{7. Ensuring Data Integrity with Checksums}
\begin{enumerate}
    \item \textbf{Checksum Tools}: \texttt{md5sum}, \texttt{sha256sum}, \texttt{sha512sum}.
    \item \textbf{Generating a Hash}:
    \begin{lstlisting}[language=bash]
sha256sum ftu.txt > sha256.txt
    \end{lstlisting}
    \item \textbf{Verifying a Hash}:
    \begin{lstlisting}[language=bash]
sha256sum -c sha256.txt
# ftu.txt: OK
    \end{lstlisting}
\end{enumerate}

\section*{8. Looking Deeper with \texttt{od} (Octal Dump)}

\begin{enumerate}
    \item \textbf{Default}
    \begin{itemize}
        \item \texttt{od file} → displays file contents in octal.
        \item Often for debugging binary or unusual text files.
    \end{itemize}
    \item \textbf{Common Options}
    \begin{itemize}
        \item \texttt{-x} → display as hexadecimal.
        \item \texttt{-c} → display as characters (escaped for non-printable).
        \item \texttt{-An} → suppress addresses/offset.
    \end{itemize}
    \item \textbf{Example}
    \begin{lstlisting}[language=bash]
    od -c file
    # shows hidden characters like \n
    \end{lstlisting}
\end{enumerate}



\section*{Workbook Exercises}


\begin{enumerate}
    \item \textbf{Basic Redirection}
    \begin{itemize}
        \item Create \texttt{test.txt}, then run \texttt{cat > test.txt} and type some lines, press \texttt{Ctrl+C} to end.
        \item Use \texttt{diff} or \texttt{cat} to confirm contents.
    \end{itemize}

    \item \textbf{Pipes}
    \begin{itemize}
        \item \texttt{ls -l /etc | grep conf}
        \item \texttt{cat /etc/passwd | wc -l} (count lines in \texttt{/etc/passwd}).
    \end{itemize}

    \item \textbf{Compressed File Viewing}
    \begin{itemize}
        \item Create a large text file (\texttt{ls -R /usr > big.txt}).
        \item Compress it with \texttt{gzip big.txt}.
        \item Use \texttt{zcat big.txt.gz | head}.
    \end{itemize}

    \item \textbf{Searching \& Paginating}
    \begin{itemize}
        \item \texttt{grep "root" /etc/passwd}
        \item \texttt{less /var/log/syslog} (scroll, search for "error" with \texttt{/error}).
    \end{itemize}

    \item \textbf{sed Basics}
    \begin{itemize}
        \item \texttt{sed -n '/root/p' /etc/passwd} → lines containing "root."
        \item \texttt{sed 's/bash/sh/' /etc/passwd | head} → replace "bash" with "sh," show first 10 lines.
    \end{itemize}

    \item \textbf{Checksum}
    \begin{itemize}
        \item Run \texttt{sha256sum ftu.txt > check.txt}.
        \item Modify \texttt{ftu.txt} and verify using \texttt{sha256sum -c check.txt} to observe the mismatch.
    \end{itemize}

    \item \textbf{Examining File Contents}
    \begin{itemize}
        \item \texttt{od -c ftu.txt} → see hidden newline chars.
        \item \texttt{od -x ftu.txt} → observe hex representation.
    \end{itemize}
\end{enumerate}

\section*{Summary}
\begin{itemize}
    \item Redirection and pipes let you chain commands and outputs.
    \item Powerful text filters include \texttt{grep}, \texttt{head}, \texttt{tail}, \texttt{less}, \texttt{nl}, \texttt{wc}, and \texttt{sed}.
    \item Checksum commands (\texttt{md5sum}, \texttt{sha256sum}, \texttt{sha512sum}) ensure data integrity.
    \item Use \texttt{od} to reveal hidden or binary data in files.
    \item Mastering these techniques streamlines text processing, a crucial skill for Linux administration.
\end{itemize}

%-------------------------------------------------------
% Multiple-Choice Questions for (103.2)
%-------------------------------------------------------
\newpage
\section*{Multiple-Choice Questions for 103.2}
\begin{enumerate}[1.]

    \item Which command reads from standard input if no file is specified and echoes the input to standard output?  
    \begin{enumerate}[A)]
        \item grep  
        \item tail  
        \item sed  
        \item cat  
    \end{enumerate}

    \item Which redirection operator \textbf{creates} or \textbf{overwrites} a file with the output of a command?  
    \begin{enumerate}[A)]
        \item >  
        \item >>  
        \item |  
        \item \&>  
    \end{enumerate}

    \item Which command can be used to display specific lines that match a pattern in a text file, \textbf{ignoring} case differences when the \texttt{-i} option is used?  
    \begin{enumerate}[A)]
        \item sort  
        \item head  
        \item grep  
        \item md5sum  
    \end{enumerate}

    \item Which command is commonly used to paginate output, allowing you to scroll through text using the arrow keys?  
    \begin{enumerate}[A)]
        \item less  
        \item cat  
        \item nl  
        \item tail  
    \end{enumerate}

    \item Which command shows only the last ten lines of a file by default?  
    \begin{enumerate}[A)]
        \item head  
        \item tail  
        \item cut  
        \item split  
    \end{enumerate}

    \item Which command can be used to generate or check the cryptographic integrity of a file using \textbf{MD5} hashing?  
    \begin{enumerate}[A)]
        \item md5sum  
        \item sha256sum  
        \item sha512sum  
        \item od  
    \end{enumerate}

    \item Which command is considered a \textbf{stream editor} that can filter and transform text, including find-and-replace operations?  
    \begin{enumerate}[A)]
        \item nl  
        \item wc  
        \item uniq  
        \item sed  
    \end{enumerate}

    \item Which command can be used to \textbf{display} a file in \textbf{octal} or \textbf{hexadecimal} representation (helpful for debugging)?  
    \begin{enumerate}[A)]
        \item less  
        \item paste  
        \item od  
        \item zcat  
    \end{enumerate}

    \item If you want to \textbf{append} the output of a command to an existing file (without overwriting), which redirection operator should you use?  
    \begin{enumerate}[A)]
        \item >  
        \item \&>  
        \item 2>  
        \item >>  
    \end{enumerate}

    \item Which command, by default, displays the \textbf{first 10 lines} of a file?  
    \begin{enumerate}[A)]
        \item tail  
        \item wc  
        \item head  
        \item cut  
    \end{enumerate}

    \item Which command is used to \textbf{decompress and display} the content of a \texttt{.gz} file without explicitly creating an uncompressed file?  
    \begin{enumerate}[A)]
        \item gzip  
        \item bzcat  
        \item xzcat  
        \item zcat  
    \end{enumerate}

    \item Which option with \texttt{gzip} activates \textbf{verbose} mode to show what is happening during compression?  
    \begin{enumerate}[A)]
        \item -v  
        \item -n  
        \item -c  
        \item -q  
    \end{enumerate}

    \item Which command can \textbf{enumerate} lines of output (by placing a line number at the beginning of each line)?  
    \begin{enumerate}[A)]
        \item wc  
        \item nl  
        \item sed  
        \item sha512sum  
    \end{enumerate}

    \item Which command can \textbf{merge lines} from multiple files side-by-side into columns?  
    \begin{enumerate}[A)]
        \item sort  
        \item uniq  
        \item paste  
        \item split  
    \end{enumerate}

    \item Which command is commonly used to \textbf{sort} lines of text in \textbf{alphabetical} or \textbf{numerical} order?  
    \begin{enumerate}[A)]
        \item split  
        \item sort  
        \item tr  
        \item nl  
    \end{enumerate}

    \item Which command is used to \textbf{transform} or \textbf{translate} characters from standard input (for example, converting uppercase to lowercase)?  
    \begin{enumerate}[A)]
        \item od  
        \item cat  
        \item tr  
        \item head  
    \end{enumerate}

    \item Which tool can be used to verify file integrity using an \textbf{SHA-256} hash?  
    \begin{enumerate}[A)]
        \item sha256sum  
        \item zcat  
        \item md5sum  
        \item tail  
    \end{enumerate}

    \item Which of these commands \textbf{decompresses} a file using the \textbf{bzip2} algorithm and sends its content to standard output?  
    \begin{enumerate}[A)]
        \item zcat  
        \item bzcat  
        \item sha512sum  
        \item nl  
    \end{enumerate}

    \item Which of the following commands will \textbf{remove duplicate lines} from a sorted list?  
    \begin{enumerate}[A)]
        \item paste  
        \item uniq  
        \item od  
        \item grep  
    \end{enumerate}

    \item Which command is used to \textbf{split} large files into smaller parts?  
    \begin{enumerate}[A)]
        \item tr  
        \item md5sum  
        \item nl  
        \item split  
    \end{enumerate}

\end{enumerate}

%-------------------------------------------------------
% Fill-in-the-Blank Questions (103.2)
%-------------------------------------------------------

%-------------------------------------------------------
% 103.3 Perform basic file management
%-------------------------------------------------------
\newpage
\section*{103.3 Perform basic file management}
\addcontentsline{toc}{section}{103.3 Perform basic file management}

\textbf{Reference to LPI Objectives}
\begin{itemize}
    \item \textbf{LPIC-1 v5, Exam 101, Objective 103.3}
    \item \textbf{Weight:} 4
\end{itemize}

\section*{Key Knowledge Areas}
\begin{itemize}
    \item Copying, moving, and removing files/directories (individually and recursively).
    \item Using wildcards (file globbing) for matching patterns.
    \item Locating files using \texttt{find} (by type, size, time).
    \item Using \texttt{tar}, \texttt{cpio}, \texttt{dd} for archiving, copying, and backup tasks.
\end{itemize}

\section*{Important Commands and Utilities}
\begin{itemize}
    \item \texttt{cp}, \texttt{mv}, \texttt{ls}, \texttt{rm}, \texttt{rmdir}, \texttt{mkdir}, \texttt{touch}
    \item \texttt{find}
    \item \texttt{tar}, \texttt{cpio}, \texttt{dd}
    \item \texttt{gzip}, \texttt{gunzip}, \texttt{bzip2}, \texttt{bunzip2}
    \item \texttt{file} (to identify file type)
    \item Wildcards: \texttt{*}, \texttt{?}, \texttt{[ ]}
\end{itemize}

\section*{1. File Listing and Basic Navigation}
\subsection*{1. \texttt{ls}}
\begin{itemize}
    \item \texttt{ls} lists contents of a directory.
    \item Common options:
    \begin{itemize}
        \item \texttt{-l} -- long listing (permissions, owner, size, date/time).
        \item \texttt{-a} -- include hidden files (dotfiles).
        \item \texttt{-h} -- human-readable sizes.
        \item \texttt{-R} -- list contents recursively.
    \end{itemize}
    \item Example:
\begin{lstlisting}[language=bash]
ls -lh /var/log
\end{lstlisting}
\end{itemize}

\subsection*{2. \texttt{touch}}
\begin{itemize}
    \item Creates empty files or updates file timestamps.
    \item Example:
\begin{lstlisting}[language=bash]
touch myfile.txt
# creates an empty file if it doesn't exist
\end{lstlisting}
\end{itemize}

\section*{2. Creating and Removing Directories}
\subsection*{1. \texttt{mkdir}}
\begin{itemize}
    \item Make new directories.
    \item \texttt{mkdir dir1} -- creates \texttt{dir1}.
    \item \texttt{mkdir -p parents/children} -- creates a nested directory path if it doesn’t already exist.
\end{itemize}

\subsection*{2. \texttt{rmdir}}
\begin{itemize}
    \item Remove empty directories.
    \item Fails if directory is not empty.
    \item \texttt{rmdir -p parents/children} -- removes nested directories if all are empty.
\end{itemize}

\section*{3. Copying, Moving, and Deleting Files}
\subsection*{1. \texttt{cp} (Copy)}
\begin{itemize}
    \item \texttt{cp file1 dir2} -- copy \texttt{file1} into \texttt{dir2}.
    \item \texttt{cp -r dir1 dir2} -- copy directory \texttt{dir1} recursively into \texttt{dir2}.
    \item Useful options:
    \begin{itemize}
        \item \texttt{-i} -- prompt before overwrite.
        \item \texttt{-f} -- force overwrite.
    \end{itemize}
\end{itemize}

\subsection*{2. \texttt{mv} (Move / Rename)}
\begin{itemize}
    \item \texttt{mv file1 dir2} -- move \texttt{file1} into \texttt{dir2}.
    \item \texttt{mv oldname newname} -- rename a file.
    \item Options:
    \begin{itemize}
        \item \texttt{-i} -- prompt before overwrite.
        \item \texttt{-f} -- force.
    \end{itemize}
\end{itemize}

\subsection*{3. \texttt{rm} (Remove)}
\begin{itemize}
    \item \texttt{rm file1 file2} -- remove multiple files.
    \item \texttt{rm -r dir1} -- remove \texttt{dir1} and its contents recursively.
    \item \texttt{rm -i file1} -- prompt before removal.
    \item \texttt{rm -f file1} -- force removal (no prompt).
    \item \textbf{WARNING}: \texttt{rm -rf /} is very dangerous.
\end{itemize}

\section*{4. Wildcards (File Globbing)}
\begin{itemize}
    \item \textbf{\texttt{*}} (asterisk): matches zero or more characters.
    \begin{itemize}
        \item Example: \texttt{ls *.txt} -- lists all \texttt{.txt} files.
    \end{itemize}
    \item \textbf{\texttt{?}} (question mark): matches exactly one character.
    \begin{itemize}
        \item Example: \texttt{ls l?st.txt} -- matches \texttt{last.txt}, \texttt{lest.txt}, \texttt{list.txt}.
    \end{itemize}
    \item \textbf{\texttt{[ ]}} (brackets): matches any one character from the group/range inside.
    \begin{itemize}
        \item Example: \texttt{ls file[1-3].txt} -- matches \texttt{file1.txt}, \texttt{file2.txt}, \texttt{file3.txt}.
    \end{itemize}
\end{itemize}
Wildcards can be combined:

\begin{lstlisting}[language=bash]
ls [plf]?st*
# e.g., matches last.txt, lest.txt, list.txt, past.txt
\end{lstlisting}

\section*{5. Finding Files: \texttt{find}}
\begin{lstlisting}[language=bash]
find STARTING_PATH [OPTIONS] [EXPRESSION]
\end{lstlisting}

\subsection*{1. Search by Name}
\begin{itemize}
    \item \texttt{find . -name "myfile.txt"} -- find \texttt{myfile.txt} in current directory.
    \item \texttt{find /home -iname "*.png"} -- case-insensitive, all \texttt{.png} under \texttt{/home}.
\end{itemize}

\subsection*{2. Search by Type}
\begin{itemize}
    \item \texttt{-type f} (regular file), \texttt{-type d} (directory), \texttt{-type l} (symlink).
    \item Example: \texttt{find /var -type d -name "log"}.
\end{itemize}

\subsection*{3. Search by Size}
\begin{itemize}
    \item \texttt{-size +2G} (bigger than 2GB).
    \item \texttt{-size -20M} (smaller than 20MB).
    \item Suffixes: \texttt{b} (bytes), \texttt{k} (KB), \texttt{M} (MB), \texttt{G} (GB).
\end{itemize}

\subsection*{4. Search by Modification Time}
\begin{itemize}
    \item \texttt{-mtime N} \textrightarrow changed exactly \texttt{N} days ago.
    \item \texttt{-mtime +N} \textrightarrow changed more than \texttt{N} days ago.
    \item \texttt{-mtime -N} \textrightarrow changed less than \texttt{N} days ago.
    \item Example:
        \begin{lstlisting}[language=bash]
find /etc -name "*.conf" -mtime -3
# conf files changed less than 3 days ago
        \end{lstlisting}
\end{itemize}


\subsection*{5. Act on Results (\texttt{-exec})}
\begin{itemize}
    \item \texttt{-exec COMMAND \{\} \;} -- run a command on each match.
    \item Example:
\begin{lstlisting}[language=bash]
find . -name "*.bak" -exec rm {} \;
\end{lstlisting}
    \item Or use \texttt{-delete} to remove matches automatically:
\begin{lstlisting}[language=bash]
find . -name "*.bak" -delete
\end{lstlisting}
\end{itemize}

\section*{6. Archiving and Compression}
\subsection*{6.1 \texttt{tar}}
\begin{enumerate}
    \item {\textbf{Creating an Archive}}
    \begin{lstlisting}[language=bash]
tar -cvf archive.tar dir1 dir2
# -c: create, -v: verbose, -f: specify file
    \end{lstlisting}
    \item {\textbf{Extracting}}
    \begin{lstlisting}[language=bash]
tar -xvf archive.tar
# -x: extract
    \end{lstlisting}

    \item{\textbf{Compression}}
    \begin{itemize}
        \item \texttt{-z} for gzip \textrightarrow \texttt{.tar.gz} or \texttt{.tgz}:
    \begin{lstlisting}[language=bash]
tar -czvf archive.tar.gz dir1
tar -xzvf archive.tar.gz
    \end{lstlisting}
        \item \texttt{-j} for bzip2 \textrightarrow \texttt{.tar.bz2}:
    \begin{lstlisting}[language=bash]
tar -cjvf archive.tar.bz2 dir1
tar -xjvf archive.tar.bz2
    \end{lstlisting}
    \end{itemize}

\end{enumerate}



\subsection*{6.2 \texttt{gzip} / \texttt{bzip2}}
\begin{itemize}
    \item \texttt{gzip file} \textrightarrow creates \texttt{file.gz}.
    \item \texttt{gunzip file.gz} \textrightarrow uncompress.
    \item \texttt{bzip2 file} \textrightarrow creates \texttt{file.bz2}.
    \item \texttt{bunzip2 file.bz2} \textrightarrow uncompress.
\end{itemize}

\subsection*{6.3 \texttt{cpio}}
\begin{itemize}
    \item Create:
\begin{lstlisting}[language=bash]
ls | cpio -o > archive.cpio
\end{lstlisting}
    \item Extract:
\begin{lstlisting}[language=bash]
cpio -id < archive.cpio
\end{lstlisting}
\end{itemize}

\subsection*{6.4 \texttt{dd}}
\begin{itemize}
    \item General form: \texttt{dd if=INFILE of=OUTFILE [options]}.
    \item Example: copy a file:
\begin{lstlisting}[language=bash]
dd if=oldfile of=newfile status=progress
\end{lstlisting}
    \item Convert text to uppercase:
\begin{lstlisting}[language=bash]
dd if=oldfile of=newfile conv=ucase
\end{lstlisting}
    \item Disk backup (be cautious):
\begin{lstlisting}[language=bash]
dd if=/dev/sda of=backup.dd bs=4096
\end{lstlisting}
\end{itemize}

\section*{Workbook Exercises}
\begin{enumerate}
    \item \textbf{Basic Operations}
    \begin{itemize}
        \item Create a directory \texttt{testdir} with \texttt{mkdir testdir}.
        \item Inside \texttt{testdir}, create files (\texttt{touch file1 file2}).
        \item List them (\texttt{ls -l}), then copy them into a new directory \texttt{copydir}.
        \item Rename one file in \texttt{copydir} to \texttt{file3}.
        \item Remove \texttt{copydir} recursively.
    \end{itemize}
    \item \textbf{Globbing}
    \begin{itemize}
        \item Create files: \texttt{fileA}, \texttt{fileB}, \texttt{fileX}, \texttt{file12}, \texttt{f\_test}, etc.
        \item Use wildcards to list or remove subsets (\texttt{ls f*}, \texttt{rm file?}, etc.).
    \end{itemize}
    \item \textbf{Finding Files}
    \begin{itemize}
        \item Run:
\begin{lstlisting}[language=bash]
find . -name "*.sh"
\end{lstlisting}
        \item Search by size (\texttt{+1M}, etc.).
        \item Use \texttt{-exec echo \{\} \;} to print each match.
    \end{itemize}
    \item \textbf{Archiving \& Compressing}
    \begin{itemize}
        \item Create a tar archive of a test directory:
\begin{lstlisting}[language=bash]
tar -cvf myarchive.tar testdir
\end{lstlisting}
        \item Compress it (\texttt{gzip myarchive.tar}) or do it in one step (\texttt{-z}):
\begin{lstlisting}[language=bash]
tar -czvf myarchive.tgz testdir
\end{lstlisting}
        \item Extract into a new location:
\begin{lstlisting}[language=bash]
tar -xzvf myarchive.tgz -C /tmp
\end{lstlisting}
    \end{itemize}
    \item \textbf{Using \texttt{dd}}
    \begin{itemize}
        \item Copy a file using \texttt{dd}, e.g.:
\begin{lstlisting}[language=bash]
dd if=testfile of=testfile_copy bs=1K status=progress
\end{lstlisting}
        \item Verify both files with \texttt{diff} or \texttt{cmp}.
    \end{itemize}
\end{enumerate}

\section*{Summary}
\begin{itemize}
    \item \texttt{ls} shows file details; \texttt{mkdir}, \texttt{rmdir} create/remove directories.
    \item \texttt{cp}, \texttt{mv}, \texttt{rm} handle copying, moving, renaming, and deleting files/directories.
    \item Use \texttt{-r} for recursive operations on directories.
    \item File globbing (\texttt{*}, \texttt{?}, \texttt{[ ]}) simplifies specifying multiple files in commands.
    \item \texttt{find} locates files by name, type, time, or size, and can act on them using \texttt{-exec} or \texttt{-delete}.
    \item \texttt{tar}, \texttt{cpio}, \texttt{dd} provide archiving, backup, and data copying capabilities, optionally with compression (gzip, bzip2).
\end{itemize}


%-------------------------------------------------------
% Multiple-Choice Questions for (103.3)
%-------------------------------------------------------
\newpage
\section*{Multiple-Choice Questions for 103.3}
\begin{enumerate}[1.]

    \item Which command is used to list files in the current directory in a \textbf{human-readable} format (showing sizes like 4.0K, 2.1M, etc.)?  
    \begin{enumerate}[A)]
        \item ls -lh  
        \item ls -lr  
        \item ls -a  
        \item ls -sd  
    \end{enumerate}

    \item Which option with the \texttt{rm} command \textbf{prompts} the user before removing the file?  
    \begin{enumerate}[A)]
        \item -r  
        \item -f  
        \item -i  
        \item -d  
    \end{enumerate}

    \item In the context of wildcard usage, what does the \texttt{*} (asterisk) match?  
    \begin{enumerate}[A)]
        \item Zero or more occurrences of \textbf{any} character  
        \item Exactly one character  
        \item Only hidden files  
        \item Only directories  
    \end{enumerate}

    \item Which command \textbf{removes} an \textbf{empty directory}?  
    \begin{enumerate}[A)]
        \item rm -r  
        \item rmdir  
        \item rm -rf  
        \item cpio  
    \end{enumerate}

    \item Which option for the \texttt{cp} command is used to copy directories \textbf{recursively}?  
    \begin{enumerate}[A)]
        \item -v  
        \item -l  
        \item -u  
        \item -r  
    \end{enumerate}

    \item In the output of \texttt{ls -l}, which character (in the first column) indicates a \textbf{directory}?  
    \begin{enumerate}[A)]
        \item -  
        \item s  
        \item c  
        \item d  
    \end{enumerate}

    \item Which command can be used to \textbf{create an empty file}?  
    \begin{enumerate}[A)]
        \item touch  
        \item newfile  
        \item cp -0  
        \item mv -n  
    \end{enumerate}

    \item What is the effect of running \texttt{mv oldname newname}?  
    \begin{enumerate}[A)]
        \item Copies \texttt{oldname} to \texttt{newname}  
        \item Renames \texttt{oldname} to \texttt{newname}  
        \item Asks for confirmation before removing \texttt{oldname}  
        \item None of the above  
    \end{enumerate}

    \item Which command displays the \textbf{type} of a file (e.g., text, directory, etc.)?  
    \begin{enumerate}[A)]
        \item file  
        \item type  
        \item grep  
        \item dd  
    \end{enumerate}

    \item Which command is used to \textbf{remove} a file \textbf{permanently} without prompts, ignoring nonexistent files?  
    \begin{enumerate}[A)]
        \item rm -f  
        \item rm -i  
        \item rm -r  
        \item rm -rf  
    \end{enumerate}

    \item Which command will \textbf{list hidden files} in the current directory?  
    \begin{enumerate}[A)]
        \item ls -l  
        \item ls -h  
        \item ls -a  
        \item ls -R  
    \end{enumerate}

    \item Which of the following wildcards matches \textbf{exactly one character}?  
    \begin{enumerate}[A)]
        \item *  
        \item [ ]  
        \item !  
        \item ?  
    \end{enumerate}

    \item Which command can \textbf{create} a directory named \texttt{parents/children} along with the necessary \textbf{parent} directory if it doesn’t exist?  
    \begin{enumerate}[A)]
        \item mkdir -r parents/children  
        \item mkdir --create parents/children  
        \item mkdir -p parents/children  
        \item mkdir -m parents/children  
    \end{enumerate}

    \item What does the following command do?  
    \begin{verbatim}
    cp -r myfiles/ backups/
    \end{verbatim}
    \begin{enumerate}[A)]
        \item Copies the entire \texttt{myfiles} directory, including its contents, to \texttt{backups}  
        \item Moves all content from \texttt{backups} to \texttt{myfiles}  
        \item Copies only hidden files from \texttt{myfiles} to \texttt{backups}  
        \item Backs up the entire filesystem  
    \end{enumerate}

    \item Which command is used to \textbf{move or rename} files in Linux?  
    \begin{enumerate}[A)]
        \item cp  
        \item rename  
        \item mv  
        \item rm  
    \end{enumerate}

    \item When using the \textbf{mv} command, which option \textbf{prompts} you before overwriting an existing file?  
    \begin{enumerate}[A)]
        \item -f  
        \item -n  
        \item -r  
        \item -i  
    \end{enumerate}

    \item In the context of the \texttt{find} command, what does \texttt{-type f} represent?  
    \begin{enumerate}[A)]
        \item Search for directories  
        \item Search for regular files  
        \item Search for symbolic links  
        \item Search for special files  
    \end{enumerate}

    \item Which command \textbf{recursively removes} a directory and all its contents?  
    \begin{enumerate}[A)]
        \item rmdir -r  
        \item rm -r  
        \item rm  
        \item rmdir  
    \end{enumerate}

    \item Which command can be used to \textbf{list} the content of a directory \textbf{together with its subdirectories}?  
    \begin{enumerate}[A)]
        \item ls -R  
        \item ls -a  
        \item ls -l  
        \item ls -m  
    \end{enumerate}

    \item Which wildcard expression would match filenames that start with \texttt{l}, followed by any \textbf{single} character, and end with \texttt{st.txt}?  
    \begin{enumerate}[A)]
        \item l*[st].txt  
        \item l?st.txt  
        \item l??st?.txt  
        \item l?st.[txt]  
    \end{enumerate}

\end{enumerate}
%-------------------------------------------------------
% Fill-in-the-Blank Questions (103.3)
%-------------------------------------------------------

%-------------------------------------------------------
% 103.4 Use streams, pipes and redirects
%-------------------------------------------------------
\newpage
\section*{103.4 Use streams, pipes and redirects}
\addcontentsline{toc}{section}{103.4 Use streams, pipes and redirects}

\textbf{Reference to LPI Objectives:}  
\begin{itemize}
    \item \textbf{LPIC-1 v5, Exam 101, Objective 103.4}  
    \item \textbf{Weight:} 4  
\end{itemize}

\subsection*{Key Knowledge Areas}
\begin{itemize}
    \item Redirecting standard input (stdin), standard output (stdout), and standard error (stderr).  
    \item Piping output of one command into another command’s input.  
    \item Using output of one command as arguments to another command.  
    \item Sending output to both stdout and a file.
\end{itemize}

\subsection*{Important Commands, Files, and Utilities}
\begin{itemize}
    \item \texttt{tee}, \texttt{xargs}  
    \item Redirection operators (\texttt{>}, \texttt{>>}, \texttt{<}, \texttt{<<}, \texttt{<<<}, \texttt{2>}, \texttt{\&>}, etc.)  
    \item Pipes (\texttt{|})  
    \item Command substitution (\texttt{`command`} or \texttt{\$(command)})
\end{itemize}

\section*{1. Standard File Descriptors}
\begin{enumerate}
    \item \textbf{stdin}: file descriptor \textbf{0} (normally keyboard input).  
    \item \textbf{stdout}: file descriptor \textbf{1} (normally terminal display).  
    \item \textbf{stderr}: file descriptor \textbf{2} (normally terminal display for errors).
\end{enumerate}

\section*{2. Redirection}
\subsection*{1. Output Redirection}
\begin{itemize}
    \item \texttt{>} : redirect stdout (file descriptor 1) to a file (overwrite).  
    \item \texttt{>>}: append stdout to a file.  
    \item \texttt{2>}: redirect stderr (file descriptor 2).  
    \item \texttt{\&>} or \texttt{>\&}: redirect \textbf{both} stdout and stderr to a file.  
    \item Example:
    \begin{lstlisting}[language=bash]
command > file.txt        # overwrites file.txt
command >> file.txt       # appends to file.txt
command 2> errors.txt     # only stderr to errors.txt
command &> all_output.txt # stdout and stderr
    \end{lstlisting}
\end{itemize}



\subsection*{2. Input Redirection}
\begin{itemize}
    \item \texttt{<} : read file content into stdin.  
    \item Example:
    \begin{lstlisting}[language=bash]
command < file.txt
    \end{lstlisting}
    \item Usually, commands can also specify a file directly (e.g., \texttt{cat file.txt}), but \texttt{<} can be used if needed.
\end{itemize}


\section*{3. Here Documents and Here Strings}

\subsection*{1. Here Document (\texttt{<<})}
\begin{itemize}
    \item Multi-line string as stdin to a command.  
    \item Terminates on a line containing a marker (like \texttt{EOF}).


\begin{lstlisting}[language=bash]
cat << EOF
line 1
line 2
EOF
\end{lstlisting}


    \item Everything up to the terminating word is fed to the command’s stdin.
\end{itemize}

\subsection*{2. Here String (\texttt{< < <})}
\begin{itemize}
    \item Single-line string to a command’s stdin.


    \item Example:
\begin{lstlisting}[language=bash]
wc -c <<< "hello world"
\end{lstlisting}


    \item If the string contains spaces, put it in quotes.
\end{itemize}

 

\section*{4. Pipes}

\subsection*{1. Definition}
\begin{itemize}
    \item \texttt{|} connects stdout of one command to stdin of another.  
    \item Multiple pipes can chain many commands.
\end{itemize}

\subsection*{2. Basic Example}
\begin{itemize}
    \item \texttt{cat file.txt | grep "pattern"}
    \item The \texttt{cat} output goes to grep’s input.
\end{itemize}

\subsection*{3. Combining with Redirection}
\begin{itemize}
    \item \texttt{command1 2>\&1 | command2} merges stderr into stdout first, so second command can read both.
\end{itemize}

\subsection*{4. tee}
\begin{itemize}
    \item Splits output so you can see on screen \textbf{and} write to a file.  
    \item \texttt{command | tee file.txt}
    \begin{itemize}
        \item Output goes to screen and also saved into \texttt{file.txt}.
    \end{itemize}
    \item \texttt{-a} appends instead of overwriting.
\end{itemize}

\section*{5. Command Substitution}

\subsection*{1. Syntax}
\begin{itemize}
    \item \texttt{\`command\`} (backticks) or \texttt{\$(command)}
    \item Example (using \texttt{\$()} recommended):

\begin{lstlisting}[language=bash]
TODAY=$(date +%Y-%m-%d)
echo "Today's date is $TODAY"
\end{lstlisting}

    \item Used to store command output in variables or pass as arguments to other commands:

\begin{lstlisting}[language=bash]
mkdir dir-$(date +%Y%m%d)
\end{lstlisting}
\end{itemize}

\section*{6. Using xargs}

\subsection*{1. Purpose}
\begin{itemize}
    \item Takes a list (from stdin) and builds arguments for another command.  
    \item Commonly used with \texttt{find} output or other multi-line data.
\end{itemize}

\subsection*{2. Basic Example}
\begin{lstlisting}[language=bash]
find . -name "*.txt" | xargs rm
\end{lstlisting}

\begin{itemize}
    \item This removes all \texttt{.txt} files found.  
    \item If filenames have spaces, use \texttt{-print0} in \texttt{find} and \texttt{-0} in \texttt{xargs}:
\begin{lstlisting}[language=bash]
find . -name "*.txt" -print0 | xargs -0 rm
\end{lstlisting}

\end{itemize}

\subsection*{3. Options}
\begin{itemize}
    \item \texttt{-n 1}: run the specified command once per line.  
    \item \texttt{-I \{\}}: placeholder to control where the arguments go in the command line.
    \item Example:


\begin{lstlisting}[language=bash]
find . -name "*.jpg" | xargs -I {} mv {} /tmp
\end{lstlisting}


\begin{itemize}
    \item Moves each \texttt{.jpg} to \texttt{/tmp}.
\end{itemize}
\end{itemize}
\section*{Workbook Exercises}

\subsection*{1. Redirecting Output}
\begin{itemize}
    \item Run \texttt{ls -l /nonexistentdir 2> error.txt}.  
    \item Inspect \texttt{error.txt} to see stderr captured.
\end{itemize}

\subsection*{2. Piping}
\begin{itemize}
    \item Display line count of \texttt{/etc/passwd}:


\begin{lstlisting}[language=bash]
cat /etc/passwd | wc -l
\end{lstlisting}


    \item Or shorter: \texttt{wc -l < /etc/passwd}.
\end{itemize}

\subsection*{3. Here Document}
\begin{itemize}
    \item Create a test file with multiple lines:


\begin{lstlisting}[language=bash]
cat << EOF > mytest.txt
line1
line2
line3
EOF
\end{lstlisting}


    \item Check contents with \texttt{cat mytest.txt}.
\end{itemize}

\subsection*{4. tee}
\begin{itemize}
    \item Pipe a command’s output to a file and terminal:


\begin{lstlisting}[language=bash]
ls -l | tee listing.txt
\end{lstlisting}

\end{itemize}


\subsection*{5. Command Substitution}
\begin{itemize}
    \item Store today’s date in a variable:


\begin{lstlisting}[language=bash]
TODAY=$(date +%Y-%m-%d)
echo $TODAY
\end{lstlisting}


    \item Use it to create a directory, e.g. \texttt{mkdir backup-\$TODAY}.
\end{itemize}

\subsection*{6. xargs}
\begin{itemize}
    \item Create some test files, e.g. \texttt{touch file1 file 2 "file space.txt"}.  
    \item Use \texttt{find} and \texttt{xargs} with \texttt{-print0}/\texttt{-0} to remove them:


\begin{lstlisting}[language=bash]
find . -name "file*" -print0 | xargs -0 rm
\end{lstlisting}
\end{itemize}

\section*{Summary}

\begin{itemize}
    \item \textbf{stdin} (0), \textbf{stdout} (1), and \textbf{stderr} (2) are standard I/O channels.
    \item \textbf{Redirects} (\texttt{>}, \texttt{>>}, \texttt{2>}, \texttt{<}, \texttt{<<}, etc.) move data between commands and files.
    \item \textbf{Pipes} (\texttt{|}) send one command’s output to another command’s input.
    \item \textbf{tee} duplicates data to both stdout and a file.
    \item \textbf{Command substitution} (\texttt{\`cmd\`} / \texttt{\$(cmd)}) captures a command’s output for variables or arguments.
    \item \textbf{xargs} transforms stdin lines into command arguments, often used with \texttt{find}.
\end{itemize}


%-------------------------------------------------------
% Multiple-Choice Questions for (103.4)
%-------------------------------------------------------
\newpage
\section*{Multiple-Choice Questions for 103.4}
\begin{enumerate}[1.]

    \item What character is used to create a pipeline in Linux?  
    \begin{enumerate}[A)]
        \item <  
        \item >>  
        \item |  
        \item \&  
    \end{enumerate}

    \item In a pipeline, data flows from \_\_\_ to \_\_\_.  
    \begin{enumerate}[A)]
        \item Filesystem to memory  
        \item Left to right  
        \item Right to left  
        \item Memory to filesystem  
    \end{enumerate}

    \item What command allows the output of a pipeline to be displayed on the screen and written to a file simultaneously?  
    \begin{enumerate}[A)]
        \item tee  
        \item uniq  
        \item grep  
        \item wc  
    \end{enumerate}

    \item Which of the following is a valid method to redirect standard error to standard output in Bash?  
    \begin{enumerate}[A)]
        \item |  
        \item 2>\&1  
        \item <\&  
        \item >>  
    \end{enumerate}

    \item In Bash, the \texttt{\$(...)} syntax is used for \_\_\_.  
    \begin{enumerate}[A)]
        \item Redirection  
        \item Command substitution  
        \item Background execution  
        \item Piping  
    \end{enumerate}

    \item Which command is used to pass the output of one program as arguments to another program?  
    \begin{enumerate}[A)]
        \item uniq  
        \item xargs  
        \item grep  
        \item wc  
    \end{enumerate}

    \item When using \texttt{find} with \texttt{xargs}, which option ensures correct handling of paths with spaces?  
    \begin{enumerate}[A)]
        \item -L  
        \item -n 1  
        \item -print0 and -0  
        \item -exec  
    \end{enumerate}

    \item Which command is used to sort output numerically?  
    \begin{enumerate}[A)]
        \item tee  
        \item uniq  
        \item sort -n  
        \item wc  
    \end{enumerate}

    \item The \texttt{xargs} option \_\_\_ specifies how many arguments to use per command execution.  
    \begin{enumerate}[A)]
        \item -L  
        \item -I  
        \item -n  
        \item -0  
    \end{enumerate}

    \item In a pipeline, which program skips duplicate lines?  
    \begin{enumerate}[A)]
        \item wc  
        \item tee  
        \item uniq  
        \item grep  
    \end{enumerate}

    \item Which of the following can capture only the standard output of a process in a pipeline?  
    \begin{enumerate}[A)]
        \item pipe (\texttt{|})  
        \item Redirect (\texttt{>})  
        \item uniq  
        \item grep  
    \end{enumerate}

    \item Which \texttt{xargs} option allows substituting input values anywhere in the target command?  
    \begin{enumerate}[A)]
        \item -n  
        \item -L  
        \item -I  
        \item -0  
    \end{enumerate}

    \item Command substitution in Bash allows you to \_\_\_.  
    \begin{enumerate}[A)]
        \item Send stderr to stdout  
        \item Use command output as an argument  
        \item Redirect stdin to a file  
        \item Save stdout in a file  
    \end{enumerate}

    \item The \texttt{-exec} option in \texttt{find} \_\_\_.  
    \begin{enumerate}[A)]
        \item Runs a command for each search result  
        \item Sorts the output of find  
        \item Moves files to a directory  
        \item Saves search results to a file  
    \end{enumerate}

    \item When using \texttt{xargs}, what option limits the number of lines used as arguments per execution?  
    \begin{enumerate}[A)]
        \item -print0  
        \item -L  
        \item -n  
        \item -exec  
    \end{enumerate}

    \item How can you ensure the \texttt{xargs} command processes paths with special characters?  
    \begin{enumerate}[A)]
        \item Use -n 1  
        \item Use -print0 and -0  
        \item Use tee  
        \item Use uniq  
    \end{enumerate}

    \item Which of the following combines input redirection and piping?  
    \begin{enumerate}[A)]
        \item \texttt{cat /proc/cpuinfo | wc}  
        \item \texttt{grep 'model name' </proc/cpuinfo | uniq}  
        \item \texttt{uniq | wc}  
        \item \texttt{tee}  
    \end{enumerate}

    \item What does the \texttt{-a} option in the \texttt{tee} command do?  
    \begin{enumerate}[A)]
        \item Appends output to a file  
        \item Prevents overwriting a file  
        \item Redirects output to stderr  
        \item Filters duplicate lines  
    \end{enumerate}

    \item The \texttt{find} option \texttt{-mindepth 2} instructs the command to \_\_\_.  
    \begin{enumerate}[A)]
        \item Process files only at a specified depth  
        \item Include hidden files in the results  
        \item Skip symbolic links in the search  
        \item Sort results by size  
    \end{enumerate}

    \item What is the purpose of using \texttt{xargs} with \texttt{-I} in a pipeline?  
    \begin{enumerate}[A)]
        \item Handle special characters  
        \item Use the input value at a specific position  
        \item Limit arguments per command  
        \item Run a command for each result  
    \end{enumerate}

\end{enumerate}

%-------------------------------------------------------
% Fill-in-the-Blank Questions (103.4)
%-------------------------------------------------------

%-------------------------------------------------------
% 103.5 Create, monitor and kill processes
%-------------------------------------------------------
\newpage
\section*{103.5 Create, monitor and kill processes}
\addcontentsline{toc}{section}{103.5 Create, monitor and kill processes}

\textbf{Reference to LPI Objectives:}
\begin{itemize}
    \item \textbf{LPIC-1 v5, Exam 101, Objective 103.5}
    \item \textbf{Weight:} 4
\end{itemize}

\subsection*{Key Knowledge Areas}
\begin{itemize}
    \item Running jobs in foreground and background
    \item Keeping processes running after logout (e.g., \texttt{nohup})
    \item Monitoring active processes (e.g., \texttt{ps}, \texttt{top}, \texttt{free}, \texttt{uptime}, \texttt{watch})
    \item Selecting/sorting processes (e.g., \texttt{ps} options, \texttt{top} interactions)
    \item Sending signals to processes (e.g., \texttt{kill}, \texttt{pkill}, \texttt{killall})
    \item Using terminal multiplexers (\texttt{screen}, \texttt{tmux})
\end{itemize}

\subsection*{Important Commands, Files, and Utilities}
\begin{itemize}
    \item \textbf{bg}, \textbf{fg}, \textbf{jobs}
    \item \textbf{kill}, \textbf{nohup}
    \item \textbf{ps}, \textbf{top}
    \item \textbf{free}, \textbf{uptime}, \textbf{watch}
    \item \textbf{pgrep}, \textbf{pkill}, \textbf{killall}
    \item \textbf{screen}, \textbf{tmux}
\end{itemize}

\section*{1. Foreground and Background Jobs}

\subsection*{1. Foreground Execution}
\begin{itemize}
    \item By default, commands run in foreground and occupy the terminal until complete.
    \item Example: \texttt{sleep 60} (blocks the terminal for 60 seconds).
\end{itemize}

\subsection*{2. Suspending a Foreground Job}
\begin{itemize}
    \item Press \textbf{Ctrl + Z} to suspend (stop) a running job.
\end{itemize}

\subsection*{3. Background Execution}
\begin{itemize}
    \item Append \textbf{\&} to run a process in the background immediately:
\begin{lstlisting}[language=bash]
sleep 60 &
\end{lstlisting}
    \item Or after suspending, use \texttt{bg} to continue a stopped job in the background.
\end{itemize}

\subsection*{4. jobs}
\begin{itemize}
    \item Lists active jobs associated with the current shell.
    \item Each job has a \textbf{job ID} (e.g., \texttt{[1]}).
\end{itemize}

\subsection*{5. fg}
\begin{itemize}
    \item Brings a background (or stopped) job to the foreground.
    \item Example: \texttt{fg \%1} → bring job \#1 forward.
\end{itemize}

\subsection*{6. Terminating a Job}
\begin{itemize}
    \item Use \texttt{kill \%1} to send a signal (default SIGTERM) to job \#1.
    \item If a job does not respond, try \texttt{kill -9 \%job\_number} (SIGKILL).
\end{itemize}

\section*{2. Detaching Jobs: \texttt{nohup}}

\subsection*{1. \texttt{nohup}}
\begin{itemize}
    \item Runs a command immune to \textbf{SIGHUP} (hangup signal).
    \item The process continues after you log out.
    \item Example:
\begin{lstlisting}[language=bash]
nohup ping localhost > ping.log 2>&1 &
\end{lstlisting}
    \item Output by default goes to \texttt{nohup.out} if not redirected.
\end{itemize}

\subsection*{2. Killing a nohup Process}
\begin{itemize}
    \item Identify the PID (e.g., \texttt{ps aux | grep ping} or \texttt{pgrep ping}) and then \texttt{kill <PID>}.
\end{itemize}

\section*{3. Monitoring Processes}

\subsection*{1. watch}
\begin{itemize}
    \item Periodically runs a command (default every 2 seconds).
    \item Example:
\begin{lstlisting}[language=bash]
watch uptime
watch -n 5 free
\end{lstlisting}
    \item Press \textbf{Ctrl + C} to exit.
\end{itemize}

\subsection*{2. free}
\begin{itemize}
    \item Displays memory usage (RAM, swap).
    \item Use \texttt{-m} or \texttt{-h} for human-readable format.
\end{itemize}

\subsection*{3. uptime}
\begin{itemize}
    \item Shows system up time, number of users, load averages.
\end{itemize}

\subsection*{4. ps}
\begin{itemize}
    \item Snapshots current processes.
    \item Common options:
    \begin{itemize}
        \item \texttt{ps aux} (show all processes in BSD style)
        \item \texttt{ps -ef} (show all processes in UNIX style)
        \item \texttt{ps -U <user>} or \texttt{ps --user <user>} (filter by user)
    \end{itemize}
    \item Fields: \textbf{PID}, \textbf{USER}, \textbf{\%CPU}, \textbf{\%MEM}, \textbf{VSZ} (virtual size), \textbf{RSS} (resident set size), \textbf{TTY}, \textbf{STAT}, \textbf{TIME}, \textbf{COMMAND}.
\end{itemize}

\subsection*{5. top}
\begin{itemize}
    \item Dynamic, real-time view of processes.
    \item Interactive keys (some important ones):
    \begin{itemize}
        \item \textbf{M} (sort by memory usage)
        \item \textbf{P} (sort by CPU usage)
        \item \textbf{k} (kill a process by PID)
        \item \textbf{r} (renice a process by PID)
        \item \textbf{q} (quit)
    \end{itemize}
\end{itemize}

\section*{4. Sending Signals to Processes}

\subsection*{1. kill}
\begin{itemize}
    \item Sends a signal to a process by PID or job spec.
    \item Default is \texttt{-TERM (SIGTERM, signal \#15)}.
    \item \texttt{-9} (SIGKILL) forcibly kills.
    \item Example:
\begin{lstlisting}[language=bash]
kill -TERM 1234
kill -9 1234
kill %1  # kills job #1
\end{lstlisting}
\end{itemize}

\subsection*{2. killall}
\begin{itemize}
    \item Kills all processes by command name.
    \item Example: \texttt{killall firefox}.
\end{itemize}

\subsection*{3. pkill}
\begin{itemize}
    \item Kills processes by name or matching pattern (use \texttt{pgrep} first to see matches).
    \item Example: \texttt{pkill -9 sleep}.
\end{itemize}
    
    \subsection*{4. Signal Reference}
    \begin{itemize}
        \item \texttt{kill -l} → list all signals.
        \item Common signals:
        \begin{itemize}
            \item SIGHUP (1), SIGINT (2), SIGQUIT (3), SIGTERM (15), SIGKILL (9), SIGSTOP (19), SIGCONT (18).
        \end{itemize}

    \end{itemize}
    
    \section*{5. Terminal Multiplexers: \texttt{screen} and \texttt{tmux}}

    \subsection*{5.1 \texttt{screen}}
    
    \subsubsection*{1. Basic Usage}
    \begin{itemize}
        \item Start: \texttt{screen} (creates a new session).
        \item Detach a session: \textbf{Ctrl + A} then \textbf{D}.
        \item Reattach: \texttt{screen -r [session\_id]} (or \texttt{screen -ls} to list sessions).
        \item Kill a session: \texttt{screen -S <session\_id> -X quit}.
    \end{itemize}
    
    \subsubsection*{2. Windows}
    \begin{itemize}
        \item Each window is like a separate terminal.
        \item Create new window: \textbf{Ctrl + A} then \textbf{C}.
        \item Move between windows: \textbf{Ctrl + A} then \textbf{N} (next) or \textbf{P} (previous).
        \item Rename a window: \textbf{Ctrl + A} then \textbf{Shift + A} (enter new name).
    \end{itemize}
    
    \subsubsection*{3. Regions (Split Screen)}
    \begin{itemize}
        \item Horizontal split: \textbf{Ctrl + A} then \textbf{S}.
        \item Vertical split: \textbf{Ctrl + A} then \texttt{|}.
        \item Move between splits: \textbf{Ctrl + A} then \textbf{Tab}.
    \end{itemize}
    
    \subsubsection*{4. Copy Mode}
    \begin{itemize}
        \item Enter copy mode: \textbf{Ctrl + A} then \texttt{[}.
        \item Use arrow keys to navigate, press \textbf{Space} to start selection, move, and \textbf{Space} again to end selection.
        \item Paste in a window: \textbf{Ctrl + A} then \texttt{]}.
    \end{itemize}
    
    \subsection*{5.2 \texttt{tmux}}
    
    \subsubsection*{1. Basic Usage}
    \begin{itemize}
        \item Start: \texttt{tmux} (creates a new session).
        \item Detach: \textbf{Ctrl + B} then \textbf{D}.
        \item List sessions: \texttt{tmux ls}.
        \item Attach: \texttt{tmux attach -t <session\_name>}.
        \item Kill session: \texttt{tmux kill-session -t <name>}.
    \end{itemize}
    
    \subsubsection*{2. Windows}
    \begin{itemize}
        \item New window: \textbf{Ctrl + B}, \textbf{C}.
        \item Switch windows: \textbf{Ctrl + B}, \textbf{N} (next) or \textbf{P} (previous).
        \item Rename: \textbf{Ctrl + B}, \texttt{,}.
    \end{itemize}
    
    \subsubsection*{3. Panes (Splits)}
    \begin{itemize}
        \item Horizontal split: \textbf{Ctrl + B}, \texttt{"}.
        \item Vertical split: \textbf{Ctrl + B}, \texttt{\%}.
        \item Switch panes: \textbf{Ctrl + B}, \textbf{Arrow keys}.
        \item Kill pane: \textbf{Ctrl + B}, \texttt{x}.
    \end{itemize}
    
    \subsubsection*{4. Copy Mode}
    \begin{itemize}
        \item Enter: \textbf{Ctrl + B}, \texttt{[}.
        \item Move around, press \textbf{Space} to start selection, arrow, press \textbf{Space} again to finish.
        \item Paste: \textbf{Ctrl + B}, \texttt{]}.
    \end{itemize}
    

    \section*{Workbook Exercises}

    \begin{enumerate}
        \item \textbf{Foreground/Background Jobs}
        \begin{itemize}
            \item Run \texttt{sleep 60} in the foreground, suspend with \textbf{Ctrl + Z}.
            \item Check \texttt{jobs}.
            \item Send it to background with \texttt{bg}.
            \item Bring it back to foreground with \texttt{fg}.
            \item Kill it with \texttt{kill \%1}.
        \end{itemize}
    
        \item \textbf{nohup}
        \begin{itemize}
            \item Run \texttt{nohup ping localhost > ping.log 2>\&1 \&}.
            \item Log out and back in.
            \item Check the process with \texttt{ps} or \texttt{pgrep}.
            \item Kill it with \texttt{kill <PID>}.
        \end{itemize}
    
        \item \textbf{Monitoring System Load}
        \begin{itemize}
            \item Run \texttt{watch uptime} or \texttt{watch -n 5 free}.
            \item Exit with \textbf{Ctrl + C}.
            \item Compare to \texttt{top} output.
        \end{itemize}
    
        \item \textbf{Killing Processes}
        \begin{itemize}
            \item Start a dummy process like \texttt{sleep 9999 \&}.
            \item Find the PID via \texttt{ps}, \texttt{pgrep sleep}.
            \item Kill it with \texttt{kill <PID>}.
            \item Confirm it’s gone (\texttt{ps} or \texttt{jobs}).
        \end{itemize}
    
        \item \textbf{Using \texttt{top}}
        \begin{itemize}
            \item Invoke \texttt{top}.
            \item Press \textbf{k} to kill a process (e.g., search a small CPU process).
            \item Sort by memory usage with \textbf{M}, or CPU usage with \textbf{P}.
            \item Quit with \textbf{q}.
        \end{itemize}
    
        \item \textbf{\texttt{screen}}
        \begin{itemize}
            \item Start a session (\texttt{screen}).
            \item Create multiple windows (\textbf{Ctrl + A}, \textbf{C}).
            \item Detach (\textbf{Ctrl + A}, \textbf{D}) and reattach (\texttt{screen -r}).
            \item Kill a session (\texttt{screen -S <session\_id> -X quit}).
        \end{itemize}
    
        \item \textbf{\texttt{tmux}}
        \begin{itemize}
            \item Start a session (\texttt{tmux}).
            \item Create new windows (\textbf{Ctrl + B}, \textbf{C}).
            \item Split horizontally (\textbf{Ctrl + B}, \texttt{"}) or vertically (\textbf{Ctrl + B}, \texttt{\%}).
            \item Detach (\textbf{Ctrl + B}, \textbf{D}) and reattach (\texttt{tmux a}).
            \item Kill with \texttt{tmux kill-session -t <name>}.
        \end{itemize}
    \end{enumerate}
    
    \section*{Summary}
    \begin{itemize}
        \item Background jobs (\texttt{bg}) and foreground (\texttt{fg}) are managed via \textbf{jobs}.
        \item \texttt{nohup} allows processes to keep running after logout.
        \item Process monitoring uses commands like \texttt{ps}, \texttt{top}, \texttt{free}, \texttt{uptime}, \texttt{watch}.
        \item Signals are sent with \texttt{kill} (by PID/job), \texttt{killall}, \texttt{pkill}, each defaulting to \textbf{SIGTERM} if unspecified.
        \item Terminal multiplexers \texttt{screen} and \texttt{tmux} provide multi-window, multi-pane sessions that can detach and persist in the background, invaluable for system administration over remote connections.
    \end{itemize}



%-------------------------------------------------------
% Multiple-Choice Questions for (103.5)
%-------------------------------------------------------
\newpage
\section*{Multiple-Choice Questions for 103.5}
\begin{enumerate}[1.]

    \item What command is used to run a process in the background directly while starting it?  
    \begin{enumerate}[A)]
        \item fg  
        \item \&  
        \item nohup  
        \item jobs  
    \end{enumerate}

    \item Which command is used to send a job to the foreground?  
    \begin{enumerate}[A)]
        \item fg  
        \item bg  
        \item kill  
        \item nohup  
    \end{enumerate}

    \item Which signal is sent by default if no specific signal is mentioned while using the \texttt{kill} command?  
    \begin{enumerate}[A)]
        \item SIGSTOP  
        \item SIGHUP  
        \item SIGTERM  
        \item SIGKILL  
    \end{enumerate}

    \item What does the \texttt{-n} option in the \texttt{watch} command do?  
    \begin{enumerate}[A)]
        \item Changes the update interval  
        \item Displays the process ID  
        \item Lists only running jobs  
        \item Lists only stopped jobs  
    \end{enumerate}

    \item What key is pressed in \texttt{top} to sort the output by memory usage?  
    \begin{enumerate}[A)]
        \item T  
        \item N  
        \item R  
        \item M  
    \end{enumerate}

    \item What is the purpose of the \texttt{nohup} command?  
    \begin{enumerate}[A)]
        \item To bring a job to the foreground  
        \item To monitor active processes  
        \item To detach a job so it continues running after logout  
        \item To kill a specific process by PID  
    \end{enumerate}

    \item Which command is used to kill all instances of a specific process by name?  
    \begin{enumerate}[A)]
        \item killall  
        \item pkill  
        \item kill  
        \item nohup  
    \end{enumerate}

    \item What is the role of the \texttt{jobs} command in Linux?  
    \begin{enumerate}[A)]
        \item To display all running processes  
        \item To list active jobs started by the current shell  
        \item To terminate jobs  
        \item To monitor memory usage  
    \end{enumerate}

    \item What does the \texttt{free} command display?  
    \begin{enumerate}[A)]
        \item Load averages  
        \item Memory usage statistics  
        \item Running jobs  
        \item Network statistics  
    \end{enumerate}

    \item Which option in the \texttt{ps} command shows all processes regardless of terminal attachment?  
    \begin{enumerate}[A)]
        \item -u  
        \item x  
        \item a  
        \item aux  
    \end{enumerate}

    \item What is a key feature of terminal multiplexers like GNU Screen and tmux?  
    \begin{enumerate}[A)]
        \item They replace terminal emulators entirely.  
        \item They are used exclusively for SSH connections.  
        \item They allow multiple terminal sessions within a single terminal window.  
        \item They are required to run graphical applications.  
    \end{enumerate}

    \item Which command in tmux splits the current window into vertical panes?  
    \begin{enumerate}[A)]
        \item Ctrl + b - "  
        \item Ctrl + b - q  
        \item Ctrl + b - \%  
        \item Ctrl + b - |  
    \end{enumerate}

    \item What is the default command prefix for GNU Screen?  
    \begin{enumerate}[A)]
        \item Ctrl + a  
        \item Ctrl + b  
        \item Ctrl + c  
        \item Ctrl + d  
    \end{enumerate}

    \item In tmux, which command creates a new session with a specific name?  
    \begin{enumerate}[A)]
        \item tmux new -n NAME  
        \item tmux create NAME  
        \item tmux new -s NAME  
        \item tmux start -n NAME  
    \end{enumerate}

    \item Which of the following commands terminates the current pane in tmux?  
    \begin{enumerate}[A)]
        \item Ctrl + b - \&  
        \item Ctrl + b - X  
        \item Ctrl + b - q  
        \item Ctrl + b - x  
    \end{enumerate}

    \item How do you reattach to a detached session in tmux?  
    \begin{enumerate}[A)]
        \item tmux attach-session  
        \item tmux attach  
        \item tmux -r  
        \item tmux reconnect  
    \end{enumerate}

    \item What is the main difference between screen "regions" and tmux "panes"?  
    \begin{enumerate}[A)]
        \item Panes are more customizable than regions.  
        \item Panes are independent pseudo-terminals, while regions are not.  
        \item Panes are temporary, but regions persist between sessions.  
        \item Panes require a session to be detached to function.  
    \end{enumerate}

    \item In GNU Screen, which command prefix is used to rename a window?  
    \begin{enumerate}[A)]
        \item Ctrl + a - r  
        \item Ctrl + a - A  
        \item Ctrl + a - n  
        \item Ctrl + a - "  
    \end{enumerate}

    \item Which tmux key combination detaches a session?  
    \begin{enumerate}[A)]
        \item Ctrl + b - a  
        \item Ctrl + b - d  
        \item Ctrl + b - t  
        \item Ctrl + b - m  
    \end{enumerate}

    \item Which configuration file can be used to customize tmux on a per-user basis?  
    \begin{enumerate}[A)]
        \item /etc/tmux.conf  
        \item ~/.tmux.conf  
        \item /usr/share/tmux/tmux.conf  
        \item ~/.tmux.config  
    \end{enumerate}

\end{enumerate}

%-------------------------------------------------------
% Fill-in-the-Blank Questions (103.5)
%-------------------------------------------------------

%-------------------------------------------------------
% 103.6 Modify process execution priorities
%-------------------------------------------------------
\newpage
\section*{103.6 Modify process execution priorities}
\addcontentsline{toc}{section}{103.6 Modify process execution priorities}


\textbf{Reference to LPI Objectives:}
\begin{itemize}
    \item \textbf{LPIC-1 v5, Exam 101, Objective 103.6}
    \item \textbf{Weight:} 2
\end{itemize}

\subsection*{Key Knowledge Areas}
\begin{itemize}
    \item Knowing the default priority (niceness) of newly created processes.
    \item Running a program with a higher or lower priority than default.
    \item Changing the priority of running processes.
\end{itemize}

\subsection*{Important Terms and Utilities}
\begin{itemize}
    \item \textbf{nice}
    \item \textbf{ps}
    \item \textbf{renice}
    \item \textbf{top}
\end{itemize}

\section*{1. Understanding Linux Process Scheduling}

\begin{itemize}
    \item \textbf{Multi-tasking (multi-processing) Systems:} Multiple processes “share” CPU time.
    \item \textbf{Preemptive Scheduling:} The kernel can forcibly switch the CPU from one process to another (even if the current process never calls an I/O or system call).
    \item \textbf{Normal vs. Real-Time Scheduling:} Real-time processes have higher priority than normal processes. We generally only adjust normal processes using “nice” values.
\end{itemize}

\subsection*{Static Priorities \& Niceness}
\begin{itemize}
    \item \textbf{Static Priorities:}
    \begin{itemize}
        \item Range \texttt{0–99} for real-time processes.
        \item Range \texttt{100–139} for normal processes (kernel internal representation).
    \end{itemize}
    \item \textbf{Nice Values:}
    \begin{itemize}
        \item Range \texttt{-20} (highest priority, "least nice") to \texttt{19} (lowest priority, "most nice").
        \item Default nice value is \texttt{0} for normal processes.
        \item Only \texttt{root} can set negative niceness.
    \end{itemize}
\end{itemize}

\section*{2. Viewing Process Priorities}

\subsection*{1. ps}
\begin{itemize}
    \item Example: \texttt{ps -el} or \texttt{ps -Al}
    \begin{itemize}
        \item \textbf{PRI:} process priority (internal to kernel, typically 80 = normal + offset).
        \item \textbf{NI:} nice value (\texttt{-20} to \texttt{19}).
    \end{itemize}
\end{itemize}

\subsection*{2. top}
\begin{itemize}
    \item Interactive system monitor, displays \textbf{PR} (priority) and \textbf{NI} (nice).
    \item Priority shown in \texttt{top} for normal processes is \texttt{PR - 100}, typically ranging \texttt{0–39}.
    \item Press \textbf{r} in \texttt{top} to renice a process.
\end{itemize}

\section*{3. Running a Program with Custom Priority}

\subsection*{nice}
\begin{itemize}
    \item \textbf{Syntax:}
\begin{lstlisting}[language=bash]
nice -n <nice_value> command
\end{lstlisting}
    \item Example:
\begin{lstlisting}[language=bash]
nice -n 10 tar czf backup.tar.gz /home
\end{lstlisting}
\begin{itemize}
    \item If \texttt{-n} is omitted, default niceness becomes \texttt{10}.
\end{itemize}
    
    \item A negative nice value (e.g., \texttt{-5}) raises priority (requires root privileges).
\end{itemize}

\section*{4. Changing Priority of Running Processes}

\subsection*{renice}
\begin{itemize}
    \item \textbf{Syntax:}
\begin{lstlisting}[language=bash]
renice <new_nice_value> [options] <target>
\end{lstlisting}
    \item Common options:
    \begin{itemize}
        \item \texttt{-p <PID>}: target by process ID.
        \item \texttt{-u <username>}: target all processes by user.
        \item \texttt{-g <groupname>}: target all processes by group.
    \end{itemize}
    \item Example (by PID):
\begin{lstlisting}[language=bash]
renice +5 -p 1234
\end{lstlisting}
    \item Example (by user):
\begin{lstlisting}[language=bash]
renice +10 -u alice
\end{lstlisting}
    \item Alternatively in \texttt{top}, press \textbf{r} to renice a running process.
\end{itemize}

\section*{Exercises}

\begin{enumerate}
    \item \textbf{Check Default Niceness}
    \begin{itemize}
        \item Open a terminal and run:
\begin{lstlisting}[language=bash]
sleep 300 &
ps -el | grep sleep
\end{lstlisting}
        \item Note \textbf{NI} value (should be \texttt{0}).
    \end{itemize}

    \item \textbf{Start a Process with Custom Niceness}
    \begin{itemize}
        \item Example:
\begin{lstlisting}[language=bash]
nice -n 5 gzip largefile
ps -el | grep gzip
\end{lstlisting}
        \item Observe \textbf{NI} column in \texttt{ps}.
    \end{itemize}

    \item \textbf{Renice a Running Process}
    \begin{itemize}
        \item Find a CPU-bound process (e.g., \texttt{md5sum /dev/zero}), get PID via \texttt{pgrep md5sum}.
        \item Increase niceness:
\begin{lstlisting}[language=bash]
renice +10 -p <PID>
\end{lstlisting}
        \item Check new niceness in \texttt{top} or \texttt{ps}.
    \end{itemize}

    \item \textbf{Interactive with top}
    \begin{itemize}
        \item Run \texttt{top}.
        \item Press \textbf{r}, enter the PID, then a new nice value.
        \item Verify changes in the NI column.
    \end{itemize}
\end{enumerate}

\section*{Summary}
\begin{itemize}
    \item \texttt{nice} sets a process’s niceness at launch, default is \texttt{0}, range \texttt{-20} to \texttt{19}.
    \item \texttt{renice} changes niceness of an already running process.
    \item \texttt{top} and \texttt{ps} display priorities; niceness is shown in \textbf{NI} columns.
    \item Higher priority → lower niceness (negative). Lower priority → higher niceness (positive).
    \item Mastering process priorities optimizes CPU usage for critical or resource-intensive tasks.
\end{itemize}



%-------------------------------------------------------
% Multiple-Choice Questions for (103.6)
%-------------------------------------------------------
\newpage
\section*{Multiple-Choice Questions for 103.6}

\begin{enumerate}[1.]

    \item What is the default nice value assigned to a process in Linux?  
    \begin{enumerate}[A)]
        \item -20  
        \item 0  
        \item 10  
        \item 19  
    \end{enumerate}

    \item How are real-time processes prioritized compared to normal processes in Linux?  
    \begin{enumerate}[A)]
        \item They are ignored.  
        \item They have the same priority as normal processes.  
        \item They have higher priority than normal processes.  
        \item They have lower priority than normal processes.  
    \end{enumerate}

    \item What command is used to check the static priority of a process?  
    \begin{enumerate}[A)]
        \item top  
        \item renice  
        \item ps  
        \item grep  
    \end{enumerate}

    \item In the ps command output, what column represents the priority?  
    \begin{enumerate}[A)]
        \item ADDR  
        \item CMD  
        \item PRI  
        \item NI  
    \end{enumerate}

    \item Which of the following is a valid nice value range?  
    \begin{enumerate}[A)]
        \item -30 to 30  
        \item -20 to 20  
        \item 0 to 100  
        \item -20 to 19  
    \end{enumerate}

    \item What is the default nice value assigned by the nice command if no option is specified?  
    \begin{enumerate}[A)]
        \item -20  
        \item 0  
        \item 19  
        \item 10  
    \end{enumerate}

    \item How can the nice value of an already running process be changed?  
    \begin{enumerate}[A)]
        \item By using the nice command.  
        \item By restarting the process.  
        \item By using the kill command.  
        \item By using the renice command.  
    \end{enumerate}

    \item What happens when a process with a lower nice value is introduced?  
    \begin{enumerate}[A)]
        \item It runs with a higher priority.  
        \item It waits for other processes to complete.  
        \item It gets executed last.  
        \item Its priority remains unchanged.  
    \end{enumerate}

    \item Which scheduling priority range is assigned to normal processes?  
    \begin{enumerate}[A)]
        \item 0 to 99  
        \item 0 to 120  
        \item 100 to 139  
        \item -20 to 19  
    \end{enumerate}

    \item What command can display dynamic priority changes continuously?  
    \begin{enumerate}[A)]
        \item ps  
        \item top  
        \item grep  
        \item nice  
    \end{enumerate}

    \item Which key is used in the top command to modify a process's priority?  
    \begin{enumerate}[A)]
        \item r  
        \item n  
        \item p  
        \item q  
    \end{enumerate}

    \item What does a nice value of -10 imply?  
    \begin{enumerate}[A)]
        \item Higher priority than default.  
        \item Default priority.  
        \item Lower priority than default.  
        \item The process cannot run.  
    \end{enumerate}

    \item What is the function of the -p option in the renice command?  
    \begin{enumerate}[A)]
        \item Modify priority of all processes.  
        \item Modify priority of all processes by a user.  
        \item Specify a PID for priority modification.  
        \item View current priorities.  
    \end{enumerate}

    \item What priority level corresponds to a real-time process with a static priority of 5?  
    \begin{enumerate}[A)]
        \item 135  
        \item 5  
        \item 40  
        \item -5  
    \end{enumerate}

    \item Which process attribute dictates execution priority in a normal scheduling policy?  
    \begin{enumerate}[A)]
        \item Static priority  
        \item Nice value  
        \item System calls  
        \item File descriptors  
    \end{enumerate}

    \item What command shows the static priority of the systemd process?  
    \begin{enumerate}[A)]
        \item ps -e  
        \item top  
        \item grep \^prio /proc/1/sched  
        \item nice -n 0  
    \end{enumerate}

    \item In the ps command, which column displays the nice value?  
    \begin{enumerate}[A)]
        \item CMD  
        \item PRI  
        \item NI  
        \item SZ  
    \end{enumerate}

    \item How does renice modify priorities for a user’s processes?  
    \begin{enumerate}[A)]
        \item -n  
        \item -p  
        \item -u  
        \item -g  
    \end{enumerate}

    \item What term describes Linux’s ability to switch processes without waiting for system calls?  
    \begin{enumerate}[A)]
        \item Nice value  
        \item Priority inversion  
        \item Preemption  
        \item Static scheduling  
    \end{enumerate}

    \item What does a nice value of 19 imply?  
    \begin{enumerate}[A)]
        \item High priority.  
        \item Default priority.  
        \item Low priority.  
        \item Real-time scheduling.  
    \end{enumerate}

\end{enumerate}
%-------------------------------------------------------
% Fill-in-the-Blank Questions (103.6)
%-------------------------------------------------------
\newpage
\section*{Fill-in-the-Blank Questions for 103.6}

\begin{enumerate}[1.]
    \item B
    \item C
    \item C
    \item C
    \item D
    \item B
    \item D
    \item A
    \item C
    \item B
    \item A
    \item A
    \item C
    \item D
    \item B
    \item C
    \item C
    \item C
    \item A
    \item C
\end{enumerate}
%-------------------------------------------------------
% 103.7 Search text files using regular expressions
%-------------------------------------------------------
\newpage
\section*{103.7 Search text files using regular expressions}
\addcontentsline{toc}{section}{103.7 Search text files using regular expressions}


\textbf{Reference to LPI Objectives:}
\begin{itemize}
    \item \textbf{LPIC-1 v5, Exam 101, Objective 103.7}
    \item \textbf{Weight:} 3
\end{itemize}

\subsection*{Key Knowledge Areas}
\begin{itemize}
    \item Creating simple regular expressions with various notational elements.
    \item Understanding differences between basic (BRE) and extended (ERE) regular expressions.
    \item Recognizing special characters, character classes, quantifiers, and anchors.
    \item Using regular expressions to search file systems or file content.
    \item Deleting, changing, and substituting text with regex tools.
\end{itemize}

\subsection*{Important Commands and Utilities}
\begin{itemize}
    \item \textbf{grep}, \textbf{egrep} (or \texttt{grep -E}), \textbf{fgrep} (or \texttt{grep -F})
    \item \textbf{sed}
    \item \texttt{regex(7)} (man page)
    \item \texttt{find} with \texttt{-regex}, \texttt{-iregex}
\end{itemize}

\section*{1. Regular Expression Basics}

\subsection*{1. Atoms}
\begin{itemize}
    \item \textbf{Ordinary characters} match themselves (e.g., \texttt{b} matches “b” literally).
    \item \textbf{. (dot)} matches any single character.
    \item \textbf{\texttt{\^}} anchors to the start of a line.
    \item \textbf{\$} anchors to the end of a line.
    \item \textbf{Brackets [ ]} define a \textit{bracket expression} (single atom that matches any character inside).
        \begin{itemize}
            \item Example: \texttt{[0-9]} matches any digit from 0 to 9.
            \item A caret \texttt{\^} inside the bracket means negation: \texttt{[\^ \ a-z]} matches non-lowercase letters.
        \end{itemize}
\end{itemize}

\subsection*{2. Character Classes (in bracket expressions)}
\begin{itemize}
    \item \texttt{[[:digit:]]} for digits, \texttt{[[:alpha:]]} for letters, \texttt{[[:alnum:]]} for alphanumeric, etc.
    \item Only valid \textbf{inside} bracket expressions like \texttt{[[:digit:]]}.
\end{itemize}

\subsection*{3. Anchors}
\begin{itemize}
    \item \texttt{\^} (start of line), \texttt{\$} (end of line).
    \item Example: \texttt{\^\ abc\$} matches lines that are \textit{exactly} \texttt{abc}.
\end{itemize}

\subsection*{4. Basic (BRE) vs. Extended (ERE)}
\begin{itemize}
    \item \textbf{BRE:} \texttt{+}, \texttt{?}, and \texttt{\{\}} usually need backslashes (\texttt{\textbackslash +}, \texttt{\textbackslash ?}, \texttt{\textbackslash \{i,j\}}) for special meaning.
    \item \textbf{ERE:} \texttt{+}, \texttt{?}, \texttt{\{\}} directly recognized as quantifiers without backslashes.
    \item Tools:
        \begin{itemize}
            \item \texttt{grep} uses BRE by default.
            \item \texttt{egrep} (or \texttt{grep -E}) uses ERE.
        \end{itemize}
\end{itemize}

\subsection*{5. Quantifiers}
\begin{itemize}
    \item \texttt{*}: zero or more occurrences of preceding atom.
    \item \texttt{+}: one or more occurrences (ERE or \texttt{\textbackslash +} in BRE).
    \item \texttt{?}: zero or one occurrence (ERE or \texttt{\textbackslash ?} in BRE).
    \item \texttt{\{i\}}, \texttt{\{i,\}}, \texttt{\{i,j\}}: exact or range bounds (with backslash in BRE).
\end{itemize}

\subsection*{6. Branches and Backreferences}
\begin{itemize}
    \item \textbf{Branching:} \texttt{|} (ERE) or \texttt{\textbackslash |} (BRE) for “OR” logic.
    \item \textbf{Backreferences:} \texttt{( )} group subexpressions (ERE) or \texttt{\textbackslash( \textbackslash)} (BRE), use \texttt{\textbackslash 1}, \texttt{\textbackslash 2} to refer to matched groups.
\end{itemize}

\section*{2. Searching with \texttt{grep} / \texttt{egrep} / \texttt{fgrep}}

\subsection*{1. \texttt{grep}}
\begin{itemize}
    \item Default uses basic RE.
    \item Common options:
        \begin{itemize}
            \item \texttt{-i} ignore case
            \item \texttt{-v} invert match
            \item \texttt{-n} show line numbers
            \item \texttt{-r} recursive
            \item \texttt{-w} match whole words
            \item \texttt{-c} count matches
            \item \texttt{-H} show filename
        \end{itemize}
    \item Example:
\begin{lstlisting}[language=bash]
grep "^error" /var/log/syslog
\end{lstlisting}
\end{itemize}

\subsection*{2. \texttt{egrep} (or \texttt{grep -E})}
\begin{itemize}
    \item Uses extended RE.
    \item Example:
\begin{lstlisting}[language=bash]
egrep "colou?r|color" file.txt
\end{lstlisting}
    \begin{itemize}
        \item Matches \texttt{colour} or \texttt{color}.
    \end{itemize}
\end{itemize}

\subsection*{3. \texttt{fgrep} (or \texttt{grep -F})}
\begin{itemize}
    \item Interprets pattern literally (no special chars).
    \item Example:
\begin{lstlisting}[language=bash]
fgrep "[a-z]*" file.txt
\end{lstlisting}
    \begin{itemize}
        \item Finds lines containing \texttt{[a-z]*} literally.
    \end{itemize}
\end{itemize}

\subsection*{4. Examples}
\begin{itemize}
    \item \texttt{grep -i "warning" /var/log/messages} → case-insensitive match “warning.”
    \item \texttt{grep -c "\^root" /etc/passwd} → count lines starting with “root.”
    \item \texttt{grep -H -r "error" /var/log/} → recursively find “error” in \texttt{/var/log/}, show filenames.
\end{itemize}


\section*{3. Substitutions and Text Editing with \texttt{sed}}

\subsection*{1. Basics}
\begin{itemize}
    \item Non-interactive stream editor, applies commands to each line.
    \item Syntax:
\begin{lstlisting}[language=bash]
sed [options] -e COMMAND file
# or sed [options] -f SCRIPT file
\end{lstlisting}
\end{itemize}


\subsection*{2. Common Sed Commands}
\begin{itemize}
    \item \texttt{d}: delete the matching line(s).
    \item \texttt{c <text>}: replace the line with \texttt{<text>}.
    \item \texttt{s/FIND/REPLACE/[flags]}: substitute \texttt{FIND} with \texttt{REPLACE}.
        \begin{itemize}
            \item \texttt{g} flag: replace all occurrences in that line (global).
            \item Example:
\begin{lstlisting}[language=bash]
sed 's/foo/bar/g' file.txt
\end{lstlisting}
        \end{itemize}
    \item \texttt{p}: print the matching line.
    \item \texttt{a <text>}: append text after the matching line.
    \item \texttt{r <file>}: read content of \texttt{<file>} after matching line.
    \item Lines or ranges can be specified by line number or regex.
        \begin{itemize}
            \item Example: \texttt{1d} deletes line 1, \texttt{2,5d} deletes lines 2 to 5, \texttt{/regex/d} deletes lines matching \texttt{regex}.
        \end{itemize}
\end{itemize}

\subsection*{3. Combining}
\begin{itemize}
    \item Multiple commands can be separated by \texttt{;}:
\begin{lstlisting}[language=bash]
sed -e '1d; /test/d' file.txt
\end{lstlisting}
\begin{itemize}
    \item Deletes line 1, then any line containing “test.”
\end{itemize}
\end{itemize}


\subsection*{4. Examples Using find + regex}
\begin{enumerate}
    \item \textbf{find -regex}
    \begin{itemize}
        \item Basic RE by default.
        \item \texttt{-regextype posix-extended} or \texttt{-regextype egrep} for extended RE.
        \item Example:
\begin{lstlisting}[language=bash]
find /usr/share/fonts -regextype egrep -regex ".*\.(ttf|otf|woff)$"
\end{lstlisting}
    \end{itemize}

    \item \textbf{find -iregex (case-insensitive)}
    \begin{itemize}
        \item Example:
\begin{lstlisting}[language=bash]
find /home/user -iregex ".*\.pdf"
\end{lstlisting}
    \end{itemize}
\end{enumerate}

\subsection*{5. Combining Tools for Advanced Searches}
\begin{itemize}
    \item \textbf{Chaining:}
    \begin{itemize}
        \item Pipe outputs through \texttt{grep}, \texttt{sed}, \texttt{uniq}, etc.
    \end{itemize}

    \item \textbf{Example: Searching logs, removing duplicates}
\begin{lstlisting}[language=bash]
lastb -n 10 -d -a --time-format notime \
    | grep -v '[0-9]$' \
    | sed 's/.* //' \
    | sort | uniq
\end{lstlisting}
    \item Explanation:
    \begin{itemize}
        \item \texttt{lastb} shows bad login attempts, \texttt{-a} puts host at the end of the line, \texttt{-n 10} shows last 10.
        \item \texttt{grep -v '[0-9]\$'} filters out lines ending in digits (assuming those are raw IPs).
        \item \texttt{sed 's/.* //'} keeps only the substring after the last space (the hostname).
        \item Then \texttt{sort | uniq} eliminates duplicates.
    \end{itemize}
\end{itemize}


\subsection*{Exercises}
\begin{enumerate}
    \item \textbf{Basic vs. Extended}
    \begin{itemize}
        \item Create a text file with lines containing special chars (\texttt{+}, \texttt{?}, \texttt{()}, etc.).
        \item Try matching them with \texttt{grep "..."} (basic RE) and \texttt{egrep "..."} (extended).
    \end{itemize}

    \item \textbf{Anchors and Dot}
    \begin{itemize}
        \item Use \texttt{grep "\^Hello" file.txt} to find lines starting with “Hello.”
        \item \texttt{grep "world\$" file.txt} to find lines ending with “world.”
        \item \texttt{grep "w.rld" file.txt} (dot matches any char).
    \end{itemize}

    \item \textbf{Bracket Expressions}
    \begin{itemize}
        \item \texttt{grep "[[:digit:]]" file.txt} → lines with digits.
        \item \texttt{grep "[A-Z]" file.txt} → uppercase letters.
    \end{itemize}

    \item \textbf{sed Substitution}
    \begin{itemize}
        \item \texttt{sed 's/foo/bar/g' file.txt} → replace “foo” with “bar.”
        \item \texttt{sed '2,4d' file.txt} → delete lines 2 to 4.
        \item \texttt{/old/d} → delete lines containing “old.”
    \end{itemize}

    \item \textbf{Combining}
    \begin{itemize}
        \item \texttt{grep -i "error" /var/log/syslog | sed -n '/warning/p'}  
        \begin{itemize}
        \item Finds lines matching “error” and then prints only lines also having “warning.”
    \end{itemize}
\end{itemize}

    \item \textbf{find with -regex}
    \begin{itemize}
        \item \texttt{find / -regextype posix-extended -regex "\texttt{.*\textbackslash
        .conf}"} → all \texttt{.conf} files.
    \end{itemize}
\end{enumerate}

\section*{Summary}
\begin{itemize}
    \item \textbf{Regular expressions} define a flexible method for searching (and sometimes replacing) text patterns.
    \item Basic RE often requires backslashes for special operators like \texttt{+}, \texttt{?}, \texttt{\{\}}, while extended RE does not.
    \item Core features include \textbf{atoms}, \textbf{character classes}, \textbf{anchors}, \textbf{quantifiers}.
    \item \textbf{Tools:}
    \begin{itemize}
        \item \texttt{grep/egrep/fgrep} → quick searches in lines.
        \item \texttt{sed} → stream editing and substitution.
        \item \texttt{find} with \texttt{-regex} → RE-based filtering of file paths.
    \end{itemize}
    \item Combining these tools (piping, advanced regex) enables powerful text processing and filtering workflows.
\end{itemize}


%-------------------------------------------------------
% Multiple-Choice Questions for (103.7)
%-------------------------------------------------------

%-------------------------------------------------------
% Fill-in-the-Blank Questions (103.7)
%-------------------------------------------------------

%-------------------------------------------------------
% 103.8 Basic file editing
%-------------------------------------------------------
\newpage
\section*{103.8 Basic file editing}
\addcontentsline{toc}{section}{103.8 Basic file editing}

\textbf{Reference to LPI Objectives:}
\begin{itemize}
    \item \textbf{LPIC-1 v5, Exam 101, Objective 103.8}
    \item \textbf{Weight:} 3
\end{itemize}

\subsection*{Key Knowledge Areas}
\begin{itemize}
    \item Navigating a document using \textbf{vi} (or \textbf{vim})
    \item Understanding vi \textbf{modes} and using them effectively
    \item Inserting, editing, deleting, copying, and finding text in vi
    \item Awareness of alternative editors (Emacs, nano, etc.)
    \item Setting or changing the \textbf{standard editor} (EDITOR or VISUAL environment variable)
\end{itemize}

\subsection*{Important Terms and Utilities}
\begin{itemize}
    \item \textbf{vi} (or \textbf{vim})
    \item Search commands (\texttt{/}, \texttt{?})
    \item Navigation keys (\texttt{h}, \texttt{j}, \texttt{k}, \texttt{l})
    \item Insert modes (\texttt{i}, \texttt{o}, \texttt{a})
    \item Deletion/copying (\texttt{d}, \texttt{p}, \texttt{y}, \texttt{dd}, \texttt{yy})
    \item Exiting and saving (\texttt{ZZ}, \texttt{:w!}, \texttt{:q!})
    \item \textbf{EDITOR} environment variable
\end{itemize}

\section*{1. Introducing vi}

\textbf{vi} is a modal text editor commonly pre-installed on Unix-like systems. Two key modes to remember:

\begin{enumerate}
    \item \textbf{Normal (Command) Mode}
    \begin{itemize}
        \item vi \textbf{starts} in this mode.
        \item Keystrokes perform navigation or editing commands.
        \item Press \texttt{i} (insert) or \texttt{a} (append) to switch to Insert mode.
    \end{itemize}

    \item \textbf{Insert Mode}
    \begin{itemize}
        \item Text typed appears in the document.
        \item Press \texttt{Esc} to return to Normal mode.
    \end{itemize}
\end{enumerate}

\subsection*{Starting vi}
\begin{lstlisting}[language=bash]
vi <filename>
\end{lstlisting}
\begin{itemize}
    \item \texttt{vi +<line> <file>} → jump to a line on opening.
    \item \texttt{vim} is a more feature-rich version, fully backward-compatible with vi.
\end{itemize}

\section*{2. Basic vi Navigation and Commands}

In \textbf{Normal mode}:

\begin{table}[h!]
\begin{tabular}{|l|l|}
\hline
\textbf{Key} & \textbf{Action} \\
\hline
\texttt{h, j, k, l} & Left, down, up, right (move cursor) \\
\texttt{0, \$} & Move to start/end of current line \\
\texttt{G} & Go to end of file (use \texttt{\#G} to go to line \#) \\
\texttt{/pattern} & Forward search for \texttt{pattern} (press \texttt{n} to find next) \\
\texttt{?pattern} & Backward search (press \texttt{N} for next backward match) \\
\texttt{u} & Undo last action \\
\texttt{Ctrl+R} & Redo \\
\texttt{:} & Enter \textbf{Colon Command mode} (e.g., \texttt{:w}, \texttt{:q!}) \\
\hline
\end{tabular}
\end{table}

\subsection*{Basic Insert Commands}
\begin{table}[h!]
\begin{tabular}{|l|l|}
\hline
\textbf{Key} & \textbf{Action} \\
\hline
\texttt{i} & Insert at cursor \\
\texttt{a} & Append after cursor \\
\texttt{o} & Open new line below cursor and enter insert mode \\
\texttt{O} & Open new line above cursor \\
\texttt{Esc} & Leave insert mode, back to normal mode \\
\hline
\end{tabular}
\end{table}

\section*{3. Editing Text in vi}

\subsection*{Deleting and Changing}
\begin{table}[h!]
\begin{tabular}{|l|l|}
\hline
\textbf{Key} & \textbf{Action} \\
\hline
\texttt{x} & Delete character under cursor \\
\texttt{dw} & Delete word (from cursor to next word boundary) \\
\texttt{dd} & Delete (cut) entire current line \\
\texttt{d\$} & Delete from cursor to end of line \\
\texttt{cc} & Change entire line (delete and start insert mode) \\
\texttt{c3w} & Change next 3 words (example of “count + command”) \\
\hline
\end{tabular}
\end{table}

\subsection*{Copying and Pasting}
\begin{table}[h!]
\begin{tabular}{|l|l|}
\hline
\textbf{Key} & \textbf{Action} \\
\hline
\texttt{yy} & Yank (copy) current line \\
\texttt{y\$} & Yank from cursor to end of line \\
\texttt{p} & Paste after cursor or below line \\
\texttt{P} & Paste before cursor or above line \\
\hline
\end{tabular}
\end{table}

\newpage
\subsection*{Saving and Exiting}
\begin{table}[h!]
\begin{tabular}{|l|l|}
\hline
\textbf{Key(s)} & \textbf{Action} \\
\hline
\texttt{:w} & Write (save) changes \\
\texttt{:q} & Quit if no unsaved changes \\
\texttt{:wq} or \texttt{:x} & Save and quit \\
\texttt{:q!} & Quit \textbf{without} saving \\
\texttt{ZZ} & Save changes (if any) and quit (another shortcut) \\
\hline
\end{tabular}
\end{table}

\subsection*{4. Searching and Substituting in vi}
\begin{itemize}
    \item \textbf{Search:} Press \texttt{/} in normal mode, type a pattern, press \textbf{Enter}.
    \item \textbf{Backward search:} Press \texttt{?}, type a pattern, press \textbf{Enter}.
    \item \textbf{Substitute} (similar to \texttt{sed}):
\begin{lstlisting}[language=bash]
:s/old/new/g
\end{lstlisting}
    \begin{itemize}
        \item Replaces all occurrences of "old" with "new" in current line.
        \item \texttt{:\%s/old/new/g} → entire file.
    \end{itemize}
\end{itemize}


\subsection*{5. Alternative Editors}
\begin{enumerate}
    \item \textbf{vim}
    \begin{itemize}
        \item “vi improved,” features syntax highlighting, multiple undo levels, plugins, etc.
    \end{itemize}

    \item \textbf{nano}
    \begin{itemize}
        \item Simple, user-friendly. Insert mode by default, commands are \textbf{Ctrl} + key combos.
        \item Common shortcuts:
        \begin{itemize}
            \item \textbf{Ctrl + O} to save, \textbf{Ctrl + X} to exit, \textbf{Ctrl + K} to cut line, \textbf{Ctrl + U} to paste.
        \end{itemize}
    \end{itemize}

    \item \textbf{Emacs}
    \begin{itemize}
        \item Very powerful, integrates IDE features.
        \item Commands use \textbf{Ctrl} + key or \textbf{Alt} + key combos.
    \end{itemize}
\end{enumerate}



\subsection*{6. Setting the Default Editor}
\begin{itemize}
    \item The shell checks \texttt{EDITOR} or \texttt{VISUAL} environment variables to decide which editor to use for tasks like \texttt{crontab -e} or \texttt{git commit}.
    \item Example:
\begin{lstlisting}[language=bash]
export EDITOR=nano
\end{lstlisting}
    \item Add to \texttt{\~\ /.bash\_profile} or similar to make persistent.
\end{itemize}

\section*{Exercises}
\begin{enumerate}
    \item \textbf{Basic vi}
    \begin{itemize}
        \item Create a test file: \texttt{vi test.txt}.
        \item Press \texttt{i}, enter some text. Press \texttt{Esc}.
        \item Move around with \texttt{h}, \texttt{j}, \texttt{k}, \texttt{l}.
        \item Delete a line with \texttt{dd}.
        \item Save and exit: \texttt{:wq}.
    \end{itemize}
    
    \item \textbf{Navigation and Searching}
    \begin{itemize}
        \item Use \texttt{:set number} inside vi to see line numbers.
        \item Jump to line 5 with \texttt{5G}.
        \item Search for "test" using \texttt{/test}.
        \item Press \texttt{n} to find the next match.
    \end{itemize}
    
    \item \textbf{Substitutions}
    \begin{itemize}
        \item In normal mode, type \texttt{:s/abc/xyz/g} to replace "abc" with "xyz" on the current line.
        \item Apply to the entire file: \texttt{:\%s/abc/xyz/g}.
    \end{itemize}
    
    \item \textbf{nano}
    \begin{itemize}
        \item Run \texttt{nano test.txt}.
        \item Type some lines, press \texttt{Ctrl + K} to cut a line, \texttt{Ctrl + U} to paste.
        \item Save with \texttt{Ctrl + O}, exit with \texttt{Ctrl + X}.
    \end{itemize}
    
    \item \textbf{Default Editor}
    \begin{itemize}
        \item \texttt{export EDITOR=vim}
        \item Then \texttt{crontab -e} should open in vim.
        \item Make it permanent in \texttt{\~/.bashrc} or \texttt{\~/.bash\_profile}.
    \end{itemize}
\end{enumerate}

\section*{Summary}
\begin{itemize}
    \item \textbf{vi} (or vim) is a modal text editor: Normal mode (commands) and Insert mode.
    \item Basic movement: \texttt{h}, \texttt{j}, \texttt{k}, \texttt{l}; insertion: \texttt{i}, \texttt{a}, \texttt{o}; deleting: \texttt{d}, \texttt{x}; copying: \texttt{y}; pasting: \texttt{p}.
    \item Searching with \texttt{/pattern}; substituting with \texttt{:s/old/new/g}.
    \item Alternative editors: \textbf{nano}, \textbf{Emacs}, \textbf{vim} with advanced features.
    \item Shell environment variable \texttt{EDITOR} (or \texttt{VISUAL}) sets the default text editor for tasks like \texttt{crontab -e}.
\end{itemize}



%-------------------------------------------------------
% Multiple-Choice Questions for (103.8)
%-------------------------------------------------------

%-------------------------------------------------------
% Fill-in-the-Blank Questions (103.8)
%-------------------------------------------------------


%=======================================================
% TOPIC 104: DEVICES, LINUX FILESYSTEMS, FHS
%=======================================================
\newpage
\chapter{Topic 104: Devices, Linux Filesystems, Filesystem Hierarchy Standard}

\section{104.1 Create partitions and filesystems}
\textit{[Brief syllabus and questions to be added here]}

\section{104.2 Maintain the integrity of filesystems}
\textit{[Brief syllabus and questions to be added here]}

\section{104.3 Control mounting and unmounting of filesystems}
\textit{[Brief syllabus and questions to be added here]}

\section{104.5 Manage file permissions and ownership}
\textit{[Brief syllabus and questions to be added here]}

\subsection*{104.5 Lesson 1}
\textit{[Brief syllabus and questions to be added here]}

\section{104.6 Create and change hard and symbolic links}
\textit{[Brief syllabus and questions to be added here]}

\section{104.7 Find system files and place files in the correct location}
\textit{[Brief syllabus and questions to be added here]}

%=======================================================
% TOPIC 105: SHELLS AND SHELL SCRIPTING
%=======================================================
\chapter{Topic 105: Shells and Shell Scripting}
\section{105.1 Customize and use the shell environment}
\textit{[Brief syllabus and questions to be added here]}

\section{105.2 Customize or write simple scripts}
\textit{[Brief syllabus and questions to be added here]}

%=======================================================
% TOPIC 106: USER INTERFACES AND DESKTOPS
%=======================================================
\chapter{Topic 106: User Interfaces and Desktops}
\section{106.1 Install and configure X11}
\textit{[Brief syllabus and questions to be added here]}

\section{106.2 Graphical Desktops}
\textit{[Brief syllabus and questions to be added here]}

\section{106.3 Accessibility}
\textit{[Brief syllabus and questions to be added here]}

%=======================================================
% TOPIC 107: ADMINISTRATIVE TASKS
%=======================================================
\chapter{Topic 107: Administrative Tasks}
\section{107.1 Manage user and group accounts and related system files}
\textit{[Brief syllabus and questions to be added here]}

\section{107.2 Automate system administration tasks by scheduling jobs}
\textit{[Brief syllabus and questions to be added here]}

\section{107.3 Localisation and internationalisation}
\textit{[Brief syllabus and questions to be added here]}

%=======================================================
% TOPIC 108: ESSENTIAL SYSTEM SERVICES
%=======================================================
\chapter{Topic 108: Essential System Services}
\section{108.1 Maintain system time}
\textit{[Brief syllabus and questions to be added here]}

\section{108.2 System logging}
\textit{[Brief syllabus and questions to be added here]}

\section{108.3 Mail Transfer Agent (MTA) basics}
\textit{[Brief syllabus and questions to be added here]}

\section{108.4 Manage printers and printing}
\textit{[Brief syllabus and questions to be added here]}

%=======================================================
% TOPIC 109: NETWORKING FUNDAMENTALS
%=======================================================
\chapter{Topic 109: Networking Fundamentals}
\section{109.1 Fundamentals of internet protocols}
\textit{[Brief syllabus and questions to be added here]}

\section{109.2 Persistent network configuration}
\textit{[Brief syllabus and questions to be added here]}

\section{109.3 Basic network troubleshooting}
\textit{[Brief syllabus and questions to be added here]}

\section{109.4 Configure client side DNS}
\textit{[Brief syllabus and questions to be added here]}

%=======================================================
% TOPIC 110: SECURITY
%=======================================================
\chapter{Topic 110: Security}
\section{110.1 Perform security administration tasks}
\textit{[Brief syllabus and questions to be added here]}

\section{110.2 Setup host security}
\textit{[Brief syllabus and questions to be added here]}

\section{110.3 Securing data with encryption}
\textit{[Brief syllabus and questions to be added here]}

%-------------------------------------------------------
% ANSWERS SECTION
%-------------------------------------------------------
\clearpage

\chapter*{Answers} 
\addcontentsline{toc}{chapter}{Answers} 

%=======================================================
% ANSWERS FOR TOPIC 101
%=======================================================
\section*{Topic 101: System Architecture}
\addcontentsline{toc}{section}{Topic 101: System Architecture}

%-------------------------------------------------------
% Answer-Sheet (101.1)
%-------------------------------------------------------


\subsection*{101.1 Determine and Configure Hardware Settings}
\subsubsection*{Multiple-Choice Questions (101.1)}
\begin{enumerate}[1.]
    \item A
    \item C
    \item D
    \item B
    \item A
    \item B
    \item C
    \item A
    \item C
    \item B
    \item B
    \item C
    \item C
    \item D
    \item B
    \item C
    \item C
    \item C
    \item D
    \item C
    \end{enumerate}




\subsubsection*{Fill-in-the-Blank Questions (101.1)}
\begin{enumerate}[1.]
    \item BIOS
    \item lsmod
    \item driver
    \item udev
    \item /proc
    \item BIOS
    \item SCSI
    \item modprobe
    \item /etc/modprobe.d
    \item lsusb
    \end{enumerate}



%-------------------------------------------------------
% Answer-Sheet (101.2)
%-------------------------------------------------------


    \subsection*{101.2 Boot the System}
    \subsubsection*{Multiple-Choice Questions (101.2)}
\begin{enumerate}[1.]
    \item C
    \item A
    \item D
    \item B
    \item A
    \item C
    \item B
    \item B
    \item B
    \item D
    \item C
    \item C
    \item D
    \item A
    \item D
    \item A
    \item D
    \item C
    \item A
    \item B
    \end{enumerate}


\subsubsection*{Fill-in-the-Blank Questions (101.2)}
\begin{enumerate}[1.]
    \item BIOS, UEFI
    \item 440, MBR
    \item EFI System Partition (ESP)
    \item init
    \item Kernel parameters
    \item regenerate
    \item kernel ring buffer
    \item POST (Power-On Self-Test)
    \item inittab
    \item daemon
    \end{enumerate}

%-------------------------------------------------------
% Answer-Sheet (101.3)
%-------------------------------------------------------


    \subsection*{101.3 Change Runlevels / Boot Targets and Shutdown or Reboot System}
    \subsubsection*{Multiple-Choice Questions (101.3)}
    \begin{enumerate}[1.]
    \item A
    \item A
    \item B
    \item C
    \item A
    \item C
    \item D
    \item B
    \item C
    \item A
    \item B
    \item D
    \item C
    \item A
    \item B
    \item D
    \item C
    \item B
    \item A
    \item D
    \end{enumerate}



\subsubsection*{Fill-in-the-Blank Questions (101.3)}
    \begin{enumerate}[1.]
    \item /etc/inittab
    \item telinit 1 or telinit s
    \item telinit q
    \item /etc/init.d/
    \item unit
    \item 
        \begin{itemize}
            \item multi-user.target
            \item graphical.target
            \item (Any valid systemd target name is acceptable here.)
        \end{itemize}
    \item isolate
    \item systemd
    \item wall
    \item shutdown
    \end{enumerate}

%=======================================================
% ANSWERS FOR TOPIC 102
%=======================================================

\section*{Topic 102: Linux Installation and Package Management}
\addcontentsline{toc}{section}{Topic 102: Linux Installation and Package Management}
\begin{enumerate}[1.]
    \item D
    \item B
    \item C
    \item A
    \item D
    \item A
    \item B
    \item C
    \item A
    \item D
    \item B
    \item D
    \item A
    \item C
    \item B
    \item D
    \item C
    \item D
    \item B
    \item A
\end{enumerate}

\
%-------------------------------------------------------
% Answer-Sheet (102.1)
%-------------------------------------------------------
    
\subsection*{102.1  Design Hard Disk Layout}
\subsubsection*{Multiple-Choice Questions (102.1)}
\begin{enumerate}[1.]
    \item C
    \item A
    \item D
    \item B
    \item B
    \item D
    \item A
    \item C
    \item B
    \item C
    \item A
    \item D
    \item C
    \item B
    \item D
    \item A
    \item D
    \item C
    \item B
    \item A
\end{enumerate}


\subsubsection*{Fill-in-the-Blank Questions (102.1)}
\begin{enumerate}[1.]
    \item /home
    \item /boot
    \item partition
    \item EFI System (or EFI System Partition / ESP)
    \item mkswap
    \item /var
    \item mounting
    \item Volume Group (VG)
    \item swap file
    \item LVM (Logical Volume Management)
\end{enumerate}

%-------------------------------------------------------
% Answer-Sheet (102.2)
%-------------------------------------------------------
\subsection*{102.2 Install a Boot Manager}
\subsubsection*{Multiple-Choice Questions (102.2)}
\begin{enumerate}[1.]
    \item D
    \item B
    \item C
    \item A
    \item D
    \item C
    \item B
    \item B
    \item D
    \item D
    \item A
    \item B
    \item C
    \item A
    \item D
    \item A
    \item B
    \item A
    \item C
    \item C
\end{enumerate}


\subsubsection*{Fill-in-the-Blank Questions (102.2)}
\begin{enumerate}[1.]
    \item boot loader
    \item Master Boot Record (MBR)
    \item vmlinuz
    \item initrd.img (or initial RAM disk)
    \item System.map
    \item /etc/default/grub
    \item update-grub (or grub-mkconfig)
    \item /boot
    \item /boot/grub/menu.lst
    \item chainloading
\end{enumerate}
%-------------------------------------------------------
% Answer-Sheet (102.3)
%-------------------------------------------------------

\subsection*{102.3 Manage Shared Libraries}
\subsubsection*{Multiple-Choice Questions (102.3)}

\subsubsection*{Fill-in-the-Blank Questions (102.3)}
\begin{enumerate}[1.]

    \item ldd
    \item LD\_LIBRARY\_PATH
    \item /etc/ld.so.conf.d
    \item static
    \item LD\_LIBRARY\_PATH
    \item soname
    \item ldconfig
    \item sudo ldconfig
    \item shared libraries
    \item .a
\end{enumerate}


%-------------------------------------------------------
% Answer-Sheet (102.4)
%-------------------------------------------------------

\subsection*{102.4 Use Debian Package Management}
\subsubsection*{Multiple-Choice Questions (102.4)}
\begin{enumerate}[1.]
    \item D
    \item C
    \item D
    \item B
    \item C
    \item B
    \item C
    \item D
    \item A
    \item B
    \item A
    \item A
    \item D
    \item A
    \item D
    \item A
    \item B
    \item C
    \item B
    \item C
\end{enumerate}


\subsubsection*{Fill-in-the-Blank Questions (102.4)}
\begin{enumerate}[1.]
\item selections
\item sources
\item -f
\item information
\item cache
\item files
\item \#
\item cache
\item clean
\item purge
\end{enumerate}
%-------------------------------------------------------
% Answer-Sheet (102.5)
%-------------------------------------------------------
\subsection*{102.5 Use RPM and YUM Package Management}
\subsubsection*{Multiple-Choice Questions (102.5)}
\begin{enumerate}[1.]
    \item B
    \item D
    \item A
    \item C
    \item D
    \item B
    \item A
    \item A
    \item C
    \item B
    \item D
    \item B
    \item A
    \item D
    \item C
    \item D
    \item B
    \item A
    \item C
    \item C
\end{enumerate}


\subsubsection*{Fill-in-the-Blank Questions (102.5)}
\begin{enumerate}[1.]
    \item -e
    \item apt
    \item search, se
    \item remove
    \item /etc/yum.conf
    \item all
    \item check-update
    \item info
    \item dnf
    \item -ql
\end{enumerate}

%-------------------------------------------------------
% Answer-Sheet (102.6)
%-------------------------------------------------------

\subsection*{102.6 Linux as a virtualization guest}
\subsubsection*{Multiple-Choice Questions (102.6)}
\begin{enumerate}[1.]
    \item D
    \item C
    \item D
    \item D
    \item A
    \item D
    \item D
    \item C
    \item B
    \item B
    \item B
    \item C
    \item B
    \item A
    \item A
    \item B
    \item C
    \item C
    \item A
    \item A
\end{enumerate}


\subsubsection*{Fill-in-the-Blank Questions (102.6)}
\begin{enumerate}[1.]
    \item hypervisor
    \item paravirtualized
    \item qcow2
    \item /etc/libvirt/qemu
    \item dbus-uuidgen --ensure
    \item /etc/machine-id
    \item template
    \item cloud-init
    \item VirtualBox
    \item Live migration
\end{enumerate}


%=======================================================
% ANSWERS FOR TOPIC 103
%=======================================================

\section*{Topic 103: GNU and Unix Commands}
\addcontentsline{toc}{section}{Topic 103: GNU and Unix Commands}

%-------------------------------------------------------
% Answer-Sheet (103.1)
%-------------------------------------------------------

\subsection*{103.1 Work on the command line}
\subsubsection*{Multiple-Choice Questions (103.1)}

\begin{enumerate}[1.]
    \item D
    \item B
    \item D
    \item C
    \item B
    \item A
    \item A
    \item A
    \item B
    \item B
    \item B
    \item A
    \item C
    \item B
    \item B
    \item D
    \item A
    \item D
    \item D
    \item D
\end{enumerate}


\subsubsection*{Fill-in-the-Blank Questions (103.1)}
%-------------------------------------------------------
% Answer-Sheet (103.2)
%-------------------------------------------------------

\subsection*{103.2 Process text streams using filters}
\subsubsection*{Multiple-Choice Questions (103.2)}

\begin{enumerate}[1.]
    \item D
    \item A
    \item C
    \item A
    \item B
    \item A
    \item D
    \item C
    \item D
    \item C
    \item D
    \item A
    \item B
    \item C
    \item B
    \item C
    \item A
    \item B
    \item B
    \item D
\end{enumerate}


\subsubsection*{Fill-in-the-Blank Questions (103.2)}
%-------------------------------------------------------
% Answer-Sheet (103.3)
%-------------------------------------------------------
\subsection*{103.3 Perform basic file management}
\subsubsection*{Multiple-Choice Questions (103.3)}

\begin{enumerate}[1.]
    \item A
    \item C
    \item A
    \item B
    \item D
    \item D
    \item A
    \item B
    \item A
    \item A
    \item C
    \item D
    \item C
    \item A
    \item C
    \item D
    \item B
    \item B
    \item A
    \item B
\end{enumerate}

\subsubsection*{Fill-in-the-Blank Questions (103.3)}

%-------------------------------------------------------
% Answer-Sheet (103.4)
%-------------------------------------------------------

\subsection*{103.4 Use streams, pipes and redirects}
\subsubsection*{Multiple-Choice Questions (103.4)}

\begin{enumerate}[1.]
    \item C
    \item D
    \item A
    \item B
    \item B
    \item B
    \item C
    \item C
    \item C
    \item C
    \item A
    \item C
    \item B
    \item A
    \item B
    \item B
    \item B
    \item A
    \item C
    \item B
\end{enumerate}

\subsubsection*{Fill-in-the-Blank Questions (103.4)}

%-------------------------------------------------------
% Answer-Sheet (103.5)
%-------------------------------------------------------

\subsection*{103.5 Create, monitor and kill processes}
\subsubsection*{Multiple-Choice Questions (103.5)}
\begin{enumerate}[1.]
    \item B
    \item A
    \item C
    \item A
    \item D
    \item C
    \item A
    \item B
    \item B
    \item B
    \item C
    \item C
    \item A
    \item C
    \item D
    \item B
    \item B
    \item B
    \item D
    \item B
\end{enumerate}


\subsubsection*{Fill-in-the-Blank Questions (103.5)}
%-------------------------------------------------------
% Answer-Sheet (103.6)
%-------------------------------------------------------

\subsection*{103.6 Modify process execution priorities}
\subsubsection*{Multiple-Choice Questions (103.6)}
\begin{enumerate}[1.]
    \item B
    \item C
    \item C
    \item C
    \item D
    \item B
    \item D
    \item A
    \item C
    \item B
    \item A
    \item A
    \item C
    \item D
    \item B
    \item C
    \item C
    \item C
    \item A
    \item C
\end{enumerate}



\subsubsection*{Fill-in-the-Blank Questions (103.5)}
%-------------------------------------------------------
% Answer-Sheet (103.7)
%-------------------------------------------------------

%-------------------------------------------------------
% Answer-Sheet (103.8)
%-------------------------------------------------------

%=======================================================
% ANSWERS FOR TOPIC 104
%=======================================================

\section*{Topic 104: Devices, Linux Filesystems, Filesystem Hierarchy Standard}
\addcontentsline{toc}{section}{Topic 104: Devices, Linux Filesystems, Filesystem Hierarchy Standard}

%-------------------------------------------------------
% Answer-Sheet (104.1)
%-------------------------------------------------------

%-------------------------------------------------------
% Answer-Sheet (104.2)
%-------------------------------------------------------

%-------------------------------------------------------
% Answer-Sheet (104.3)
%-------------------------------------------------------

%-------------------------------------------------------
% Answer-Sheet (104.4)
%-------------------------------------------------------

%-------------------------------------------------------
% Answer-Sheet (104.5)
%-------------------------------------------------------

%-------------------------------------------------------
% Answer-Sheet (104.6)
%-------------------------------------------------------

%-------------------------------------------------------
% Answer-Sheet (104.7)
%-------------------------------------------------------

%=======================================================
% ANSWERS FOR TOPIC 105
%=======================================================

\section*{Topic 105: Shells and Shell Scripting}
\addcontentsline{toc}{section}{Topic 105: Shells and Shell Scripting}

%-------------------------------------------------------
% Answer-Sheet (105.1)
%-------------------------------------------------------

%-------------------------------------------------------
% Answer-Sheet (105.2)
%-------------------------------------------------------

%=======================================================
% ANSWERS FOR TOPIC 106
%=======================================================

\section*{Topic 106: User Interfaces and Desktops}
\addcontentsline{toc}{section}{Topic 106: User Interfaces and Desktops}

%-------------------------------------------------------
% Answer-Sheet (106.1)
%-------------------------------------------------------

%-------------------------------------------------------
% Answer-Sheet (106.2)
%-------------------------------------------------------

%-------------------------------------------------------
% Answer-Sheet (106.3)
%-------------------------------------------------------

%=======================================================
% ANSWERS FOR TOPIC 107
%=======================================================

\section*{Topic 107: Administrative Tasks}
\addcontentsline{toc}{section}{Topic 107: Administrative Tasks}

%-------------------------------------------------------
% Answer-Sheet (107.1)
%-------------------------------------------------------

%-------------------------------------------------------
% Answer-Sheet (107.2)
%-------------------------------------------------------

%-------------------------------------------------------
% Answer-Sheet (107.3)
%-------------------------------------------------------

%=======================================================
% ANSWERS FOR TOPIC 108
%=======================================================

\section*{Topic 108: Essential System Services}
\addcontentsline{toc}{section}{Topic 108: Essential System Services}

%-------------------------------------------------------
% Answer-Sheet (108.1)
%-------------------------------------------------------

%-------------------------------------------------------
% Answer-Sheet (108.2)
%-------------------------------------------------------

%-------------------------------------------------------
% Answer-Sheet (108.3)
%-------------------------------------------------------

%-------------------------------------------------------
% Answer-Sheet (108.4)
%-------------------------------------------------------

%=======================================================
% ANSWERS FOR TOPIC 109
%=======================================================

\section*{Topic 109: Networking Fundamentals}
\addcontentsline{toc}{section}{Topic 109: Networking Fundamentals}

%-------------------------------------------------------
% Answer-Sheet (109.1)
%-------------------------------------------------------

%-------------------------------------------------------
% Answer-Sheet (109.2)
%-------------------------------------------------------

%-------------------------------------------------------
% Answer-Sheet (109.3)
%-------------------------------------------------------

%-------------------------------------------------------
% Answer-Sheet (109.4)
%-------------------------------------------------------

%=======================================================
% ANSWERS FOR TOPIC 110
%=======================================================

\section*{Topic 110: Security}
\addcontentsline{toc}{section}{Topic 110: Security}

%-------------------------------------------------------
% Answer-Sheet (110.1)
%-------------------------------------------------------

%-------------------------------------------------------
% Answer-Sheet (110.2)
%-------------------------------------------------------

%-------------------------------------------------------
% Answer-Sheet (110.3)
%-------------------------------------------------------


\end{document}
